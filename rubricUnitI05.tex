\documentclass[a4j]{jreport}
\input rubricHead.tex
\input rubricPresentation.tex
\input rubricUnitIHead.tex
\begin{document}
\setcounter{chapter}{4}
\chapter{イベント処理入門}
\setDate{5/23}
\Probs{\Must}{演習のビデオ1を見て次の問いに答えよ。}{
{a}{0}
}
\Probs{\Must}{演習のビデオ2を見て次の問いに答えよ。}{
{a}{0}
}
\Probs{a}{演習のビデオ3を見て次の問いに答えよ。}
{
{a}{0}
}
\Rubric{第\arabic{chapter}回(\Date)}{ノートの内容}{
今回からJavaScriptによるプログラミングが始まる。細かい文法の説明を特には
しないので今までのプログラミング言語と比較して違いに気を付けること。
\newline
ルーブリック表の項目の最後の文字はそれぞれの項目の評価である。
リ(プログラム等のリスト)、説(プログラ
ム説明が手書きまたは印刷である)、図(結果のキャプチャ画面)、考( 考察が手
書きまたは印刷である)を意味し、次の記号で評価を示す。
$\times$(不備)、$\triangle$(もう一息)、$\bigcirc$(良い)、
$\circledcirc$(大変良い)
}
{{課題1}{20}
{
  {使用中のブラウザと「開発者ツール」の開き方\ResultFI}
  {SVGファイルに対して要素の属性を
  直接変えた結果に前後の図に開発者ツールの「Elements」タブが表示\ResultA}
  {開発者ツールのコンソールで直接、簡単な算術式や
  \texttt{document}\newline\texttt{.getElementsByTagName}を実行\ResultA}
}
{
  {使用中のブラウザと「開発者ツール」の開き方の図または考察がない\ResultFI}
  {SVGファイルに対して要素の属性を
  直接変えた結果に前後の図に開発者ツールの「Elements」タブがない\ResultA}
  {開発者ツールのコンソールで簡単な算術式や
  \texttt{document}\newline\texttt{.getElementsByTagName}を実行が一部な
  い\ResultA}
}
{
  {使用中のブラウザと「開発者ツール」の開き方の図と考察がないか不十分\ResultFI}
  {SVGファイルに対して要素の属性を
  直接変えた結果に前後の図のいずれかがないか開発者ツールの「Elements」タ
  ブがなく不十分\ResultA}
  {開発者ツールのコンソールで簡単な算術式や
  \texttt{document}\newline\texttt{.getElementsByTagName}を実行がな
  い\ResultA}
}
 {課題2}{30}
 {
 {クリックイベントの処理関数を\texttt{window.onload}で
 登録し、クリックされた円の塗りつぶし以外の属性を表示している。\ResultA}
 {\Must 図形の種類をを変えて、クリックしたとき、複数の属性値をテンプレー
 トリテラルの形式で表示している。\ResultA}
 {\Must\texttt{alert}の代わりに\texttt{console.log}を用いた実行結果があ
 り、\texttt{alert}を使用した場合との違いがある。 \ResultFI}
 {\Must 正方形をいくつか置いてクリックすると移動する。\ResultA}
 {正方形をいくつか置いてクリックすると色が変化する。\ResultA}
 {異なる種類の要素を置いてクリックすると位置が変化する。\ResultA}
 }
 {
 {イベントの処理関数を\texttt{window.onload}で
 登録していない。クリックされた円の塗りつぶし以外の属性を表示している。\ResultA}
 {\Must 図形の種類をを変えて、クリックしたとき、複数の属性値をテンプレー
 トリテラルの形式で一部しか表示していない。\ResultA}
 {\Must\texttt{alert}の代わりに\texttt{console.log}を用いた実行結果があ
 り、\texttt{alert}を使用した場合との違いの考察が不十分 \ResultFI}
 {\Must 単独の正方形を置いてクリックすると移動する。\ResultA}
 {単独の正方形をでクリックすると色が変化する。\ResultA}
 {異なる種類の要素に対してクリックすると位置が変化するようになっていない。\ResultA}
 }
 {
 {}
 }
 {課題3}{25}
 {
 {\Must 画面全体を覆う長方形の属性\texttt{fill}を\texttt{none}にした報告
 \ResultEI}
 {\Must 円にイベント処理関数を登録してその上をクリックしたときに移動す
 る。\ResultA}
 {2つ以上の要素を置き、要素以外のところでクリックしたら最後にクリッ
        クした要素がクリックした位置に移動する。\ResultA}
 {\Must 円の上をクリックすると円の色を表示(3)の動作確認\ResultA}
 {\texttt{svg}要素の属性\texttt{id}を\texttt{Canvas}にした時の動作\ResultA}
 {画面全体を覆う長方形なしで、円上も含めてクリックした位置を円の中心にする\ResultA}
 }
 {
 {}
 }
 {
 {}
 }{課題4}{25}
 {
 {\Must ビデオ内の「マウスのドラッグを処理」の動作の気になる点\ResultFI}
{\Must ビデオ内の「マウスのドラッグを処理(改良版)」で前問の気に
なる点が修正の確認とDOMツリーの変化の確認\ResultA}
{\Must 正方形をドラッグするようにしなさい。}
{Windows 上でアイコンのドラッグとの相違点と改善\ResultA}
{円と正方形の図形に対し、1種類のイベント処理関数でする方法
			 図形の種類が増えてもイベント処理関数に手を付けないようにするには
			 どのような方法の検討\ResultA}
 }
 {
 {}
 }
 {
 {}
 }
}

\rublicPresenII{第\thechapter 回(\Date)}

\end{document}