\input ../beamerHead.tex
\TITLE{8}{1}{WebStorage--\JSKey{localStorage}}{6/13}
\begin{document}
\frame{\maketitle}
%\frame{\tableofcontents}
\section{Web ページのデータの保存}
\begin{frame}[containsverbatim]
\frametitle{ブラウザでのデータの保存}
\begin{itemize}
 \item Web ページ上で作成したデータをそのままブラウザが動作しているコンピュータ
上に保存することはセキュリティ上できない。
 \item ブラウザが管理する領域に保存することは可能
 \item 昔からあるのがクッキー
			 \begin{itemize}
				\item クライアントとサーバーの間で共有して常にクライアントからサー
							バーに提供
				\item 小さなデータしか扱えない
				\item サーバーとクライアントとのやりがあるたびに送られる
			 \end{itemize}
 \item HTML5ではブラウザーのあるページ以降のページにだけ
存在する\JSKey{sessionStorage}とページが閉じられてもデータが存続できる
\JSKey{localStorage}が提案
\end{itemize}
\end{frame}
\section{\texttt{localStorage}の利用}
\section{やってみよう}
\begin{frame}[containsverbatim]
\frametitle{やってみよう}
次のことを確かめよ。
\begin{itemize}
 \item ブラウザをいったん閉じてから再度ページを開くと、最後に設定された
       値がテキストボックスに表示
 \item \JSKey{localStorage}の値を確認
 \item 新しいフォルダを作成し、そのフォルダにファイルをコピーして、それ
       を表示させたときにどうなるか
 \item 9行目のコメントを外したらときに上と同じことを行う。
\end{itemize} 
\end{frame}
 \end{document}
\begin{frame}[containsverbatim]
\frametitle{}
\end{frame}
