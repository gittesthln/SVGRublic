\documentclass[a4j]{jreport}
\input ../rubricHead.tex
\input ../rubricPresentation.tex
\input ../rubricUnitIHead2.tex
\begin{document}
\setcounter{chapter}{9}
\chapter{Ajaxによる非同期通信}
\changePage{6/27}
今回の演習の目的は次のとおりである。
\begin{itemize}
 \item データをサーバーから非同期通信で得て、ページの一部を書き直すAjax
       の基本的な使い方を理解する。
 \item 同期通信と非同期通信の理解
 \item データを利用したページの作成法
\end{itemize}
課題に\Must と書かれたものを最低行うこと。それ以外の課題は
いくつか選択してよい。
\Probs{PHPの基礎}{演習のビデオ1を見て次の問いに答えよ。}{
 {次のことについて報告をする。
 \begin{itemize}
	\item \Must 正$n$角形の頂点の位置だけをサーバーから取得するプログラム
				を実行する。
	\item \Must 何回か辺の数が異なる正多角形を表示後、ブラウザの「戻る」ボ
				タンそ押したときに起こること
	\item \Must ブラウザのアドレスバーに直接\Verb+svg-polygon-ajax.php?N=5+と入
        力するとなにが起こるか。
	\item アドレスバーに\Verb+svg-polygon-ajax.php?N=2+としたら何が起こる
				か。問題点を指摘し、修正する。
  \item テキストボックスを追加して、サーバーに図形の大きさも渡すようにす
        る。複数の値を渡す時には渡すパラメータ名と値の間を\texttt{\&}で
        つなげる。
\end{itemize}
 }{0}
 }
%  \newpage
 \newcommand{\VV}[1]{\Verb+"#1"+}
 \newcommand{\VVV}[1]{\Verb+#1+}
\Probs{Ajaxによる同期通信と非同期通信の違い}{演習のビデオ2を見て次の問いに答えよ。}{
 {
 \begin{itemize}
	\item \Must 「素数の数を数える」を自分のサーバーで実行した結果を報告す
				る。同期通信と非同期通信の両方を報告すること。非同期通信では2回
				以上実行すること。
	\item 素数を求める区間を$10,000,000$まで広げるようにする。求める範囲の
				最大値や区間の幅をメニューから変更できるとよい。
	\item 非同期通信で起こる結果の表示に不具合があれば直す。
 \end{itemize}
 }{0}
 }
 \newpage
 \Probs{この演習のHP}{演習のビデオ3を見て次の問いに答えよ。}{
 {
 \begin{itemize}
	\item \Must このリストで使用されている、ファイルに関するPHPの関数のリストと簡単な説
				明をする。
	\item \Must \texttt{date}関数の仕様を調べて、機能を一覧表にする。
 \end{itemize}}{0}
{データをもとに簡単なHPを作成するPHPまたはJavaScriptのプログラム
				を作成する。(例:国民の休日の日付と名称を組みにしたデータから、
				一覧を作成する。データはテキストでも配列でも構わない。)}{0}
 }
%\newpage
\RubricN{第9回(6/20)}{ノートの内容}{
\GradeLegend
}
{
{課題1}{30}
{
  {「正$n$角形の頂点の位置だけをサーバーから取得するプログラムを実行する」
	の報告がすべてあり、考察も正しい。}
  {アドレスバーに直接\Verb+svg-polygon-ajax.php?N=5+と乳録した
	ときに起きる結果の報告が正しい。考察も十分にある。}
	{テキストボックスを追加して、他の要素もサーバーに渡して結果を得るプロ
	グラムが正しく実行されている。考察も十分にある。}
}
{
  {「正$n$角形の頂点の位置だけをサーバーから取得するプログラムを実行する」
	の報告で複数回実行したことをコンソールからわかるようにしている。}
	{複数回実行後、「戻る」ボタンを押したときの画面の説明が不十分である。}
  {アドレスバーに直接\Verb+svg-polygon-ajax.php?N=5+と乳録した
	ときに起きる結果の報告が一部正しくない。考察が少し足りない。}
	{テキストボックスを追加して、他の要素もサーバーに渡して結果を得るプロ
	グラムの戻り値にJSONを利用していない。考察が不十分である。}
}
{
  {「正$n$角形の頂点の位置だけをサーバーから取得するプログラムを実行する」
	の報告で複数回実行していることを示す図が足りないか、足りない。}
	{複数回実行後、「戻る」ボタンを押したときの画面の説明がない。}
  {アドレスバーに直接\Verb+svg-polygon-ajax.php?N=5+と乳録した
	ときに起きる結果の報告が正しくない。考察が足りないかない。}
	{テキストボックスを追加して、他の要素もサーバーに渡して結果を得るプロ
	グラムがない。}
}
{\ResultA}
{課題2}{40}
{
  {「素数の数を数える」を自分のサーバーで実行した結果の報告が十分ある。}
  {「素数の数を数える」について最大値や区間の幅を変更したものを自分のサー
	バーで実行した結果の報告が十分ある。}
	{非同期通信で起こる結果の表示に不具合が修正されている。}
	}
{
	{同期通信と非同期通信の両方の報告はあるが、非同期通信では2回
				以上実行していない。}
  {非同期通信と同期通信の違いの説明が不十分である。}
	{非同期通信で起こる結果の表示に不具合の修正が不十分である。}
}
{
	{同期通信と非同期通信の両方の報告のどちらかまたは両方ない。}
  {非同期通信と同期通信の違いの説明がないか、間違っている。}
	{非同期通信で起こる結果の表示に不具合の修正がないか間違っている。}
	}
	{\ResultEI}
{課題3}{30}
{
  {このリストで使用されている、ファイルに関するPHPの関数のリストと簡単な説明が十分にあ
	る。}
	{\texttt{date}関数の機能を一覧表の内容が十分にある。}
	{データをもとに簡単なHPを作成するPHPまたはJavaScriptのプログラムがあり、
	リストの説明、動作を示す図などが十分にある。}
}
{
  {このリストで使用されている、ファイルに関するPHPの関数のリストと簡単な
	説明が少し足りない。}
	{\texttt{date}関数の機能を一覧表の内容が少し足りない。}
	{データをもとに簡単なHPを作成するPHPまたはJavaScriptのプログラムがあり、
	リストの説明、動作を示す図などが少し足りない。}
}
{
  {このリストで使用されている、ファイルに関するPHPの関数のリストと簡単な
	説明がほとんどないか、説明に間違いが多い。}
	{\texttt{date}関数の機能を一覧表の内容が足りない。}
	{データをもとに簡単なHPを作成するPHPまたはJavaScriptのプログラムがない
	か、リストの説明、動作を示す図などがない。}
}
{\ResultA}
}
\rublicPresenP{第10回(6/27)}

\end{document}