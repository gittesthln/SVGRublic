\input ../beamerHead.tex
\TITLE{6}{3}{SVGとHTMLの間でデータを交換する}{5/30}
\begin{document}
\frame{\maketitle}
\section{SVGとHTMLの間でデータ交換}
\begin{frame}[containsverbatim]
 \frametitle{HTMLのソースコード(1)}
 \LISTN{ShowSetClickPos4.html}{1}{10}{\scriptsize}
 \begin{itemize}
	\item 6行目で外部JavaScriptファイルを読み込み
	\item 7行目で外部CSSファイルを読み込むための\ELM{link}\\
        CSSファイルの呼び出しが\ELM{style}でないことに注意
 \end{itemize}
\end{frame}
\begin{frame}[containsverbatim]
 \frametitle{HTMLのソースコード(2)--SVGの部分(解説)}
 \LISTN{ShowSetClickPos4.html}{11}{27}{\scriptsize}
 \begin{itemize}
  \item SVGの要素はHTML内の表の一部として現れる。
  \item SVGの大きさが設定されている(17行目)のが以前と異なる
  \item クリック範囲をわかりやすくするため、外枠を設定(24行目)
 \end{itemize}
\end{frame}
\begin{frame}[containsverbatim]
 \frametitle{ソースコード(3)--テキストボックス等の部分}
 \LISTN{ShowSetClickPos4.html}{28}{last}{\scriptsize}
 \begin{itemize}
  \item 円の中心を指定するテキストボックス(29目から32行目)
  \item 色を選択するためのプルダウンメニュー(34行目)
  \item 設定を実行するためのボタン(35行目)
 \end{itemize}
\end{frame}
\begin{frame}[containsverbatim]
 \frametitle{JavaScript ソースコード(1)--プルダウンメニューの作成}
 \LISTN{SSClickPos.js}{1}{9}{\scriptsize}
 HTMLファイルの34行目にある\ELM{select}に現れる\ELM{option}を作成
 \begin{itemize}
	\item 6行目で\ELM{select}を得ている
	\item それぞれの色に対して8行目で\ELM{option}を作成し、
        \ATTR{value}を配列のキーに設定
  \item 8行目でテキストノードを作成し、\ELM{option}の子要素に設定
 \end{itemize}
\end{frame}
\begin{frame}[containsverbatim]
 \frametitle{ソースコード(2)--初期化}
 \LISTN{SSClickPos.js}{10}{16}{\scriptsize}
 \begin{itemize}
	\item 10,11行目で円の中心の座標が入るHTML文書内のテキストボックスの要
				素を得ている
	\item 12行目では円の要素を得ている
	\item 13,14行目では円の中心の座標が入るSVG内のテキスト表示位置の要
				素を得ている
	\item 15,16行目ではSVG内の円の中心位置の座標をHTML内のテキストボック
				スに設定
 \end{itemize}
\end{frame}
\begin{frame}[containsverbatim]
 \frametitle{ソースコード(3)--初期化}
 \LISTN{SSClickPos.js}{17}{23}{\scriptsize}
\end{frame}
\begin{frame}[containsverbatim]
 \frametitle{ソースコード(3)--初期化(解説)}
 \begin{itemize}
	\item 17,18行目でSVG上と「設定」ボタンのクリックイベントの処理関数を
        設定
	\item 19行目ではSVG文書の、HTML文書内での位置を示す
				\texttt{BoundingClientRect}オブジェクトを得ている。
	\item イベントオブジェクトで渡されるクリックした位置はHTML文書内からの
        位置で、そのまま使うと円の位置がずれる。
	\item SVGは縁取りが\texttt{5px}あるように作成しているので20,21行目で
				その分も補正値に加えている
	\item 22行目で画面の情報などを更新する関数
				\texttt{refreah()}を呼んでいる。
 \end{itemize}
\end{frame}
\begin{frame}[containsverbatim]
 \frametitle{ソースコード(4)--イベント処理関数}
 \LISTN{SSClickPos.js}{24}{last}{\scriptsize}
\end{frame}
\begin{frame}[containsverbatim]
 \frametitle{ソースコード(4)--イベント処理関数(解説)}
\begin{itemize}
	\item 24行目から28行目では画面がクリックされたときの処理関数が定義

				イベントが起きた座標から補正値を引いた値をHTML内に設定し(25行目
				と26行目)その後画面の書き直しを実行
	\item 29行目から33行目で画面の情報をアップデートする関数を定義

				SVG内の座標表示と円の\ATTR{fill}の属性値と中心位置を変更
	\item 34行目から41行目はSVG内の座標位置の表示関数
 \end{itemize}
\end{frame}
 \begin{frame}[containsverbatim]
  \frametitle{スタイルシート(その1)}
  \LISTN{HTML.css}{1}{7}{\scriptsize}
  \begin{itemize}
   \item \texttt{.display}はクラス名が\texttt{display}の要素に適用
   \item ここでは表題の部分でフォントの大きさを指定
   \item \texttt{.textStyle}はSVG内の座標の表示の部分に適用
   \item フォントの大きさと文字の位置(右寄せ)を指定
  \end{itemize}
 \end{frame}
\begin{frame}[containsverbatim]
  \frametitle{スタイルシート(その2)}
 \LISTN{HTML.css}{8}{13}{\scriptsize}
 \ATTR{class}が\texttt{Cell}の要素は左側の\ELM{svg}と右側の\ELM{input}
 をそれぞれ内部に持つ\ELM{div}である。
 \begin{itemize}
  \item \texttt{display}は表示方法を指定。ここではひとつの文字のように取
        り扱うことを指定(上記の2つの要素が横に並ぶ)
  \item \texttt{vertical-align}は垂直方向の位置指定。ここでは中央部を基
        準線に配置することを指定
  \item \texttt{padding-left}は要素の左側の空白を指定
 \end{itemize}
 \end{frame}
\begin{frame}[containsverbatim]
  \frametitle{スタイルシート(その3)}
 \LISTN{HTML.css}{8}{last}{\scriptsize}
 \begin{itemize}
  \item \texttt{\#XP}は\ATTR{id}が\texttt{XP}の要素に対して適用
  \item ここでは数が入るので右寄せを指定
  \item \texttt{,}で並べるとそれぞれのセレクターに対して適用される
  \item 残りも同じ
 \end{itemize}
\end{frame}
 \section{やってみよう}
\begin{frame}[containsverbatim]
  \frametitle{やってみよう}
 \begin{itemize}
  \item 色の選択の種類を増やす
  \item HTML内の表題の部分のフォントの大きさ(\texttt{.display}の
        \texttt{font-size})を変えてから再起動してもクリックの位置と値が
        一致していることを確かめる。
  \item 画面を表示した後、ブラウザの横幅を変えて表題部分の行数が増えると
        クリックした位置と円の移動位置が一致しないことを確かめる。
  \item 上記の不具合を直す。
  \item SVG内に図形を追加し、HTML内に図形を選択するプルダウンメニューま
        たはラジオボタンを置き、クリックしたとき選択された図形が移動
        するようにする
 \end{itemize}
\end{frame}
\end{document}
\begin{frame}[containsverbatim]
 \frametitle{}
\end{frame}
