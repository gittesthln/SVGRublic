\newcommand{\rublicPresen}[1]{
  \Rubric{#1}{プレゼンテーション}
{この回の予習に基づいてグループ内で議論したことや
	工夫した点について報告する。
  \Member
}
{{発表技法}{20}{
	{はっきりと丁寧に説明していた。}
	{発表の際に聴衆の反応を確かめていた。}
	{間の取り方がよかった。}
	{決められた時間に近い範囲で行った。}
	{機材の設定や準備がすぐできていた。}
	}
	{
	{説明が途切れることが2,3か所あった。}
	{声が少し大きすぎたり小さすぎた。}
	{発表の際に聴衆の方をあまり見ていないか反応を確かめていなかった。}
	{決められた発表時間を少し外れた。}
	{機材の設定や準備に少し時間がかかった。}
	}
	{
	{声が小さすぎて聞き取れなかった。}
	{聴衆のほうを全く見ない、反応を無視して行った。}
	{間がない発表であった。}
	{発表時間が極端に短い、または長すぎた。}
	{手元の資料やPC画面を見て発表していた。}
	{機材の取り扱いや発表の準備がほとんどできていなかった。}
	}
	{発表構成}{30}
	{
	{初めに発表内容に関する概要があった。}
	{発表内容の順序に必然性があった}
	{各構成の部分のバランスが良かった。}
	{図や表を使い簡潔にまとめられていた。}
	{引用は適切である。}
	}
	{
	{文字だけの発表で、概略が少しつかみづらかった。}
	{図の内容が少し見づらかった。}
	{項目の内容の分量にばらつきが少しあった。}
	{スライドごとに情報の詳しさが一部異なりすぎていた。}
	{ページの分量が発表時間に対して少し足りない、または多すぎた。}
	}
	{
	{ほとんどのスライドで情報量が少なかった。}
	{一つのスライドに文字を詰めすぎていた。}
	{図が大きすぎたまたは小さすぎた。}
	{スライドごとに情報の詳しさが異なりすぎていた。}
	{スライドの内容が情報ごとにまとまっていなかった。}
	}
	{発表内容}{50}
	{
	{内容は適切であった。}
	{図の使い方がよかった。}
	{内容が自分の言葉で述べられていた。}
	{それぞれの項目の関連性とバランスがよかった。}
	{発表したいことが十分に説明されていた。}
	{自分の意見が明確であった。}
	}
	{
	{内容のごく一部に説明不足なところがあった。}
	{図が少なくて説明が少しわかりずらかった。}
	{内容に関して他からの引用が少し多かった。}
	{それぞれの項目の関連性に少し不十分なところがあった。}
	{発表内容の必要性の説明が少し足りなかった。}
	}
	{
	{内容が少なすぎる。}
	{図を使用していないのでわかりずらい。}
	{スライドの記述と発言内容に差がありすぎる。}
	{内容が多すぎて散漫である。}
	{内容が引用ばかりで自分でまとめた形跡がなかった。}
	{内容の説明が不十分であった。}
	}
}
}

\newcommand{\rublicPresenII}[1]{
  \Rubric{#1}{プレゼンテーション}
{この回の予習に基づいてグループ内で議論したことや
	工夫した点について報告する。
  \Member
}
{{発表技法}{20}{
	{はっきりと丁寧に説明していた。}
	{発表の際に聴衆の反応を確かめていた。}
	{間の取り方がよかった。}
	{決められた時間に近い範囲で行った。}
	{機材の設定や準備がすぐできていた。}
	}
	{
	{説明が途切れることが2,3か所あった。}
	{声が少し大きすぎたり小さすぎた。}
	{発表の際に聴衆の方をあまり見ていないか反応を確かめていなかった。}
	{決められた発表時間を少し外れた。}
	{機材の設定や準備に少し時間がかかった。}
	}
	{
	{声が小さすぎて聞き取れなかった。}
	{聴衆のほうを全く見ない、反応を無視して行った。}
	{間がない発表であった。}
	{発表時間が極端に短い、または長すぎた。}
	{手元の資料やPC画面を見て発表していた。}
	{機材の取り扱いや発表の準備がほとんどできていなかった。}
	}
	{発表構成}{30}
	{
	{初めに発表内容に関する概要があった。}
	{発表内容の順序に必然性があった}
	{各構成の部分のバランスが良かった。}
	{図や表を使い簡潔にまとめられていた。}
	{引用は適切である。}
	}
	{
	{文字だけの発表で、概略が少しつかみづらかった。}
	{図の内容が少し見づらかった。}
	{項目の内容の分量にばらつきが少しあった。}
	{スライドごとに情報の詳しさが一部異なりすぎていた。}
	{ページの分量が発表時間に対して少し足りない、または多すぎた。}
	}
	{
	{ほとんどのスライドで情報量が少なかった。}
	{一つのスライドに文字を詰めすぎていた。}
	{図が大きすぎたまたは小さすぎた。}
	{スライドごとに情報の詳しさが異なりすぎていた。}
	{スライドの内容が情報ごとにまとまっていなかった。}
	}
	{発表内容}{50}
	{
	{内容は適切であった。}
	{図の使い方がよかった。}
  {SVGファイルのデモが適切であった。}
	{内容が自分の言葉で述べられていた。}
	{それぞれの項目の関連性とバランスがよかった。}
	{発表したいことが十分に説明されていた。}
	{自分の意見が明確であった。}
	}
	{
	{内容のごく一部に説明不足なところがあった。}
	{図が少なくて説明が少しわかりずらかった。}
  {SVGファイルのデモの内容が少し足りなかった。}
	{内容に関して他からの引用が少し多かった。}
	{それぞれの項目の関連性に少し不十分なところがあった。}
	{発表内容の必要性の説明が少し足りなかった。}
	}
	{
	{内容が少なすぎる。}
	{図を使用していないのでわかりずらい。}
  {SVGファイルのデモの全くなかったか足りなかった。}
	{スライドの記述と発言内容に差がありすぎる。}
	{内容が多すぎて散漫である。}
	{内容が引用ばかりで自分でまとめた形跡がなかった。}
	{内容の説明が不十分であった。}
	}
}
}
