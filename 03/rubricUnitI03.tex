\documentclass[a4j]{jreport}
\input rubricHead.tex
\input rubricPresentation.tex
\input rubricUnitIHead.tex
\begin{document}
\setcounter{chapter}{2}
\chapter{複雑な図形を描く}
\changePage{5/9}
今回の演習では次のことについて学ぶ。
\begin{itemize}
 \item 曲線の描き方
 \item 一定のパターンで塗りつぶす
 \item 文字列の扱い
\end{itemize}

\Probs{折れ線、多角形を描く}{演習のビデオ1を見て次の問いに答えよ。}{
{ビデオ内の「多角形を描く」において\texttt{linejoin}を変えて、正5角形な
どを描き、その差について考察しなさい。キャプチャした図をつけること。図形
は変えてもよい。図形が同じであればソースは不要である。}{0}
{一つの\Verb+<path>+要素で2つの長方形を描きなさい。2つの長方形を別の要素
を用いて描く時との違いを考察しなさい。考察の部分はこの下の欄に書きなさ
い。}{0.1}
{穴が開いた正方形を描きなさい。実際に穴が開いていることを示すようなアイ
デアを付ければなおよい。}{0}
{\Verb+<path>+要素の属性\texttt{d}で\texttt{M}と\texttt{m}、\texttt{L}と
\texttt{l}の違いを例を挙げて説明しなさい。ソースコードと画面のキャプチャ
をつけること。}{0}
{B\'ezier 曲線で図形を描きなさい。}{0}
}
\Probs{パターンを使う}{演習のビデオ2を見て次の問いに答えよ。}{
{今までに作成した図形などをパターンを用いたものに書き直しなさい。ない場
合には新規に作成すること}{0}
{前問で作成したパターンを用いた場合と使用しないで同じ図形を描いたときの
コードの長さ、変更の手間などを比較し、考察を付けなさい}{0}
{パターンの構成要素にアニメーションをつけて見え方の変化を報告しなさい。
錯視図形であるともっと良い。思いつかない場合には、ビデオ内のモーガンのね
じれのひもに別のアニメーションをつけてもよい。}{0}
}
\Probs{パターンを使う(2)}{余力問題}{
{長方形以外の図形をパターンを用いて塗りつぶしたものを錯視しなさい}{0}
{配布資料を参考にして、図形を傾いたパターンで塗りつぶしなさい。なた、傾
きの角度にアニメーションがつけられるか調べなさい。}{0}
}
\Probs{文字列の扱い}{演習のビデオ2を見て次の問いに答えよ。}{
{自分の名前を表示する。次に、それを姓と名前の部分に分け、移動したり、色を変えたアニ
メーションをつける。}{0}
{適当な道のりに沿って自分の名前が移動するアニメーションを作成する。}{0}
{文字列表示の属性の\texttt{text-anchor} や\texttt{dominant-baseline} の
違いを例とともに報告する。}{0}}
\Rubric{第3回(5/9)}{ノートの内容}{
今回の内容は少し多くなっている。いくつかの課題の内容をまとめたものを作成
してもよい。}
{{課題1}{35}
  {
	{\texttt{linejoin}の違いについてキャプチャした図があり、十分な考察がある。}
	{一つの\texttt{path} 要素で2 つの長方形を描くことに関して、キャプチャした図とリスト
	があり、十分な考察がある。}
	{穴が開いた正方形を描くソースコードとキャプチャがあり、十分な考察があ
	る。}
	{穴が開いていることを示す十分な証拠を画面の動きからわかり、ソースコードと考察がある。}
	{\texttt{path} 要素の属性\texttt{d} で\texttt{M} と\texttt[m]、
	\texttt{L} と\texttt{l} の違いを例を挙げて説明していて、十分な考察があ
	る。}
	{B\'ezier 曲線で描かれた独自の図形のキャプチャとソースコードがあり、考
	察も十分にある。}
	}
	{
	{\texttt{linejoin}の違いについてキャプチャした図があるが、
考察が十分ではない}
	{一つの\texttt{path} 要素で2 つの長方形を描くことに関して、キャプチャした図とリスト
	があるが、考察が十分ではない。}
	{穴が開いた正方形を描くソースコードとキャプチャがあるが、考察が十分ではない。}
	{穴が開いていることを示す十分な証拠を画面の動きからわかりにくい。ソー
	スコードと考察はある。}
	{\texttt{path} 要素の属性\texttt{d} で\texttt{M} と\texttt[m]、
	\texttt{L} と\texttt{l} の違いを例を挙げて説明しているが、考察が十分で
	はないる。}
	{B\'ezier 曲線で描かれた独自の図形のキャプチャとソースコードがあるが、考
	察が十分ではない。}
	}
	{
	{\texttt{linejoin}の違いについてキャプチャした図がないか、考察がほとん
	どない。}
	{一つの\texttt{path} 要素で2 つの長方形を描くことに関して、キャプチャした図、リスト
	のいずれかがない。考察が不十分であるかない。}
	{穴が開いた正方形を描くソースコードとキャプチャのいずれかがない。考察が
不十分である。}
	{穴が開いていることを示す十分な証拠を画面の動きからわからない。ソー
	スコードと考察のいずれかまたは
両方がない。}
	{\texttt{path} 要素の属性\texttt{d} で\texttt{M} と\texttt[m]、
	\texttt{L} と\texttt{l} の違いを例を挙げて説明していのいずれかまたは
両方がない。}
	{B\'ezier 曲線で描かれた独自の図形のキャプチャ、ソー
スコードがなく、考察が不十分である。}
	}
	{課題2}{25}
	{
	{独自のパターンを利用した図形とソースコードがあり、考察も十分である。}
	{同じ図形でパターンを利用しなかった場合のコードがあり、コードの長さ、変更
	の手間などを比較し、考察が十分にある。}
	{パターンの構成要素にアニメーションをつけた図が3枚以上あり、ソースコード
と十分な考察がある。}
	}
	{
	{パターンを利用した図形があり、ソースコードもあるが、考察も少し足りな
	い。}
	{同じ図形でパターンを利用しなかった場合のコードがない。コードの長さ、変更
の手間などを比較し、考察が十分にない。}
	{パターンの構成要素にアニメーションをつけた図が2枚以下である。ソースコー
ドと十分な考察のいずれかがない。}
	}
	{
	{パターンを利用した図形があるが、ソースコードがない。考察が不十分である。}
	{同じ図形でパターンを利用しなかった場合のコードがない。コードの長さ、変更
の手間などを比較し、考察がない。}
	{パターンの構成要素にアニメーションをつけた図が1枚以下である。ソースコー
ドと考察がない。}
	}
	{課題2余力問題}{25}
	{
	{長方形以外の図形をパターンを用いて塗りつぶした図形とソースコードがあり、
考察も十分である。}
	{傾いたパターンを用いて塗りつぶした図形とソースコードがあり、考察も十分
である。}
	{傾いたパターンにアニメーションを付けたもので塗りつぶした図形とソースコー
ドがあり、考察も十分である。または、できないことの十分な説明がある。}
	}
	{
	{長方形以外の図形をパターンを用いて塗りつぶした図
形とソースコードがあり、考察が少し足りない。}
	{傾いたパターンを用いて塗りつぶした図形とソース
コードがあり、考察が少し足りない。}
	{傾いたパターンにアニメーションを付けたもので塗りつぶした図形とソースコー
ドがあり、考察が少し足りない。または、できないことの説明が少し足りないある。}
	}
	{
	{長方形以外の図形をパターンを用いて塗りつぶした図形とソースコードがない。
考察が足りない。}
	{傾いたパターンを用いて塗りつぶした図形とソース
コードがない。考察が足りない。}
	{傾いたパターンにアニメーションを付けたもので塗り
つぶした図形とソースコードがあり、考察が足りない。
または、できないことの説明が足りないある。}
	}
	{課題3}{25}
	{
	{自分の名前を表示している。それを姓と名前の部分に分け、移動したり、色を変え
たアニメーションをつけている。図とソースコードと十分な考察もある。}
	{適当な道のりに沿って自分の名前が移動するアニメー
ションの図とソースコードと十分な考察がある。}
	{文字列表示の属性の\texttt{text-anchor} や\texttt{dominant-baseline}
の違いが分かるような図があり、ソースコードとともに十分な考察がある。}
	}
	{
	{自分の名前を表示している。それを姓と名前の部分に分け、移動したり、色を変え
たアニメーションをつけている。図とソースコードと考察のいずれかがないか不十分である。}
	{適当な道のりに沿って自分の名前が移動するアニメーションの図とソースコード
と考察のいずれかが不十分である。}
	{文字列表示の属性の\texttt{text-anchor} や\texttt{dominant-baseline}
の違いが図をみてもすぐにわからない。ソースコードがないか考察が少し不十分である。}
	}
	{
	{自分の名前を表示していない。それを姓と名前の部分に分け、移動したり、色を
変えたアニメーションがない。図とソースコードと考察のいずれかがない。}
	{適当な道のりに沿って自分の名前が移動するアニメーションの図とソースコード
と考察のいずれかがない。}
	{文字列表示の属性の\texttt{text-anchor} や\texttt{dominant-baseline}
の違いが図をみてもわからない。ソースコードや考察がない。}
	}
}
\rublicPresenII{第3回(5/9)}

\end{document}