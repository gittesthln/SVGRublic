\documentclass[a4j]{jreport}
\input ../rubricHead.tex
\input ../rubricPresentation.tex
\input ../rubricUnitIHead2.tex
\begin{document}
\setcounter{chapter}{8}
\chapter{Web サーバーの基礎}
\changePage{6/20}
今回の演習の目的は次のとおりである。
\begin{itemize}
 \item PHPのプログラムを用いて簡単なページの作成を行うこと
 \item PHPの処理がどこで行われているかを理解する
 \item サーバーとのデータを送受信の基本を理解する
 \item PHPによるファイルの作成とその送信方法を理解する
\end{itemize}
課題に\Must と書かれたものを最低行うこと。それ以外の課題は
いくつか選択してよい。
\Probs{PHPの基礎}{演習のビデオ1を見て次の問いに答えよ。}{
 {次のことについて報告をする。
 \begin{itemize}
	\item \Must ビデオ1にある自分の名前をPHPプログラムで表示するページを作
				成し、表示されることを確認する。
	\item \Must ビデオ1にある自分の名前を10回PHPプログラムで表示するページを作
				成し、表示されることを確認する。
	\item \Must 上記の2つのプログラムで表示されたページのソースコードを確
				認し、その理由を説明する。
	\item 「10回表示する」ページにおける表示に不具合があれば直す。
 \end{itemize}
 }{0}
 {PHPと他の言語での文法上の類似点と相違点について記せ(ビデオ3の内容も考
 慮すること)。
 \begin{itemize}
	\item 類似点\\[4\baselineskip]
	\item 相違点\\[3\baselineskip]
 \end{itemize}}{0}
 }
  \newpage
 \newcommand{\VV}[1]{\Verb+"#1"+}
 \newcommand{\VVV}[1]{\Verb+#1+}
\Probs{クライアントとサーバーでのデータのやり取りの基本}{演習のビデオ2を見て次の問いに答えよ。}{
 {
 データの送受信が動作していることを次のことを参考にして確認する
 \begin{itemize}
	\item \Must ビデオ2にある、自分の名前を記入定してそれを表示するものが
				動作する。
	\item \Must 9行目の\texttt{method=}\VV{POST}を\texttt{method=}\VV{PUT}に変
				更し\texttt{hello.php}の9行目と10行目の
				\VVV{\$\_POST}を
				\VVV{\$\_GET}に変更したもので同じように実行さ
				せて動作する。
	\item \Must \texttt{method=}\VV{POST}との\texttt{method=}\VV{PUT}による違いが
				あるか答えよ。
 \end{itemize}
 }{0}
 {サーバーに伝えられる情報にはどのようなものがあるか確認せよ。内容量が多
 いのと個人情報の観点から、情報の種類だけ見えるようにした図にすること。}{0}
 }
% \newpage
 \Probs{サーバーでSVGファイルを作成}{演習のビデオ3を見て次の問いに答え
 よ。}{
 {「サーバーでSVGファイルを作成」の実行ができることを確認する。
 \begin{itemize}
	\item \Must ソースコードを見て、期待したSVGファイルがリストとして得られる。
	\item \Must アドレスバーに\VV{http://localhost/svg-polygon.php?N=10}
				と打ち込んだ場合の実行結果(\texttt{method=}\VV{POST}との
				\texttt{method=}\VV{PUT}による違いがあるか答えよ。)
	\item リスト8.3 の16行目の後に次の行を追加したときの動作

				\VVV{header('Content-Disposition: attachment;
				filename="'.\$\_GET['N'].'-polygon.svg"');}

 \end{itemize}}{0}
 {MIME について説明しなさい。}{0}
 }
%\newpage
\RubricN{第9回(6/20)}{ノートの内容}{
\GradeLegend
}
{
{課題1-1}{20}
{
  {自分の名前をPHPプログラムで表示するページを作成し、表示されることを確
	認し、表示ページのソースも確認してある。}
  {自分の名前をPHPプログラムで10回表示するページを作成し、表示されることを確
	認し、表示ページのソースも確認してある。}
	{2つのプログラムで表示されたページのソースコードの形になることの説明
	が正しい。}
	{「10回表示する」ページにおける表示の不具合の説明が正しく、それに対応
	してソースを直し方が正しい。}
}
{
  {自分の名前をPHPプログラムで表示するページを作成し、表示されることを確
	認しているが、ソースコードの確認をPHPのプログラムと思っている。}
  {自分の名前をPHPプログラムで10回表示するページを作成し、表示されることを確
	認しているが、ソースコードの確認をPHPのプログラムと思っている。}
	{2つのプログラムで表示されたページのソースコードの確
	 認をしているが理由が正しくない。}
	{「10回表示する」ページにおける表示の不具合の説明が不正確であり、それに対応
	してソースを直し方が\texttt{<br>}要素以外を用いている。}
}
{
  {自分の名前をPHPプログラムで表示するページを作成し、表示されることを確
	認しているが、ソースコードの確認がない。}
  {自分の名前をPHPプログラムで10回表示するページを作成し、表示されることを確
	認しているが、ソースコードの確認がない。}
	{2つのプログラムで表示されたページのソースコードの確
	 認をPHPのプログラムとしてしか確認していないので課題を解答していない。}
	{「10回表示する」ページにおける表示の不具合の説明が正しくない。
	ソースを直し方が\texttt{<br>}要素を用いている。}
}
{\ResultA}
{課題1-2}{15}
{
  {PHPとその他の言語の文法上の類似点と相違点がそれぞれ3つ以上挙げられて
	いて指摘が正しい。}
}
{
  {PHPとその他の言語の文法上の類似点と相違点が合わせて5つ以下挙げられて
	いるか、指摘が一部正しくない。}
}
{
	{データの種類に関するものがない。}
  {変数名の規約に関するものがない。}
	{変数のスコープルールに関するものがない。}
	{制御構造に関するものがない。}
	{文字列の取り扱いに関するものがない。}
	{関数の定義に関するものがない。}
}
{\ResultEI}
{課題2-1}{25}
{
  {自分の名前を記入定してそれを表示するものが動作している。リスト、考察も
	ある。}
	{\texttt{method}を\VVV{POST}から\VVV{PUT}に変更したものの報
	告がある。}
	{\texttt{method}の\VVV{POST}と\VVV{PUT}の違いの説明が正し
	い。}
}
{
  {自分の名前を記入定してそれを表示するものが動作している。リスト、考察
	のどちらかが不十分である。}
	{\texttt{method}を\VVV{POST}から\VVV{PUT}に変更したものの報
	告があるが、説明がやや不足している。}
	{\texttt{method}の\VVV{POST}と\VVV{PUT}の違いの説明に正し
	くないところがある。}
}
{
  {自分の名前を記入定してそれを表示するものが動作していないか、リスト、考察
	のないか、まったく不十分である。}
	{\texttt{method}を\VVV{POST}から\VVV{PUT}に変更したものの報
	告がないか、説明が不足している。}
	{\texttt{method}の\VVV{POST}と\VVV{PUT}の違いの説明が正し
	くない。}
}
{\ResultA}
{課題2-2}{10}
{
  {サーバーに伝えられる情報の確認が十分ある}
	{個人情報の観点から、情報の内容に制限をかけている。}
	{考察も十分にある。}
}
{
  {サーバーに伝えられる情報の確認があるが、重要なところとそうでないとこ
	ろの分類が十分なされていない。}
	{個人情報の観点から、	情報の内容に制限を十分にかけていない。}
	{考察が少し足りない。}
}
{
  {サーバーに伝えられる情報の確認がないか、重要なところとそうでないとこ
	ろの分類が全くない。}
	{個人情報の観点から、	情報の内容に制限をかけていない。}
	{考察がないか、足りない。}
}
{\ResultEFI}
{課題3-1}{20}
{
  {サーバーに伝えられる情報のプログラムが正しく動作し、何回かデータを変
	えて実行している。}
	{\VVV{svg-polygon.php}がアドレスバーに表示されているとき、そのあとに\VVV{?N=10}
				と打ち込んで実行した場合の報告があり、考察が正しい。}
	{\VVV{header(...)}によるファイルの転送方法を変えたものが正しく実行され
	ている。}
}
{
  {サーバーに伝えられる情報のプログラムが正しく動作しているが、何回かデータを変
	えて実行していない。}
	{\VVV{svg-polygon.php}がアドレスバーに表示されているとき、そのあとに\VVV{?N=10}
				と打ち込んで実行した場合の報告がないか、考察が正しくない。}
	{\VVV{header(...)}によるファイルの転送方法を変えたものが正しく実行され
	ていない。}
}
{
  {サーバーに伝えられる情報のプログラムが正しく動作していない。}
	{\VVV{svg-polygon.php}がアドレスバーに表示されているとき、そのあとに\VVV{?N=10}
				と打ち込んで実行した場合の報告がない。}
	{\VVV{header(...)}によるファイルの転送方法を変えたものの報告がない。}
}
{\ResultA}
{課題3-2}{10}
{
  {MIME についての説明が自分なりにまとめて十分にあり、正しい。}
}
{
  {MIME についての説明が自分なりにまとめられていない。}
	{演習で取り上げている MIME タイプの説明がある。}
	{重要な MIME タイプを取り上げている。}
}
{
  {MIME についての説明がネットからのものを用いている。}
	{演習で取り上げている MIME タイプの説明がない。}
	{重要な MIME タイプを取り上げていない。}
}
{\ResultEI}
}
\rublicPresenP{第9回(6/20)}

\end{document}