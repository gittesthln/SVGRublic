\documentclass[a4j]{jreport}
\input ../rubricHead.tex
\input ../rubricPresentation.tex
\input ../rubricUnitIHead2.tex
\begin{document}
\setcounter{chapter}{8}
\chapter{WebStorageとJSON、Webサーバーの動作確認}
\changePage{6/20}
今回の演習の目的は次のとおりである。
\begin{itemize}
 \item PHPのプログラムを用いて簡単なページの作成を行うこと
 \item PHPの処理がどこで行われているかを理解する
 \item サーバーとのデータを送受信の基本を理解する
 \item PHPによるファイルの作成とその送信方法を理解する
\end{itemize}
課題に\Must と書かれたものを最低行うこと。それ以外の課題は
いくつか選択してよい。
\Probs{PHPの基礎}{演習のビデオ1を見て次の問いに答えよ。}{
 {次のことについて報告をする。
 \begin{itemize}
	\item \Must ビデオ1にある自分の名前をPHPプログラムで表示するページを作
				成し、表示されることを確認する。
	\item \Must ビデオ1にある自分の名前を10回PHPプログラムで表示するページを作
				成し、表示されることを確認する。
	\item \Must 上記の2つのプログラムで表示されたページのソースコードを確
				認し、その理由を説明する。
	\item 「10回表示する」ページにおける表示に不具合があれば直す。
 \end{itemize}
 }{0}
 }
% \newpage
\Probs{クライアントとサーバーでのデータのやり取りの基本}{演習のビデオ2を見て次の問いに答えよ。}{
 {
 \begin{itemize}
	\item \Must ビデオ2にある、自分の名前を記入定してそれを表示するものが
				動作することを確認する。
	\item 
 \end{itemize}
 }{0}
 \Probs{サーバーでSVGファイルを作成}{演習のビデオ3を見て次の問いに答えよ。}{
 }
%\newpage
\RubricN{第7回(6/13)}{ノートの内容}{
\GradeLegend
}
{
{課題1-1}{10}
{
  {バンジオ・ビンナの錯視図形の初期状態、色の変更後の図が共にある。}
}
{
  {バンジオ・ビンナの錯視図形の初期状態、色の変更後の図のどちらかがない。}
}
{
  {バンジオ・ビンナの錯視図形の初期状態、色の変更後の図がないか見にくい。}
}
{\ResultA}
{課題1-2}{20}
{
  {バンジオ・ビンナの錯視図形(\texttt{{localStorage}}版)の初期状態、
  色の変更後の図が共にある。}
  {バンジオ・ビンナの錯視図形(\texttt{localStorage}版)について
  \texttt{localStorage}の値の確認の図が見やすい。}
  {バンジオ・ビンナの錯視図形(\texttt{localStorage}版)でブラウザを閉
  じた後に、再度表示を説明をするためにコンソールに出力で確認している。}
	{\texttt{localStorage}版の9行目のコメントを外して上記と
				同様の結果がある。}
	{バンジオ・ビンナの錯視図形の形を変えるパ
				ラメータを外部から指定できる。同様の動作は
				\texttt{localStorage}版でもできる。}
}
{
  {バンジオ・ビンナの錯視図形(\texttt{{localStorage}}版)の初期状態、
  色の変更後の図のどちらかがない。}
  {バンジオ・ビンナの錯視図形(\texttt{localStorage}版)について
  \texttt{localStorage}の値の確認の図が少し見にくい。}
  {バンジオ・ビンナの錯視図形(\texttt{localStorage}版)でブラウザを閉
  じた後に、再度表示を説明をするための確認が不十分。}
	{\texttt{localStorage}版の9行目のコメントを外して上記と
				同様の結果が一部ない。}
	{バンジオ・ビンナの錯視図形の形を変えるパ
				ラメータを外部から指定と同様の動作の
				\texttt{localStorage}版が少し不十分。}
}
{
  {バンジオ・ビンナの錯視図形(\texttt{{localStorage}}版)の初期状態、
  色の変更後の図がないか見にくい。}
  {バンジオ・ビンナの錯視図形(\texttt{localStorage}版)について
  \texttt{localStorage}の値の確認の図が見にくい。}
  {バンジオ・ビンナの錯視図形(\texttt{localStorage}版)でブラウザを閉
  じた後に、再度表示を説明をするための確認がない。}
	{\texttt{localStorage}版の9行目のコメントを外して上記と
				同様の結果がない。}
	{バンジオ・ビンナの錯視図形の形を変えるパ
				ラメータを外部から指定と同様の動作の
				\texttt{localStorage}版が不十分かない。}
}
{\ResultA}
{課題2-1}{15}
{
	{バンジオ・ビン
				ナの錯視図形の\texttt{localStorage}版でページを表示させ、図形の
				データをJSON形式で保存できている図がある。}
	{構成する色を変えた後でページを閉じ、再度表示させたときに最後の
	色が表示されることをコンソールで示している。}
	{バンジオ・ビンナの錯視図形の形を変えるパ
				ラメータを外部から指定できるものでデータをJSON形式で
				\texttt{localStorage}の一つのところに保存するようになっている。}
}
{
	{バンジオ・ビン
				ナの錯視図形の\texttt{localStorage}版でページを表示させ、図形の
				データをJSON形式で保存できている図が少し見にくい。}
	{構成する色を変えた後でページを閉じ、再度表示させたときに最後の
	色が表示されることの説明が不十分。}
	{バンジオ・ビンナの錯視図形の形を変えるパ
				ラメータを外部から指定できるものでデータをJSON形式で
				\texttt{localStorage}の一つのところに保存していない。}
}
{
	{バンジオ・ビンナの錯視図形の\texttt{localStorage}版でページを表示させ、
	図形のデータをJSON形式で保存できている図が見にくいかない。}
	{構成する色を変えた後でページを閉じ、再度表示させたときに最後の
	色が表示されることの説明がない。}
	{バンジオ・ビンナの錯視図形の形を変えるパ
				ラメータを外部から指定できるものでデータを個別に
				保存している。}
}
{\ResultA}
{課題2-2}{15}
{
 {「HTMLとSVGの間でデータを交換」において最終のデータを個別に
 \texttt{localStorage}に保存している図がある。}
 {「HTMLとSVGの間でデータを交換」において最終のデータをJSON形式で
 \texttt{localStorage}に保存してある図がある。}
 {今までに作成したSVG図形の各種パラメータを\texttt{localStorage}に
				保存している。}
}
{
 {「HTMLとSVGの間でデータを交換」において最終のデータを個別に
 \texttt{localStorage}に保存している図が少し見にくい。}
 {「HTMLとSVGの間でデータを交換」において最終のデータをJSON形式で
 \texttt{localStorage}に保存してある図が少し見にくい。}
 {今までに作成したSVG図形の各種パラメータを\texttt{localStorage}に
				保存している図が少し見にくい。}
}
{
 {「HTMLとSVGの間でデータを交換」において最終のデータを個別に
 \texttt{localStorage}に保存している図がないか見にくい。}
 {「HTMLとSVGの間でデータを交換」において最終のデータをJSON形式で
 \texttt{localStorage}に保存してある図がないか見にくい。}
 {今までに作成したSVG図形の各種パラメータを\texttt{localStorage}に
				保存している図がないか見にくい。}
}
{\ResultA}
{課題2-3}{20}
{
  {分割代入で2つの変数の値を入れ替える式を1回の代入で済ませ
				るプログラムを示し、実行結果の図がある。}
  {設問にある分割代入の例をすべて実行し、正しい結果が得られている。考察
	も十分にある。}
	{\texttt{getElementsByTagName()}で得られたリストに対して
	\texttt{forEach}メソッドが使えないことの確認がある}
}
{
  {分割代入で2つの変数の値を入れ替える式を1回の代入で済ませ
				るプログラムを示しているが、実行結果の図が見にくい。}
  {設問にある分割代入の例をすべて実行し、正しい結果が得られている。考察
	が不十分である。}
	{\texttt{getElementsByTagName()}で得られたリストに対して
	\texttt{forEach}メソッドが使えないことの確認が不十分。}
}
{
  {分割代入で2つの変数の値を入れ替える式を1回の代入で済ませ
				るプログラムがないか、間違っている。}
  {設問にある分割代入の例をすべて実行しているが、正しい結果が得られてい
	ない。また、考察がない。}
	{\texttt{getElementsByTagName()}で得られたリストに対して
	\texttt{forEach}メソッドが使えないことの確認がない}
}
{\ResultEI}
{課題3}{20}
{
{\texttt{XAMPP}の設定に関する質問にすべて答えていて十分な考察ある。}
}
{
{\texttt{XAMPP}のインストール場所を確認が少し間違っている。}
{ \texttt{localhost}にアクセスしたときに表示される画面が少し見にくい。}
{\texttt{localhost/index.html}と
       \texttt{localhost/index.php}のアクセス画面がともにあるが少し見に
			 くい。}
{前問の結果についての考察が不十分である。}
}
{
{\texttt{XAMPP}のインストール場所を確認していないか間違っている。}
{ \texttt{localhost}にアクセスしたときに表示される画面がないか見にくい。}
{\texttt{localhost/index.html}と
       \texttt{localhost/index.php}のアクセス画面がないか見にくい。}
{前問の結果についての考察がないか間違っている。}
}
{\ResultEFI}
}
\rublicPresenP{第8回(6/13)}

\end{document}