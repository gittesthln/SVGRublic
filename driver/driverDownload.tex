\documentclass[a4j]{jarticle}
\usepackage{graphicx}
\title{OKI D312dn プリンタードライバーインストール}
%\author{%平野 照比古}
\date{2017/4/10}
\newcommand{\Fig}[4]{%
\begin{figure}[h]
\centering  \includegraphics[width=#1\textwidth]{#2.eps}
 \caption{#3}\label{#4}
\end{figure}}
\newcommand{\FigL}[3]{\Fig{0.7}{#1}{#2}{#3}}
\newcommand{\FigS}[3]{\Fig{0.4}{#1}{#2}{#3}}
\setcounter{topnumber}{5}
\renewcommand{\textfraction}{0.05}
\renewcommand{\topfraction}{0.95}
\begin{document}
%\maketitle
\begin{center}
 {\Large OKI D312dn プリンタードライバーインストールと設定}\\
 \hspace*{\fill}2017/4/10
\end{center}
\section{プリンタードライバーのダウンロード}
使用するプリンタはOKIデータのC312dnである。

OKIデータのホームページ
(\texttt{http://www.okidata.co.jp/printing})に行く(図\ref{OkiData})。
\FigL{OKI1}{OKIデータのホームページ}{OkiData}
%\newpage

バーナー下の「ドライバー\&ユーティティ」のタブをクリックすると図
\ref{Driver}の製品カテゴリー選択の画面(図\ref{Driver})になる。
%\newpage

\FigL{OKI2}{OKIデータのドライバー製品カテゴリー選択画面}{Driver}
\newpage

「カラーLEDプリンター」をクリックすると製品選択画面になる(図\ref{Driver1})。
\FigL{OKI3}{OKIデータの製品選択ページ}{Driver1}

製品選択で「C312dn」をクリックすると(図\ref{Driver2})になる。
%\newpage

\FigL{OKI4}{OKIデータのドライバー製品選択選択ページ(1)}{Driver2}
\newpage
使用しているOSを選択してから、スクロールすると図\ref{Driver3}の画面になる。
\FigL{OKI5}{OKIデータのドライバー製品選択ページ(2)}{Driver3}

「詳細はこちら」をクリックすると図\ref{Driver4}の画面になる。
\FigL{OKI6}{OKIデータのドライバー選択ページ(3)}{Driver4}
%\newpage

「使用許諾に同意してダウンロードする」ボタンを押してダウンロードする。
%\clearpage
\renewcommand{\FigL}[3]{\Fig{0.4}{#1}{#2}{#3}}

\section{ドライバーのインストール}
ダウンロードしたファイル(\texttt{OKJ3H04N114\_90390.exe})は自己解凍形式の実行ファ
イルなのでダブルクリックで実行できる。
\newpage

インストール開始後に言語選択の画面が現れる(図\ref{Install01})。
\FigS{Install01}{OKIデータのドライバーインストール(1)}{Install01}
%\newpage

その後、ライセンスに同意を求める画面が現れる(図\ref{Install02})。
\FigL{Install02}{OKIデータのドライバーインストール(2)}{Install02}
%\newpage

「同意する」ボタンを押すと「インストールの事前確認」の画面が現れる(図
\ref{Install03})。
\FigL{Install03}{OKIデータのドライバーインストール(3)}{Install03}
\newpage

「次へ」のボタンを押すと「インストール方法の選択」画面が現れる(図\ref{Install04})。

\FigL{Install04}{OKIデータのドライバーインストール(4)}{Install04}

ここでは「かんたんインストール(ネットワーク接続)」を選択する。
%\newpage

その後「インストール対象の確認」画面が現れる(図\ref{Install05})。
\FigL{Install05}{OKIデータのドライバーインストール(5)}{Install05}

ここでは図\ref{Install06}のように OKI C312 にチェックを入れ、プリンターIP
アドレスラジオボタンを選択し、その下のテキストボックスに
\texttt{192.168.11.10}を入力する(図\ref{Install06})。
\FigL{Install06}{OKIデータのドライバーインストール(6)}{Install06}

「次へ」のボタンを押すと与えられたIPアドレスにプリンターがあるかチェック
する。
%\newpage

事前インストールなのでプリンターが見つからない画面が現れる(図
\ref{Install065})。
\FigS{Install065}{OKIデータのドライバーインストール(7)}{Install065}

%「はい」のボタンを押すと「Setup」の画面が現れます(図\ref{Install06})。
メッセージにあるように今回は事前インストールなので「はい」を選択する。
%\newpage

その後、「インストール中」の画面(図\ref{Install07})が現れる。
\FigL{Install07}{OKIデータのドライバーインストール(8)}{Install07}

インストールが完了すると図\ref{Install08}の画面になる。
\FigL{Install08}{OKIデータのドライバーインストール(9)}{Install08}
%\newpage

「完了」のボタンを押せばプリンターのインストールは終了である。
\end{document}