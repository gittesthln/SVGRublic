\input ../beamerHead.tex
\TITLE{7}{3}{配列のメソッドの利用}{6/6}
\begin{document}
\frame{\maketitle}
\section{配列のメソッド}
\begin{frame}[containsverbatim]
 \frametitle{配列のメソッド}
\begin{itemize}
 \item 通常、配列の各要素について操作をするためには\JSKey{for}文などのルー
 プで処理を行う。
 \item JavaScriptでは\JSKey{for}文の代わりになるメソッドがいくつか定義さ
       れている。それぞれのメソッドは引数に関数を必要とする。
       \begin{itemize}
        \item 引数に渡される関数には最大で3つの引数を渡される。
        \item 一つ目の引数は配列の要素
        \item 2つ目の引数は配列のインデックス
        \item 3つ目の引数はメソッドが適用される配列自身である。\\
              これを利用すると元来の配列の要素を変更できる。
       \end{itemize}
\end{itemize}
 \JSKey{forEach(func)}:引数に関数を
\end{frame}
\begin{frame}[containsverbatim]
 \frametitle{配列のメソッド--\JSKey{forEach(func)}}
 \begin{itemize}
  \item 与えられた関数を配列の各要素に対して実行
  \item 与えられた関数の戻り値は無視される。
  \item 途中でループの処理を中断できない。
  \item 要素の値が\JSKey{undefined}のところは実行されない。
 \end{itemize}
\end{frame}
\begin{frame}[containsverbatim]
 \frametitle{配列のメソッド--\JSKey{forEach(func)実行例}}

\begin{Verbatim}[fontsize=\tiny]
 >A = [1,2,3,4];
(4) [1, 2, 3, 4]
>A[5]=5;
5
>A;
(6) [1, 2, 3, 4, undefined × 1, 5]
>A.forEach(function(V,i,Vs){
    Vs[i] = V*V;
    console.log(V*V);
});
1
4
9
16
25
undefined
>A;
[1, 4, 9, 16, undefined × 1, 25]
>undefined*undefined
NaN
\end{Verbatim}
 \begin{itemize}
  \item 配列の各要素を2乗している。
  \item 元の配列に代入しているので、実行後、要素の値
 が変化する。
  \item \JSKey{undefined}の値のところは実行されていない。
 \end{itemize}
\end{frame}
\begin{frame}[containsverbatim]
 \frametitle{配列のメソッド--\JSKey{map(func)}}
 \JSKey{map(func)}\\配列の各要素に対して引数の関数を実行し、その戻り値で
 新しい配列を作成する。
 \begin{Verbatim}[fontsize=\tiny]
>A=[1,2,3]
(3) [1, 2, 3]
>A[5]=5
5
>A;
(6) [1, 2, 3, undefined × 2, 5]
>>A.map(function(V){return V*V;});
(6) [1, 4, 9, undefined × 2, 25]
>A;
(6) [1, 2, 3, undefined × 2, 5]
 \end{Verbatim}
 \begin{itemize}
  \item 要素を2乗した配列を作成している。
  \item \JSKey{undefined}の要素はそのまま\JSKey{undefined}
  \item 元の配列は変化していない。
 \end{itemize}
\end{frame}
\begin{frame}[containsverbatim]
 \frametitle{配列のメソッド--\JSKey{map(func)実行例}}
\end{frame}
\begin{frame}[containsverbatim]
 \frametitle{配列のメソッド}
\begin{itemize}
 \item \JSKey{every(func)}\\配列の各要素に対して引数の関数を実行し、そ
       の戻り値がすべて\JSKey{true}のとき、\JSKey{true}を返す。どこかで
       \JSKey{false}になると実行はそこで打ち切られ、\JSKey{false}を返す。
 \item \JSKey{some(func)}\\配列の各要素に対して引数の関数を実行し、そ
       の戻り値がすべて\JSKey{false}のとき、\JSKey{false}を返す。どこかで
       \JSKey{true}になると実行はそこで打ち切られ、\JSKey{true}を返す。
 \item \JSKey{filter(func)}\\配列の各要素に対して引数の関数を実行し、そ
       の戻り値が\JSKey{true}になるものだけを集めた新しい配列を返す。
\end{itemize} 
\end{frame}
\end{document}
\begin{frame}[containsverbatim]
 \frametitle{}
\end{frame}
