\input ../beamerHead.tex
\TITLE{7}{1}{変数のスコープと仮引数}{6/6}
\begin{document}
\frame{\maketitle}
\section{JavaScriptにおける変数のスコープ}
\begin{frame}[containsverbatim]
 \frametitle{変数のスコープとは}
 \begin{itemize}
  \item 変数のスコープとは使用されている変数が有効な範囲のこと
  \item JavaScriptでは変数は宣言しなくて使用できる。この場合、変数はグロー
        バル変数(非推奨)
  \item 関数内で\JSKey{var}で宣言された変数はその関数内で有効で、関数の
        外からは見えない。
  \item 関数の途中で新たに\JSKey{var}で変数を宣言しても、関数内の先頭で
        宣言されたものとみなされる
  \item \JSKey{let}で宣言された変数は宣言されたブロック内で有効
  \item 関数の仮引数もそこで宣言されているものとして扱われる
 \end{itemize}
\end{frame}
\begin{frame}[containsverbatim]
 \frametitle{\texttt{var}と\texttt{let}の違い--実行例}
 \LISTN{07-1varVSlet.html}{1}{last}{\tiny}
\end{frame}
\begin{frame}[containsverbatim]
 \frametitle{確かめ(2)}
\begin{Verbatim}[fontsize=\small]
x=undefined
x=defined as global
x=defined as local?
x=defined as local?
v=defined as global
v=defined as local
v=defined as global
\end{Verbatim}
\begin{itemize}
 \item 変数\texttt{x}は12行目で宣言されている。
 \item 9行目の段階では変数に値が代入されていないので\texttt{undefined}が
			 その値となっている。
 \item 11行目から14行目のブロック内の\texttt{var}による変数の宣言と代入
			 の結果がブロックの外に持ち出されている(15行目)。
 \item 19行目から22行目のブロック内で\texttt{let}で宣言されている変
			 数\texttt{v}の代入結果はブロックの終了後には存在しない
			 ことが23行目の出力でわかる。
 \item 16行目のコメントを外すとこの時点で変数\texttt{v}が定義されていな
			 いというエラーが発生する。
\end{itemize}
\end{frame}
\begin{frame}[containsverbatim]
 \frametitle{やってみよう}
	\begin{enumerate}
	 \item 17行目と20行目にある変数\texttt{v}の宣言を\texttt{let}または
				 \texttt{var}に置き換え(\texttt{let}以外の組み合わせ3通り)、16行
				 目のコメントを外した時の動作とその理由について報告する。
	 \item \texttt{var}と\texttt{let}による変数の宣言の違いをまとめる。
	\end{enumerate}
	\end{frame}
 \begin{frame}[containsverbatim]
	\frametitle{関数内の変数のスコープ}
	\LISTN{07-2scope.html}{1}{last}{\scriptsize}
	\end{frame}
 \begin{frame}[containsverbatim]
 \frametitle{関数内の変数のスコープ--解説}
 関数\texttt{F1(x)}について
\begin{itemize}
 \item 変数\texttt{a}に2を代入
 \item 変数\texttt{a}は宣言されていないのでグローバル変数となる
 \item 仮引数\texttt{x}の値を1増加
\end{itemize}
 関数\texttt{F2(x)}について
 \begin{itemize}
	\item 変数\texttt{a}を関数内で宣言(ローカル変数)し、20を代入
	\item 仮引数は配列を想定
	\item 配列の先頭の要素の値を1増加
 \end{itemize}
 関数\texttt{F3(x)}
 \begin{itemize}
	\item 変数\texttt{b}を関数内に200を代入
	\item 仮引数は配列を想定
	\item 配列の先頭の要素の値を1増加
	\item その後で変数\texttt{b}を宣言(19行目)
	\item したがって、17行目の代入はこの関数のローカル変数に代入しているこ
				とになる
 \end{itemize}
 \end{frame}
\begin{frame}[containsverbatim]
 \frametitle{実行結果}
\begin{Verbatim}[fontsize=\footnotesize]
>x=5;     //変数 x に値を設定(グローバル変数xが定義)
5
>F1(x);
undefined //関数の戻り値。return がないので undefined
>x;
5         //F1(x)では仮引数の値を増加させているので
	        //グローバルには影響がない
>a;       //F1(x)内で変数aがグローバル変数として定義された
2
>y=[1,2,3];  //関数F2に渡すための配列を定義
[1, 2, 3]
>F2(y);      //F2(y)を呼ぶ
undefined
y;
 >[2, 2, 3]  //y[0]が変化している
>b;          //グローバル変数bは存在していない
Uncaught ReferenceError: b is not defined(…)
>F3(y);
undefined
>b;          //F3(y)を実行した後でもグローバル変数bは存在していない
Uncaught ReferenceError: b is not defined(…)
\end{Verbatim}
\end{frame}
\begin{frame}[containsverbatim]
 \frametitle{関数の引数について(まとめ)}
 \begin{itemize}
	\item 関数の引数で数値のような単純なものは値渡し
	\item 関数の引数で配列は参照渡し
	\item 配列で渡された引数に手を加えると、元の配列が変化する。
				これにより、見つけにくいバグが発生する。
	\item JavaScriptでは配列が簡単に作れるので、関数の戻り値として配列を使
				うとよい。
 \end{itemize}
\end{frame}
\begin{frame}[containsverbatim]
 \frametitle{やってみよう}
 \begin{itemize}
	\item 関数の引数がオブジェクトのとき、そのオブジェクトの値を変えたり、
				キーを追加すると、呼び出し元の値に変化があるか調べる。
	\item JavaScriptの変数のスコープルールや関数の引数の仕様を他の言語と比較してそ
 の特徴を述べる。
 \end{itemize}

\end{frame}

\end{document}
\begin{frame}[containsverbatim]
 \frametitle{}
\end{frame}
