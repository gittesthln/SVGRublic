\input ../beamerHead.tex
\TITLE{3}{1}{複雑な図形を描く}{5/9}
\begin{document}
\frame{\maketitle}
%\frame{\tableofcontents}
\section{折れ線と多角形}
\begin{frame}[containsverbatim]
 \frametitle{折れ線と多角形}
 いくつかの直線をつなげて折れ線や多角形を描くことができる
 \begin{itemize}
  \item 折れ線は\ELM{polyline}、多角形は\ELM{polygon}を用いる
  \item 各頂点は\ATTR{points}で指定する
  \item 折れ線も多角形も内部が塗られる。塗りたくなければ\ATTR{fill}に
        \VAL{none}を指定する
  \item 内部の塗り方を指示する\ATTR{fill-rule}がある
  \item 各頂点における線分の交わりの形状を定める\ATTR{linejoin}がある\REF{41}
 \end{itemize}
\end{frame}
\begin{frame}[containsverbatim]
 \frametitle{多角形の例--\REF{39}}
 正五角形や星形を描いてみました。
% \FIG{0.9}{3}{polygon5}
\end{frame}
\begin{frame}[containsverbatim]
\frametitle{多角形の例--ソースコード}
 \LISTAll{3}{svg-polygon5.svg}
\end{frame}
\begin{frame}[containsverbatim]
 \frametitle{多角形の例--ソースコード(解説)}
 \begin{itemize}
  \item 7行目で$y$座標の向きを数学で使われる方向に直している
  \item 左の正五角形は8行目から10行目で定義
  \item 座標は前もって計算しておいた。
  \item 中央の星形は12行目から14行目で定義
  \item 正五角形の頂点を一つ置きに取っている
  \item 右の図形は17行目から19行目で定義
  \item 頂点の並びは中央と同じ
  \item 違いは\ATTR{fill-rule}が\VAL{evenodd}
  \item 領域内の点と遠方を直線で結んだとき、図形の縁と交わる回数が奇数の
        ところだけ塗られる
 \end{itemize}
\end{frame}
\begin{frame}[containsverbatim]
 \frametitle{\ELM{polygon}と\ELM{polyline}の注意}
 \ATTR{linejoin}\REF{41}
 \begin{itemize}
  \item 折れ線の曲がるところの処理の指定
  \item 値の変化が激しいグラフを書く時などには\VAL{miter}は使わないよう
        に
 \end{itemize}
% \ATTR{stroke-linecap}\REF{12}
% \FIG{0.45}{2}{linecap}
\end{frame}
\section{直線で構成される図形を描く}
\begin{frame}[containsverbatim]
 \frametitle{\ELM{path}}
 \ELM{path}は多角形や曲線を含む図形を描ける。

\begin{itemize}
 \item 図形の形は\ATTR{d}で指定
 \item 記述するパラメータは大文字と小文字がある
 \item 大文字は絶対座標
 \item 小文字は直前の点からの相対座標
 \item \VAL{M},\VAL{m}:指定した位置に移動(moveto)
 \item \VAL{L},\VAL{l}:指定したまで道のりを指定(lineto)。これらは省略可
       能
 \item \VAL{z}:初めの位置と最後の位置を結んで閉じる
 \item 座標は空白またはコンマ(\VAL{,})で区切る\\
       テキストでは座標の対応をはっきりさせるため、ここでは$x$座標と$y$
       座標の間はコンマにしている
\end{itemize}
\end{frame}
\iffalse
\begin{frame}[containsverbatim]
 \frametitle{プロッター}
 平面に置かれた紙などに図形を描く機材(\texttt{http://www.i-cadcam.jp/plotter.html})
 \FIGN{0.6}{large-scale-plotter}
\end{frame}
\fi
\iffalse
\begin{frame}[containsverbatim]
 \frametitle{カッティングマシーン}
 カッティングマシーンはカッターを上下させて移動することにより、シートを
 カットする機械
 \begin{itemize}
  \item カッターを上げて移動する(切らない)のが\VAL{moveto}
  \item カッターを下げて移動する(切る)のが\VAL{lineto}
 \end{itemize}
\end{frame}
\fi
\begin{frame}[containsverbatim]
 \frametitle{長方形をいろいろな方法で描く}
 \LISTAll{3}{svg-rect.svg}
\end{frame}
\begin{frame}[containsverbatim]
 \frametitle{どのような図形が描かれるでしょうか?}
 \begin{itemize}
  \item 一番上に基本の長方形(6行目)
  \item その下に\ELM{polygon}(9行目)と\ELM{polyline}(12行目)
  \item さらに下に\ELM{path}で位置を絶対座標で与え、最後に
        \VAL{z}なし(15行目)とあり(18行目)
  \item 21行目は相対座標(\VAL{l})を用いた場合
 \end{itemize}
 デモに移る
\end{frame}
\begin{frame}[containsverbatim]
 \frametitle{次のような図形を描いてみよう。}
\begin{itemize}
 \item 一つの\ELM{path}で二つの長方形を描く
       \FIGN{0.3}{svg-path-rect2}
 \item 穴が開いた正方形を一つの\ELM{path}で描く
      \footnote{中が塗られていないことを示すために初めに小さな円を描いて
       いる}
              \FIGN{0.3}{svg-square-hole}

\end{itemize} 
\end{frame}
\iftrue
\section{曲線を描く---\Bezier 曲線}
\begin{frame}[containsverbatim]
 \frametitle{\Bezier 曲線}
 \Bezier 曲線はPhotoShopなどドロー系のソフトで曲線を描くために利用される

 フリーなドロー系のソフトであるGIMPで曲線を描いているところをデモする。
\end{frame}
\begin{frame}[containsverbatim]
 \frametitle{\Bezier 曲線の特徴}
 \begin{itemize}
  \item (3次の)\Bezier 曲線を定めるためには4点
        $\mathrm{P_0}, \mathrm{P_1}, \mathrm{P_2}, \mathrm{P_3}$ が必要
  \item この曲線は始点が $\mathrm{P_0}$ で 終点が $\mathrm{P_3}$
  \item $\mathrm{P_1}$ は点 $\mathrm{P_0}$ における接線の方向を定める
  \item $\mathrm{P_2}$ は点 $\mathrm{P_3}$ における接線の方向を定める
  \item 曲線はこの4点からなる凸包の中に含まれる
 \end{itemize}
 具体的な性質についてはデモで行う
\end{frame}
\begin{frame}[containsverbatim]
 \frametitle{デモ内容の説明}
\begin{enumerate}
 \item 4点の座標がそれぞれ右に表示
 \begin{itemize}
  \item 各点はドラッグ可能
  \item 各点の座標を直接設定可能(入力後、「設定」ボタンを押す)
  \item 「次」のボタンを押すことで\Bezier 曲線の性質が確認できる
 \end{itemize}
 \item \Bezier 曲線が表示
 \item 4点を含む凸包が表示
 \item $\mathrm{P_0}, \mathrm{P_1}$ の中点$\mathrm{P}_{01}$が表示
 \item $\mathrm{P_1}, \mathrm{P_2}$ の中点$\mathrm{P}_{12}$が表示
 \item $\mathrm{P_2}, \mathrm{P_3}$ の中点$\mathrm{P}_{23}$が表示
 \item $\mathrm{P_{01}}, \mathrm{P_{12}}$ の中点$\mathrm{P}_{012}$が表示
 \item $\mathrm{P_{12}}, \mathrm{P_{23}}$ の中点$\mathrm{P}_{123}$が表示
 \item $\mathrm{P_{012}}, \mathrm{P_{123}}$ の中点$\mathrm{P}_{0123}$が
       表示。この点は\Bezier 曲線上にある
 \item $\mathrm{P}_{0123}$の前後で2つに分けた曲線もまた、\Bezier 曲線に
       なっている
\end{enumerate}
\end{frame}
\begin{frame}[containsverbatim]
 \frametitle{面白い曲線をデザインしてみよう}
 これを使って曲線の一部を作成してみよう。

 半端な座標は手動で直すことを忘れないこと
\end{frame}
\begin{frame}[containsverbatim]
 \frametitle{\ELM{path}における指定方法}
 \begin{itemize}
  \item 4点のうち初めの点$\mathrm{P}_{0}$はそれまでに描いた図形の最終位
        置が使用される
  \item 新規に定めたい場合には\VAL{M}または\VAL{m}で指定
  \item 残りの3点は\VAL{C}または\VAL{c}の後に指定する。
  \item 2つの\Bezier 曲線を滑らかにつなぐためには共通点の接線方向を合わ
        せる
 \end{itemize}
\end{frame}
\begin{frame}[containsverbatim]
 \frametitle{\Bezier 曲線の例}
 次のコードは\REF{58}の問題3.8のハートのマークのソースコード
\begin{Verbatim}[fontsize=\footnotesize]
<?xml version="1.0" encoding="UTF-8" ?>
<svg xmlns="http://www.w3.org/2000/svg"
     xmlns:xlink="http://www.w3.org/1999/xlink"
     height="100%" width="100%">
  <title>スーツマークを描く</title>
  <g transform="translate(150,300)">
    <path d="M0,0 C220,-150 100,-320 0,-240 C-100,-320 -220,-150 0,0z"
      fill="pink" stroke-width="4" stroke="red" />
  </g>
</svg>
 \end{Verbatim}
 左右対称なので$x$座標の値を符号が逆になるように座標系をとっている
\end{frame}
\else
\begin{frame}[containsverbatim]
 \frametitle{次のビデオは...}
次のビデオは曲線の書き方
\end{frame}
\fi
\end{document}
\begin{frame}[containsverbatim]
\frametitle{}
\end{frame}

