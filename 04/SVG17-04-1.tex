\input ../beamerHead.tex
\TITLE{4}{1}{簡単なイベント処理と開発者ツールの使い方}{5/16}
\begin{document}
\frame{\maketitle}
%\frame{\tableofcontents}
\section{開発者ツール}
\begin{frame}[containsverbatim]
 \frametitle{開発者ツール}
 各種ブラウザで表現は異なるが機能はほぼ同じ
 \begin{itemize}
  \item 開くためのショートカットキーは「F12」か「Ctrl+Shift+I」
  \item DOMツリーの表示と修正など(Elements)
  \item インターラクティブな操作やエラー表示、プログラムからの出力
        (Console)
  \item JavaScript などのプログラムソースの表示、ブレイクポイントの設定
        (Source)
  \item ...
 \end{itemize}
 順番に見ていく
\end{frame}
\begin{frame}
 \frametitle{開発者ツール--まとめ}
\begin{itemize}
 \item  「Elements」
 \begin{itemize}
	\item DOMツリーを見る
	\item 要素の属性を変えることができる
 \end{itemize}
 \item 「Console」タブ
 \begin{itemize}
	\item プログラムの実行中にメッセージが出せる
	\item その場でJavaScript のプログラムが実行できる
	\item オブジェクトや変数の値の確認ができる
 \end{itemize}
 \item 「Source」タブ
 \begin{itemize}
	\item JavaScriptのソースコードにブレイクポイントを設定できる
	\item 1行ごとに実行させることができる
 \end{itemize}
			 \end{itemize}
\end{frame}
\begin{frame}[containsverbatim]
 \frametitle{確認しましよう}
 自分の使用しているブラウザで次のことを確認する(ノートにメモすること)
 \begin{itemize}
  \item 「開発者ツール」に相当する機能の名称
  \item 「開発者ツール」を開くショートカットキー
  \item どこかのWebページを開いてDOMツリーのノードを展開し、DOMツリー上
        でマウスカーソルを動かしたときの現象
  \item SVGファイルを開いて要素のプロパティを修正する
  \item コンソールの入力がどこでできるか確認する。
  \item コンソールで\texttt{2+3;}と入力して結果が表示されることを確認
  \item コンソールで次のように入力したときの結果を確かめる(結果の展開が可能な
        はずなので確認すること)
\\
        \texttt{document.getElementsByTagName("circle");}
 \end{itemize}
 ブラウザの機能、いかがでしたか
 \end{frame}
\end{document}
\begin{frame}[containsverbatim]
 \frametitle{}
\end{frame}
