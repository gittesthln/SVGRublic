\documentclass[a4j]{jreport}
\input rubricHead.tex
\input rubricPresentation.tex
\input rubricUnitIHead.tex
\begin{document}
\setcounter{chapter}{1}
\chapter{アニメーション}
\changePage{4/25}
今回の演習の予習ビデオのはじめの2つはアニメーションの基礎である。
3つ目のものはより複雑なものができる。最終的に複雑な動きをするものを作成
すること。また、キャプチャする画面はアニメーションの開始時、経過時、終了
時をすべてつけること。
\Probs{位置のアニメーション}{予習ビデオ1を見て次の問に答えよ。}{
{アニメーションを指定するために必要な属性をまとめよ。また、指定しな
ければならない属性はどれか答えよ。}{0}
{複数の図形のそれぞれに平行移動、回転、拡大縮小のアニメーションをつけ、
さらにグループ化したものにも同様のアニメーションをつけた図形を作成せよ。
なお、\texttt{defs}内にある要素に対してアニメーションをつけることも
可能である。参考資料の問題4.2、4.3などを解答するのもよい}{0}
{アニメーションの要素とアニメーションがつけられる要素との関係を答え
よ。}{0.1}
{アニメーションの属性\texttt{fill}を指定しなかった場合は終了時はどうなる
か答えよ。}{0.05}}
\Probs{一般の要素に対するアニメーション}{予習ビデオ2を見て次の問に答え
よ。}{
{色に対するアニメーションの属性をまとめよ。}{0}
{参考例のアニメーションの色を変えたものを作成せよ。}{0}
{グラデーションが横に流れるアニメーションで\texttt{gradientUnits} の値を
\texttt{objectBoundingBox}にするとグラデーションはどうなるか答えよ。}{0.05}
{ビデオ内にあるグラデーションのアニメーションに関する「やってみよう」の
課題を答えよ。}{0}
{円の大きさを変えるアニメーションとして円の属性\texttt{r}にアニメーショ
ンをつける方法と\texttt{translate}属性の\texttt{scale}で行う方法があるが、
その違いがあるか答えよ。}{0.05}
}
\Probs{複数の値を指定するアニメーション}{予習ビデオ3を見て次の問に答え
よ。}{{複数の値を指定するアニメーションで値の指定で注意することを挙げ
よ。}{0.05}
{ビデオ3内にある複数の値を与えるアニメーションの「やってみよう」の課題を
一つ以上作成する。}{0}
{ビデオ3内にあるイベントを利用したアニメーションの「やってみよう」の課題
の1番目について検討せよ。同じように動くアニメーションを2通りの方法で作
成し、比較するのが望ましい}{0}
}
\Rubric{第2回(4/25)}{ノートの内容}
{今回の内容のテキストは第3章の内容を含んでいるのでそれを利用するアニメー
ションは第3回で行う。\par
この回の予習の目的は次のとおりである。
\begin{itemize}
 \item アニメーションで指定すべき属性を理解する
 \item アニメーション要素とそれがつけられる要素との関係を理解する
 \item 複数の値を指定するアニメーション
\end{itemize}
これらの点をまとめて、作成したSVGによる画像とリストとその解説を付ける。
演習内での議論は手書きでよい。}
{{課題1}{20}
  {
  {アニメーションに関する属性がまとめられていて、例を挙げて説明が的確で
  ある}
  {\texttt{transform}属性の属性値\texttt{transform}、\texttt{rotate}、
  \texttt{scale}がすべて使われた実行例がある}
  {ソースリストがあり、その解説の内容が十分にある。}
  {画面のキャプチャが開始時、途中経過時、終了時とすべてある}
  }
  {
  {アニメーションに関する属性がまとめられているが、例がほとんどなく説明
  が不十分である}
  {\texttt{transform}属性の属性値\texttt{transform}、\texttt{rotate}、
  \texttt{scale}のうち使わていないものが一つある}
  {ソースリストがあり、その解説の内容が不十分である。}
  {画面のキャプチャが開始時、途中経過時、終了時のすべてがない}
  }
  {
  {アニメーションに関する属性が配布資料とほとんど同じで工夫がない}
  {\texttt{transform}属性の属性値\texttt{transform}、\texttt{rotate}、
  \texttt{scale}のうち使わていないものが2つ以上ある}
  {ソースリストがあり、その解説がほとんどないか、記述に誤りがある。}
  {画面のキャプチャが不十分でどのようなアニメーションかが理解できない}
  }
  {課題2}{30}
  {
  {色のアニメーションの属性に関する属性がまとめられていて、例を挙げて説明が的確で
  ある}
  {一般要素のアニメーションに関する属性がまとめられていて、例を挙げて説
  明が的確である}
  {グラデーションのアニメーションの「やってみよう」の課題を例にして説明
  が的確である}
  {ソースリストがあり、その解説の内容が十分にある。}
  {画面のキャプチャが開始時、途中経過時、終了時とすべてある}
  }
  {
  {色のアニメーションの属性に関する属性がまとめられているが、例がほとんど
  なくなく説明が不十分である}
  {一般要素のアニメーションに関する属性がまとめられているが、例がほとんど
  なくなく説明が不十分である}
  {グラデーションのアニメーションの「やってみよう」の課題をすべて行って
  おらず説明も不十分である}
  {ソースリストがあり、その解説の内容に説明不足の点がある。}
  {画面のキャプチャが開始時、途中経過時、終了時のすべてがない}
  }
  {
  {色のアニメーションに関する属性が配布資料とほとんど同じで工夫がない}
  {一般要素のアニメーションに関する属性が配布資料とほとんど同じで工夫がない}
  {グラデーションのアニメーションの「やってみよう」の課題をほとんどまた
  は全く行っておらず説明がない}
  {ソースリストはあるが、その解説がほとんどないか、記述に誤りがある。}
  {画面のキャプチャが不十分でどのようなアニメーションかが理解できない}
  }
	{課題3}{30}{
	{複数の値を指定するアニメーションの値の指定する際の注意すべき点につい
	て十分な記述がある}
	{ビデオ3内にある複数の値を与えるアニメーションの「やってみよう」の課題を
一つ以上作成している}
	{ビデオ3内にあるイベントを利用したアニメーションの「やってみよう」の課題を
作成している}
	{すべての作成したコードに対して、十分なキャプチャ画面がある}
	{すべての作成したコードに対して、リストがあり、その解説が十分である}
  }
	{
	{複数の値を指定するアニメーションの値の指定する際の注意すべき点につい
	て記述が少し足りない}
	{ビデオ3内にある複数の値を与えるアニメーションの「やってみよう」の課題を
一つ以上作成しているが内容に問題がある}
	{ビデオ3内にあるイベントを利用したアニメーションの「やってみよう」の課題を
作成しているが、内容に問題がある}
	{すべての作成したコードに対して、キャプチャ画面が少し足りない}
	{すべての作成したコードに対して、リストがあり、その解説が少し足りない}
  }
	{
	{複数の値を指定するアニメーションの値の指定する際の注意すべき点につい
	て記述が足りないか間違っている}
	{ビデオ3内にある複数の値を与えるアニメーションの「やってみよう」の課題を
作成していないか内容について非常に問題がある}
	{ビデオ3内にあるイベントを利用したアニメーションの「やってみよう」の課題を
作成がないか、内容について非常に問題がある}
	{すべての作成したコードに対して、キャプチャ画面がないか不足している}
	{すべての作成したコードに対して、リストがないか、その解説が足りない}
	}
 	}
\rublicPresen{第2回(4/25)}
\end{document}