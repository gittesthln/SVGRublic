\documentclass[a4j]{jreport}
\input rubricHead.tex
\input rubricPresentation.tex
\input rubricUnitIHead.tex
\begin{document}
\chapter{SVGの基礎}
\changePage{4/18}
演習の課題については、この中に直接記入してもよい。実行結果が伴う場合には、
印刷したものを添えること。
\Probs{演習の準備}{予習ビデオ1を見て次の問に答えよ。}
 {
 {使用するテキストエディタ名とそのエディタでBOMなしのUTF-8の文字コードで保存
  する方法について報告する。}{0.1}
 {テキストファイルをBOMありとなしで保存したときのファイルサイズ
  の違いがあるか調べて報告しなさい。}{0.1}
  {BOMとは何か。また、なぜ必要なのかを調査しなさい。}{0.2}
%  
  {SVGファイルがBOMなしで保存しなければいけない理由について答えなさい}{0.2}
 }
\Probs{簡単なSVGファイルの作成}{予習ビデオ2を見て次の問いに答えよ。}
 {
 {SVGファイルを表示したブラウザについて報告する。}{0.05}
 {直線の属性をまとめる。1年で学んだProcessingの仕様と比較すること(別紙で
 解答すること)}{0}
 {Fickの錯視を$90^{\circ}$回転させた図形でも同様の錯視が起こることを確認
 する。}{0.0}
 {直線をいくつか組み合わせて図形を作成する。色や幅を変えること。直線の代
 わりに円や長方形を用いてもよい。}{0}
 }
 \Probs{グラデーション}{予習ビデオ3を見て次の問いに答えよ}
 {
 {線形グラデーションで長方形や円を塗りつぶすこと。\texttt{gradientUnits}
 も変えて違いを説明すること。配布資料29ページのリスト2.11を用いてもよ
 い。}{0}
 {放射グラデーションで長方形や円を塗りつぶす。属性\texttt{fx}や\texttt{fy}も
 変えたものをいくつか作成すること}{0}
 }
\Rubric{第1回(4/18)}{ノートの内容}
{この回の予習の目的は次のとおりである。
\begin{itemize}
 \item 簡単なSVGファイルをテキストエディタで作成し、ブラウザで表示できる
 \item 直線、長方形、円をSVGの要素で表示できる
 \item グラデーションの基本を理解する
\end{itemize}
これらの点をまとめて、作成したSVGによる画像とリストとその解説を付ける。
演習内での議論は手書きでよい。}
{{課題1}{20}{
	{使用したテキストエディタの名称、バージョンがある。}
	{ファイルの保存形式の注意が書いてある。}
  {BOMが存在する理由が正しく書かれている。}
  {BOMつきとなしの場合でのファイルの大きさの差とその説明がある。}
  {SVGファイルがBOMなしでない理由が正しく書かれている。}
	}
	{
	{使用したテキストエディタの名称がある。}
	{ファイルの保存形式の説明に図がない。}
  {BOMが存在する理由が一部不正確である。}
  {BOMつきとなしの場合でのファイルの大きさの報告がある。}
  {SVGファイルがBOMなしでない理由が一部不正確である。}
	}
	{
	{使用したテキストエディタの名称がない。}
	{ファイルの保存形式に言及がない。}
  {BOMが存在する理由が不正確である。}
  {BOMつきとなしの場合でのファイルの大きさの差の説明がない。}
  {SVGファイルがBOMなしでない理由がないか他間違っている。}
	}
	{課題2}{30}
	{
	{使用したブラウザの名称、バージョンがある。}
  {直線の属性のまとめが十分にあり、Processingとの比較も十分である。}
  {Fickの錯視で回転させたものや線の幅を変えたものがあり、それらに考察が
  ある。}
  {直線や円、長方形を組み合わせた充分な量の図形を作成し、それに関する考
  察がある。}
	}
	{
	{使用したブラウザの名称がある。}
  {直線の属性のまとめが十分にあるが、Processingとの比較がないか不十分である。}
  {Fickの錯視で回転させたものや線の幅を変えたものがないか、それらの考察が
  不十分である。}
  {直線や円、長方形を組み合わせた充分な量の図形を作成し、それに関する考
  察が不十分である。}
	}
	{
	{使用したブラウザの名称がない。}
  {直線の属性のまとめが十分にないか全くない。}
  {Processingとの比較がない。}
  {Fickの錯視で回転させたものや線の幅を変えたものがない。}
  {直線や円、長方形を組み合わせた充分な量の図形ない。}
	}
	{課題3}{30}
	{
  {線形グラデーションでいろいろな図形を塗りつぶした例が十分にあり、それ
  らの考察も適切である。}
  {線形グラデーションの\texttt{gradientUnits}の違いについて、例を示して、
  正しい考察をしている。}
  {放射グラデーションでいろいろな図形を塗りつぶした例が十分にあり、それ
  らの考察も適切である。}
	}
	{
  {線形グラデーションでいろいろな図形を塗りつぶした例があり、それ
  らの考察もある。}
  {線形グラデーションの\texttt{gradientUnits}の違いについて、例を示しい
  ないが正しい考察をしている。}
  {放射グラデーションでいろいろな図形を塗りつぶした例があり、それ
  らの考察もある。}
	}
	{
  {線形グラデーションでいろいろな図形を塗りつぶした例がないか、それ
  らの考察がないか不十分である。}
  {線形グラデーションの\texttt{gradientUnits}の違いについて、例がなく
  考察がないか、不十分である。}
  {放射グラデーションでいろいろな図形を塗りつぶした例がないか、それ
  らの考察もないか不十分である。}
	}
  {ノートの使い方}{20}
  {
  {プログラムのリスト、実行結果がすべてあり、考察も適切である。}
  {画面のキャプチャが図形を表示するのに十分な程度の大きさになっており、
  ブラウザ全体になっている。}
  {予習の内容が各項目できれいにまとめられている。}
  {ノートの余白が十分にあり、後から記述を追加できるような配慮がある。}
  {グループ内での議論がまとめられていて、経過と解決法が明確になっている。}
  }
  {
  {プログラムのリスト、実行結果が一部ないか、考察が一部不十分である。}
  {画面のキャプチャが図形を表示するよりも大きすぎるか、
  ブラウザ全体になっていない。}
  {予習の内容の項目が一部ない。}
  {ノートの余白があまりない。}
  {グループ内での議論がまとめられているが、経過と解決法が明確になってい
  ない。}
  }
  {
  {プログラムのリスト、実行結果が少なすぎる。}
  {考察の量が不十分である。}
  {画面のキャプチャがデスクトップ全体になっているか、
  ブラウザで実行したかがわからない。}
  {予習の要点のまとめが少なすぎる。}
  {ノートに余白が全くないか、ほとんどない。}
  {グループ内での議論がまとめられていなくて、経過と解決法が明確になって
  ない。}
  }
}
\rublicPresen{第1回(4/18)}
\chapter{アニメーション}
\changePage{4/25}
今回の演習の予習ビデオのはじめの2つはアニメーションの基礎である。
3つ目のものはより複雑なものができる。最終的に複雑な動きをするものを作成
すること。また、キャプチャする画面はアニメーションの開始時、経過時、終了
時をすべてつけること。
\Probs{位置のアニメーション}{予習ビデオ1を見て次の問に答えよ。}{
{アニメーションを指定するために必要な属性をまとめよ。また、指定しな
ければならない属性はどれか答えよ。}{0}
{複数の図形のそれぞれに平行移動、回転、拡大縮小のアニメーションをつけ、
さらにグループ化したものにも同様のアニメーションをつけた図形を作成せよ。
なお、\texttt{defs}内にある要素に対してアニメーションをつけることも
可能である。参考資料の問題4.2、4.3などを解答するのもよい}{0}
{アニメーションの要素とアニメーションがつけられる要素との関係を答え
よ。}{0.1}
{アニメーションの属性\texttt{fill}を指定しなかった場合は終了時はどうなる
か答えよ。}{0.05}}
\Probs{一般の要素に対するアニメーション}{予習ビデオ2を見て次の問に答え
よ。}{
{色に対するアニメーションの属性をまとめよ。}{0}
{参考例のアニメーションの色を変えたものを作成せよ。}{0}
{グラデーションが横に流れるアニメーションで\texttt{gradientUnits} の値を
\texttt{objectBoundingBox}にするとグラデーションはどうなるか答えよ。}{0.05}
{ビデオ内にあるグラデーションのアニメーションに関する「やってみよう」の
課題を答えよ。}{0}
{円の大きさを変えるアニメーションとして円の属性\texttt{r}にアニメーショ
ンをつける方法と\texttt{translate}属性の\texttt{scale}で行う方法があるが、
その違いがあるか答えよ。}{0.05}
}
\Probs{複数の値を指定するアニメーション}{予習ビデオ3を見て次の問に答え
よ。}{{複数の値を指定するアニメーションで値の指定で注意することを挙げ
よ。}{0.05}
{ビデオ3内にある複数の値を与えるアニメーションの「やってみよう」の課題を
一つ以上作成する。}{0}
{ビデオ3内にあるイベントを利用したアニメーションの「やってみよう」の課題
の1番目について検討せよ。同じように動くアニメーションを2通りの方法で作
成し、比較するのが望ましい}{0}
}
\Rubric{第2回(4/25)}{ノートの内容}
{今回の内容のテキストは第3章の内容を含んでいるのでそれを利用するアニメー
ションは第3回で行う。\par
この回の予習の目的は次のとおりである。
\begin{itemize}
 \item アニメーションで指定すべき属性を理解する
 \item アニメーション要素とそれがつけられる要素との関係を理解する
 \item 複数の値を指定するアニメーション
\end{itemize}
これらの点をまとめて、作成したSVGによる画像とリストとその解説を付ける。
演習内での議論は手書きでよい。}
{{課題1}{20}
  {
  {アニメーションに関する属性がまとめられていて、例を挙げて説明が的確で
  ある}
  {\texttt{transform}属性の属性値\texttt{transform}、\texttt{rotate}、
  \texttt{scale}がすべて使われた実行例がある}
  {ソースリストがあり、その解説の内容が十分にある。}
  {画面のキャプチャが開始時、途中経過時、終了時とすべてある}
  }
  {
  {アニメーションに関する属性がまとめられているが、例がほとんどなく説明
  が不十分である}
  {\texttt{transform}属性の属性値\texttt{transform}、\texttt{rotate}、
  \texttt{scale}のうち使わていないものが一つある}
  {ソースリストがあり、その解説の内容が不十分である。}
  {画面のキャプチャが開始時、途中経過時、終了時のすべてがない}
  }
  {
  {アニメーションに関する属性が配布資料とほとんど同じで工夫がない}
  {\texttt{transform}属性の属性値\texttt{transform}、\texttt{rotate}、
  \texttt{scale}のうち使わていないものが2つ以上ある}
  {ソースリストがあり、その解説がほとんどないか、記述に誤りがある。}
  {画面のキャプチャが不十分でどのようなアニメーションかが理解できない}
  }
  {課題2}{30}
  {
  {色のアニメーションの属性に関する属性がまとめられていて、例を挙げて説明が的確で
  ある}
  {一般要素のアニメーションに関する属性がまとめられていて、例を挙げて説
  明が的確である}
  {グラデーションのアニメーションの「やってみよう」の課題を例にして説明
  が的確である}
  {ソースリストがあり、その解説の内容が十分にある。}
  {画面のキャプチャが開始時、途中経過時、終了時とすべてある}
  }
  {
  {色のアニメーションの属性に関する属性がまとめられているが、例がほとんど
  なくなく説明が不十分である}
  {一般要素のアニメーションに関する属性がまとめられているが、例がほとんど
  なくなく説明が不十分である}
  {グラデーションのアニメーションの「やってみよう」の課題をすべて行って
  おらず説明も不十分である}
  {ソースリストがあり、その解説の内容に説明不足の点がある。}
  {画面のキャプチャが開始時、途中経過時、終了時のすべてがない}
  }
  {
  {色のアニメーションに関する属性が配布資料とほとんど同じで工夫がない}
  {一般要素のアニメーションに関する属性が配布資料とほとんど同じで工夫がない}
  {グラデーションのアニメーションの「やってみよう」の課題をほとんどまた
  は全く行っておらず説明がない}
  {ソースリストはあるが、その解説がほとんどないか、記述に誤りがある。}
  {画面のキャプチャが不十分でどのようなアニメーションかが理解できない}
  }
	{課題3}{30}{
	{複数の値を指定するアニメーションの値の指定する際の注意すべき点につい
	て十分な記述がある}
	{ビデオ3内にある複数の値を与えるアニメーションの「やってみよう」の課題を
一つ以上作成している}
	{ビデオ3内にあるイベントを利用したアニメーションの「やってみよう」の課題を
作成している}
	{すべての作成したコードに対して、十分なキャプチャ画面がある}
	{すべての作成したコードに対して、リストがあり、その解説が十分である}
  }
	{
	{複数の値を指定するアニメーションの値の指定する際の注意すべき点につい
	て記述が少し足りない}
	{ビデオ3内にある複数の値を与えるアニメーションの「やってみよう」の課題を
一つ以上作成しているが内容に問題がある}
	{ビデオ3内にあるイベントを利用したアニメーションの「やってみよう」の課題を
作成しているが、内容に問題がある}
	{すべての作成したコードに対して、キャプチャ画面が少し足りない}
	{すべての作成したコードに対して、リストがあり、その解説が少し足りない}
  }
	{
	{複数の値を指定するアニメーションの値の指定する際の注意すべき点につい
	て記述が足りないか間違っている}
	{ビデオ3内にある複数の値を与えるアニメーションの「やってみよう」の課題を
作成していないか内容について非常に問題がある}
	{ビデオ3内にあるイベントを利用したアニメーションの「やってみよう」の課題を
作成がないか、内容について非常に問題がある}
	{すべての作成したコードに対して、キャプチャ画面がないか不足している}
	{すべての作成したコードに対して、リストがないか、その解説が足りない}
	}
 	}
\rublicPresen{第2回(4/25)}
\chapter{複雑な図形を描く}
\changePage{5/9}
今回の演習では次のことについて学ぶ。
\begin{itemize}
 \item 曲線の描き方
 \item 一定のパターンで塗りつぶす
 \item 文字列の扱い
\end{itemize}

\Probs{折れ線、多角形を描く}{演習のビデオ1を見て次の問いに答えよ。}{
{ビデオ内の「多角形を描く」において\texttt{linejoin}を変えて、正5角形な
どを描き、その差について考察しなさい。キャプチャした図をつけること。図形
は変えてもよい。図形が同じであればソースは不要である。}{0}
{一つの\Verb+<path>+要素で2つの長方形を描きなさい。2つの長方形を別の要素
を用いて描く時との違いを考察しなさい。考察の部分はこの下の欄に書きなさ
い。}{0.1}
{穴が開いた正方形を描きなさい。実際に穴が開いていることを示すようなアイ
デアを付ければなおよい。}{0}
{\Verb+<path>+要素の属性\texttt{d}で\texttt{M}と\texttt{m}、\texttt{L}と
\texttt{l}の違いを例を挙げて説明しなさい。ソースコードと画面のキャプチャ
をつけること。}{0}
{B\'ezier 曲線で図形を描きなさい。}{0}
}
\Probs{パターンを使う}{演習のビデオ2を見て次の問いに答えよ。}{
{今までに作成した図形などをパターンを用いたものに書き直しなさい。ない場
合には新規に作成すること}{0}
{前問で作成したパターンを用いた場合と使用しないで同じ図形を描いたときの
コードの長さ、変更の手間などを比較し、考察を付けなさい}{0}
{パターンの構成要素にアニメーションをつけて見え方の変化を報告しなさい。
錯視図形であるともっと良い。思いつかない場合には、ビデオ内のモーガンのね
じれのひもに別のアニメーションをつけてもよい。}{0}
}
\Probs{パターンを使う(2)}{余力問題}{
{長方形以外の図形をパターンを用いて塗りつぶしたものを錯視しなさい}{0}
{配布資料を参考にして、図形を傾いたパターンで塗りつぶしなさい。なた、傾
きの角度にアニメーションがつけられるか調べなさい。}{0}
}
\Probs{文字列の扱い}{演習のビデオ2を見て次の問いに答えよ。}{
{自分の名前を表示する。次に、それを姓と名前の部分に分け、移動したり、色を変えたアニ
メーションをつける。}{0}
{適当な道のりに沿って自分の名前が移動するアニメーションを作成する。}{0}
{文字列表示の属性の\texttt{text-anchor} や\texttt{dominant-baseline} の
違いを例とともに報告する。}{0}}
\Rubric{第3回(5/9)}{ノートの内容}{
今回の内容は少し多くなっている。いくつかの課題の内容をまとめたものを作成
してもよい。}
{{課題1}{35}
  {
	{\texttt{linejoin}の違いについてキャプチャした図があり、十分な考察がある。}
	{一つの\texttt{path} 要素で2 つの長方形を描くことに関して、キャプチャした図とリスト
	があり、十分な考察がある。}
	{穴が開いた正方形を描くソースコードとキャプチャがあり、十分な考察があ
	る。}
	{穴が開いていることを示す十分な証拠を画面の動きからわかり、ソースコードと考察がある。}
	{\texttt{path} 要素の属性\texttt{d} で\texttt{M} と\texttt[m]、
	\texttt{L} と\texttt{l} の違いを例を挙げて説明していて、十分な考察があ
	る。}
	{B\'ezier 曲線で描かれた独自の図形のキャプチャとソースコードがあり、考
	察も十分にある。}
	}
	{
	{\texttt{linejoin}の違いについてキャプチャした図があるが、
考察が十分ではない}
	{一つの\texttt{path} 要素で2 つの長方形を描くことに関して、キャプチャした図とリスト
	があるが、考察が十分ではない。}
	{穴が開いた正方形を描くソースコードとキャプチャがあるが、考察が十分ではない。}
	{穴が開いていることを示す十分な証拠を画面の動きからわかりにくい。ソー
	スコードと考察はある。}
	{\texttt{path} 要素の属性\texttt{d} で\texttt{M} と\texttt[m]、
	\texttt{L} と\texttt{l} の違いを例を挙げて説明しているが、考察が十分で
	はないる。}
	{B\'ezier 曲線で描かれた独自の図形のキャプチャとソースコードがあるが、考
	察が十分ではない。}
	}
	{
	{\texttt{linejoin}の違いについてキャプチャした図がないか、考察がほとん
	どない。}
	{一つの\texttt{path} 要素で2 つの長方形を描くことに関して、キャプチャした図、リスト
	のいずれかがない。考察が不十分であるかない。}
	{穴が開いた正方形を描くソースコードとキャプチャのいずれかがない。考察が
不十分である。}
	{穴が開いていることを示す十分な証拠を画面の動きからわからない。ソー
	スコードと考察のいずれかまたは
両方がない。}
	{\texttt{path} 要素の属性\texttt{d} で\texttt{M} と\texttt[m]、
	\texttt{L} と\texttt{l} の違いを例を挙げて説明していのいずれかまたは
両方がない。}
	{B\'ezier 曲線で描かれた独自の図形のキャプチャ、ソー
スコードがなく、考察が不十分である。}
	}
	{課題2}{25}
	{
	{独自のパターンを利用した図形とソースコードがあり、考察も十分である。}
	{同じ図形でパターンを利用しなかった場合のコードがあり、コードの長さ、変更
	の手間などを比較し、考察が十分にある。}
	{パターンの構成要素にアニメーションをつけた図が3枚以上あり、ソースコード
と十分な考察がある。}
	}
	{
	{パターンを利用した図形があり、ソースコードもあるが、考察も少し足りな
	い。}
	{同じ図形でパターンを利用しなかった場合のコードがない。コードの長さ、変更
の手間などを比較し、考察が十分にない。}
	{パターンの構成要素にアニメーションをつけた図が2枚以下である。ソースコー
ドと十分な考察のいずれかがない。}
	}
	{
	{パターンを利用した図形があるが、ソースコードがない。考察が不十分である。}
	{同じ図形でパターンを利用しなかった場合のコードがない。コードの長さ、変更
の手間などを比較し、考察がない。}
	{パターンの構成要素にアニメーションをつけた図が1枚以下である。ソースコー
ドと考察がない。}
	}
	{課題2余力問題}{25}
	{
	{長方形以外の図形をパターンを用いて塗りつぶした図形とソースコードがあり、
考察も十分である。}
	{傾いたパターンを用いて塗りつぶした図形とソースコードがあり、考察も十分
である。}
	{傾いたパターンにアニメーションを付けたもので塗りつぶした図形とソースコー
ドがあり、考察も十分である。または、できないことの十分な説明がある。}
	}
	{
	{長方形以外の図形をパターンを用いて塗りつぶした図
形とソースコードがあり、考察が少し足りない。}
	{傾いたパターンを用いて塗りつぶした図形とソース
コードがあり、考察が少し足りない。}
	{傾いたパターンにアニメーションを付けたもので塗りつぶした図形とソースコー
ドがあり、考察が少し足りない。または、できないことの説明が少し足りないある。}
	}
	{
	{長方形以外の図形をパターンを用いて塗りつぶした図形とソースコードがない。
考察が足りない。}
	{傾いたパターンを用いて塗りつぶした図形とソース
コードがない。考察が足りない。}
	{傾いたパターンにアニメーションを付けたもので塗り
つぶした図形とソースコードがあり、考察が足りない。
または、できないことの説明が足りないある。}
	}
	{課題3}{25}
	{
	{自分の名前を表示している。それを姓と名前の部分に分け、移動したり、色を変え
たアニメーションをつけている。図とソースコードと十分な考察もある。}
	{適当な道のりに沿って自分の名前が移動するアニメー
ションの図とソースコードと十分な考察がある。}
	{文字列表示の属性の\texttt{text-anchor} や\texttt{dominant-baseline}
の違いが分かるような図があり、ソースコードとともに十分な考察がある。}
	}
	{
	{自分の名前を表示している。それを姓と名前の部分に分け、移動したり、色を変え
たアニメーションをつけている。図とソースコードと考察のいずれかがないか不十分である。}
	{適当な道のりに沿って自分の名前が移動するアニメーションの図とソースコード
と考察のいずれかが不十分である。}
	{文字列表示の属性の\texttt{text-anchor} や\texttt{dominant-baseline}
の違いが図をみてもすぐにわからない。ソースコードがないか考察が少し不十分である。}
	}
	{
	{自分の名前を表示していない。それを姓と名前の部分に分け、移動したり、色を
変えたアニメーションがない。図とソースコードと考察のいずれかがない。}
	{適当な道のりに沿って自分の名前が移動するアニメーションの図とソースコード
と考察のいずれかがない。}
	{文字列表示の属性の\texttt{text-anchor} や\texttt{dominant-baseline}
の違いが図をみてもわからない。ソースコードや考察がない。}
	}
}
\rublicPresenII{第3回(5/9)}
\chapter{イベント処理入門}
\changePage{5/16}
今回の演習の目的は次の通りである。
\begin{itemize}
 \item 使用しているブラウザにおける「開発者ツール」の使い方
 \item 簡単なイベント処理をJavaScriptで処理すること
 \item イベント処理関数をつける要素の位置の理解
\end{itemize}
今回からは課題に\Must と書かれたものを最低行うこと。それ以外のかだいはは
いくつか選択してよい。
\Probs{「開発者ツール」の使い方\Must}{演習のビデオ1を見て次の問いに答えよ。}{
{使用しているブラウザと「開発者ツール」の開き方を記せ。}{0.05}
{「Elements」タブを開き、今までに作成したSVGファイルに対して要素の属性を
直接変える。\ignorespaces}{-0.03}
{「Console」タブを開き、コンソールへの入力位置を確認する。}{0}
{コンソールで適当な算術式($2+3$など)を入力して結果が表示されることを確認
する。}{0}
{SVGファイルを開き、
\newline\texttt{document.getElementsByTagName("polyline");}\newline を実
行し、その結果を報告する。\texttt{polyline}のところは該当する要素名に変
えること}{0}
}
\Probs{クリックイベントの処理}{演習のビデオ2を見て次の問いに答えよ。イベ
ント処理関数の設定は\texttt{window.onload}内で行うこと。}{
{クリックされた円の塗りつぶし以外の属性を表示する。}{0}
{\Must 図形の種類をを変えて、クリックしたとき、複数の属性値を表示させる。文字
列の表示はテンプレートリテラルの形式を用いること}{0}
{\Must\texttt{alert}の代わりに\texttt{console.log}を用いると開発者ツールの
Consoleに表示される。\texttt{alert}を使用した場合との違いを考察せよ。}{0}
{2種類以上の図形を表示させ、その上でクリックしたときに要素名が表示される
ようにプログラムしなさい。}{0}
{\Must 正方形をいくつか置いてクリックすると移動するようにプログラムしなさい。}{0}
{正方形をいくつか置いてクリックすると色が変化するようにプログラムしなさい。}{0}
{異なる種類の要素を置いてクリックすると位置が変化するようにプログラムしなさい。}{0}
}
\Probs{イベント処理をつける場所}{演習のビデオ3の前半部「クリックした位置
に円を移動」を見て次の問いに答えよ。}
{
{\Must 画面全体を覆う長方形の属性\texttt{fill}を\texttt{none}にしたら動作に変
化があるか、その理由も含めて答えなさい。}{0.07}
{\Must 円に対してもイベント処理関数を登録してその上をクリックしたときに移動す
るようにしなさい。}{0}
{\Must 円の上をクリックしたら色が変化するようにしなさい。}{0}
{2つ以上の要素を置き、要素以外のところでクリックしたら最後にクリッ
        クした要素がクリックした位置に移動するようにしなさい。初期値は適
        当に設定してよい。}{0}
{\Must 円の上をクリックすると円の色を表示(3)のリストを実行し、動作することを確
認すること。}{0}
{前問と同じリストに対し、 16行目の\texttt{g}を取り除いて、\texttt{svg}要
素の属性\texttt{id}を\texttt{Canvas}にすると正しく動くか確認する。}{0}
{「クリックした位置に円を移動」において、画面全体を覆う長方形なしで、円
上も含めてクリックした位置を円の中心にするようにできるか検討しなさい。}{0}
}
\Probs{ドラッグ処理}{演習のビデオ3の後半部「マウスのドラッグを処理」を見
て次の問いに答えよ。}
{
{\Must ビデオ内の「マウスのドラッグを処理」を実行し、気になる点を記せ。}{0.07}
{\Must ビデオ内の「マウスのドラッグを処理(改良版)」を実行し、前問の気に
なる点が修正されているか述べなさい。さらに、DOMツリーが変化しているか確
認しなさい}{0}
{\Must 正方形をドラッグするようにしなさい。}{0}
{通常、Windows 上でアイコンなどをドラッグするときと、ここでのドラッグす
るときの相違点を述べ、それを改善しなさい。}{0.07}
{円と正方形の図形に対し、1種類のイベント処理関数でする方法を考えよ。
			 図形の種類が増えてもイベント処理関数に手を付けないようにするには
			 どのような方法があるか検討しなさい。}{0}
       }
\newcommand{\ResultA}{{ \bfseries\normalsize リ 説 図 考}}
\newcommand{\ResultEI}{{ \bfseries\normalsize 説 考}}
\newcommand{\ResultFI}{{ \bfseries\normalsize 図 考}}
\Rubric{第4回(5/16)}{ノートの内容}{
今回からJavaScriptによるプログラミングが始まる。細かい文法の説明を特には
しないので今までのプログラミング言語と比較して違いに気を付けること。
\newline
ルーブリック表の項目の最後の文字はそれぞれの項目の評価である。
リ(プログラム等のリスト)、説(プログラ
ム説明が手書きまたは印刷である)、図(結果のキャプチャ画面)、考( 考察が手
書きまたは印刷である)を意味し、次の記号で評価を示す。
$\times$(不備)、$\triangle$(もう一息)、$\bigcirc$(良い)、
$\circledcirc$(大変良い)
}
{{課題1}{20}
{
  {使用中のブラウザと「開発者ツール」の開き方\ResultFI}
  {SVGファイルに対して要素の属性を
  直接変えた結果に前後の図に開発者ツールの「Elements」タブが表示\ResultA}
  {開発者ツールのコンソールで直接、簡単な算術式や
  \texttt{document}\newline\texttt{.getElementsByTagName}を実行\ResultA}
}
{
  {使用中のブラウザと「開発者ツール」の開き方の図または考察がない\ResultFI}
  {SVGファイルに対して要素の属性を
  直接変えた結果に前後の図に開発者ツールの「Elements」タブがない\ResultA}
  {開発者ツールのコンソールで簡単な算術式や
  \texttt{document}\newline\texttt{.getElementsByTagName}を実行が一部な
  い\ResultA}
}
{
  {使用中のブラウザと「開発者ツール」の開き方の図と考察がないか不十分\ResultFI}
  {SVGファイルに対して要素の属性を
  直接変えた結果に前後の図のいずれかがないか開発者ツールの「Elements」タ
  ブがなく不十分\ResultA}
  {開発者ツールのコンソールで簡単な算術式や
  \texttt{document}\newline\texttt{.getElementsByTagName}を実行がな
  い\ResultA}
}
 {課題2}{30}
 {
 {クリックイベントの処理関数を\texttt{window.onload}で
 登録し、クリックされた円の塗りつぶし以外の属性を表示している。\ResultA}
 {\Must 図形の種類をを変えて、クリックしたとき、複数の属性値をテンプレー
 トリテラルの形式で表示している。\ResultA}
 {\Must\texttt{alert}の代わりに\texttt{console.log}を用いた実行結果があ
 り、\texttt{alert}を使用した場合との違いがある。 \ResultFI}
 {\Must 正方形をいくつか置いてクリックすると移動する。\ResultA}
 {正方形をいくつか置いてクリックすると色が変化する。\ResultA}
 {異なる種類の要素を置いてクリックすると位置が変化する。\ResultA}
 }
 {
 {イベントの処理関数を\texttt{window.onload}で
 登録していない。クリックされた円の塗りつぶし以外の属性を表示している。\ResultA}
 {\Must 図形の種類をを変えて、クリックしたとき、複数の属性値をテンプレー
 トリテラルの形式で一部しか表示していない。\ResultA}
 {\Must\texttt{alert}の代わりに\texttt{console.log}を用いた実行結果があ
 り、\texttt{alert}を使用した場合との違いの考察が不十分 \ResultFI}
 {\Must 単独の正方形を置いてクリックすると移動する。\ResultA}
 {単独の正方形をでクリックすると色が変化する。\ResultA}
 {異なる種類の要素に対してクリックすると位置が変化するようになっていない。\ResultA}
 }
 {
 {}
 }
 {課題3}{25}
 {
 {\Must 画面全体を覆う長方形の属性\texttt{fill}を\texttt{none}にした報告
 \ResultEI}
 {\Must 円にイベント処理関数を登録してその上をクリックしたときに移動す
 る。\ResultA}
 {2つ以上の要素を置き、要素以外のところでクリックしたら最後にクリッ
        クした要素がクリックした位置に移動する。\ResultA}
 {\Must 円の上をクリックすると円の色を表示(3)の動作確認\ResultA}
 {\texttt{svg}要素の属性\texttt{id}を\texttt{Canvas}にした時の動作\ResultA}
 {画面全体を覆う長方形なしで、円上も含めてクリックした位置を円の中心にする\ResultA}
 }
 {
 {}
 }
 {
 {}
 }{課題4}{25}
 {
 {\Must ビデオ内の「マウスのドラッグを処理」の動作の気になる点\ResultFI}
{\Must ビデオ内の「マウスのドラッグを処理(改良版)」で前問の気に
なる点が修正の確認とDOMツリーの変化の確認\ResultA}
{\Must 正方形をドラッグするようにしなさい。}
{Windows 上でアイコンのドラッグとの相違点と改善\ResultA}
{円と正方形の図形に対し、1種類のイベント処理関数でする方法
			 図形の種類が増えてもイベント処理関数に手を付けないようにするには
			 どのような方法の検討\ResultA}
 }
 {
 {}
 }
 {
 {}
 }
}

\rublicPresenII{第4回(5/16)}

\end{document}