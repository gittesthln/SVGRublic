%-*- coding: utf-8 -*-
\input ../beamerHead.tex
\TITLE{10}{1}{Ajaxによる非同期通信}{6/27}
\begin{document}
\frame{\maketitle}
%\frame{\tableofcontents}
\section{Ajaxとは}
\begin{frame}[containsverbatim]
 \frametitle{Ajaxの提唱}
 \begin{itemize}
  \item Asyncronous JavaScript + XML の略称
  \item Webアプリケーションの手段としてサーバーと非同期
(Asyncronous)通信を行いながら、JavaScriptを用いてページを書き換える手法
  \item 名称が与えられただけですでにあった技術の組み合わせ
  \item Google Maps がこの手法を用いたことで有名になった。
 \begin{itemize}
  \item 同期通信で地図をダウンロードすると、完了時までユーザーは操作が不
        可能
  \item 追加のデータが必要になったら、裏方でサーバーと通信を行い、データ
        がそろった時点でページをDOMの技法で書き換える
 \end{itemize}
 \end{itemize}
\end{frame}
\section{Ajaxの例}
\begin{frame}[containsverbatim]
 \frametitle{正$n$角形の頂点の位置だけをサーバーから取得}
 \FIGN{0.75}{test-ajax-start}
 この画面でテキストボックスに適当な数を入れてクリックする
\end{frame}
\begin{frame}[containsverbatim]
\frametitle{}
サーバーから送られてきたデータがメッセージボックスに表示
 \FIGN{0.75}{test-ajax-data}
\end{frame}
\begin{frame}[containsverbatim]
\frametitle{}
 「OK」ボタンを押すと多角形が表示
 \FIGN{0.75}{test-ajax-show}
\end{frame}

\begin{frame}[containsverbatim]
 \frametitle{Ajaxを利用した正多角形の表示--HTMLファイル(1)}
 \LISTN{test-ajax.html}{1}{8}{\footnotesize}
 5行目でJavaScriptファイルを読み込む
\end{frame}
\begin{frame}[containsverbatim]
 \frametitle{Ajaxを利用した正多角形の表示--HTMLファイル(2)}
 \LISTN{test-ajax.html}{9}{last}{\footnotesize}
\end{frame}
\begin{frame}[containsverbatim]
 \frametitle{Ajaxを利用した正多角形の表示--HTMLファイル(2)解説}
 \begin{itemize}
  \item 13行目から18行目で\ELM{svg}     
  \item その中に\ATTR{points}がない\ELM{polygon}がある
  \item 19行目からフォームを定義
	\item \ELM{input}の\ATTR{type}が\VAL{number}
	\item これは整数しか受け付けない。
  \item \texttt{SUBMIT}ボタンが押されると\texttt{getData()}が呼び出され
        る
 \end{itemize}
\end{frame}
\begin{frame}[containsverbatim]
 \frametitle{Ajaxを利用した正多角形の表示}
 \LISTN{Ajax.js}{1}{5}{\scriptsize}
 \framesubtitle{通信オブジェクトを作成し、初期化を行う関数}
 \begin{itemize}
	\item 引数は通信が完了したときに呼び出される関数(コールバック関数)
	\item 配布資料にあった古いブラウザにも対応するコードは省略 
	\item 2行目で裏でサーバーと通信を行う
				\texttt{XMLHttpRequest()}オブジェクトを作成
	\item 作成できたときにはそのオブジェクトに通信が完了時のコール
				バック関数が格納されるプロパティ
				\texttt{onreadystatechange}に指定されたコールバック関数を
				登録(3行目)
	\item 関数の戻り値は作成されたオブジェクト(5行目)
 \end{itemize}
\end{frame}
\begin{frame}[containsverbatim]
 \frametitle{Ajaxを利用した正多角形の表示}
 \framesubtitle{HTML文書上でボタンがクリックされたときに呼び出される関
 数}
 \LISTN{Ajax.js}{7}{19}{\scriptsize}
\end{frame}
\begin{frame}[containsverbatim]
 \frametitle{Ajaxを利用した正多角形の表示--解説}
 \begin{itemize}
	\item 8行目で辺の数を変数\texttt{N}に代入
	\item 3より小さければ正多角形ができないのでメッセージボックスを開く
	\item その後、\VAL{false}を返す。これにより\texttt{SUBMIT}がキャンセル
  \item 13行目でAjax通信を行うオブジェクトを作成。引数にコールバック関数
        \texttt{displayPolygon}を渡している
  \item 通信オブジェクトが作成できたら(14行目)
        \texttt{open}メソッドを用いて通信を開始(15行目でから16行目)
        \begin{itemize}
  \item 初めの引数はメソッド(ここでは\texttt{GET}を指定)
  \item 2番目の引数はデータを要求するURL
  \item \texttt{GET}による通信なので引数も付け加える。引数の
        部分は\texttt{encodeURI()}関数で日本語などをエスケープする
  \item 3番目の引数(オプション)は非同期通信(\VAL{true})を行うことを
        指定
  \item 17行目ではデータの終了を知らせるために\VAL{nul}を送る
        \end{itemize}
 \end{itemize}
\end{frame}
\begin{frame}[containsverbatim]
 \frametitle{Ajaxを利用した正多角形の表示}
 \framesubtitle{コールバック関数}
 \LISTN{Ajax.js}{20}{last}{\scriptsize}
 \begin{itemize}
  \item Ajaxの通信が終了(\texttt{httpObj.readyState}が$4$)し、要求が正常
        終了(\texttt{httpObj.status}が$200$)たときに、送られてきたデータ
        を通常のテキスト(\texttt{httpObj.responseText})として、メッセー
        ジボックスに表示(23行目)
  \item この値はそのまま\ELM{polyon}の\ATTR{points}の値として利用できる
        のでそのまま設定する(24行目と25行目)
 \end{itemize}
\end{frame}
\begin{frame}[containsverbatim]
 \frametitle{Ajaxを利用した正多角形の表示--サーバー側のファイル}
 \LISTN{svg-polygon-ajax.txt}{1}{last}{\scriptsize}
\begin{itemize}
 \item 前の頂点の計算プログラムとほとんど同じ
 \item \texttt{header}関数で送るデータの種類を\texttt{text}に指定
\end{itemize}
\end{frame}
\section{やってみよう}
\begin{frame}[containsverbatim]
 \frametitle{確認して報告}
 \begin{itemize}
  \item  いくつかの正多角形を表示した後、ブラウザでの戻るボタンを押したらどう
 なるか。
  \item ブラウザのアドレスバーに直接\verb+svg-polygon-ajax.php?N=5+と入
        力するとなにが起こるか。
	\item アドレスバーに\Verb+svg-polygon-ajax.php?N=2+としたら何が起こる
				か。問題点を指摘し、修正しなさい。
	\item HTTPのステータスコードについて調べよ。
 \end{itemize}
\end{frame}
\end{document}
\begin{frame}[containsverbatim]
\frametitle{}
\end{frame}

