\documentclass[a4j]{jreport}
\input ../rubricHead.tex
\input ../rubricPresentation.tex
\input ../rubricUnitIHead.tex
\begin{document}
\setcounter{chapter}{4}
\chapter{テキストの挿入と新しい要素の追加}
\changePage{5/23}
今回の演習の目的は次の通りである。
\begin{itemize}
 \item クリックした位置をSVGの図形内に表示する
 \item DOMツリーに新しい要素を追加する
 \item 初期化の段階で要素を作成する
 \item 自前でアニメーションを作る
 \item JavaScriptの配列とオブジェクトの取り扱い
 \item 自前のライブラリーを利用したプログラムの書き直し
\end{itemize}
課題に\Must と書かれたものを最低行うこと。それ以外の課題は
いくつか選択してよい。
\Probs{SVG内にテキストを表示}{演習のビデオ1を見て次の問いに答えよ。}{
 {\Must 「クリックした位置をSVG内に表示」を実行し、位置を表示する
 \texttt{text}要素内に空白がないとエラーが起こることをコンソールで確認せ
 よ。}{0}
 {\Must 28行目にある\texttt{rect}要素を23行目にある\texttt{circle}要素の
 前に置くと起こる不備な点を指摘せよ。
        \begin{itemize}
         \item \ \\[-0.05\baselineskip]
         \item \ \\[-0.05\baselineskip]
         \item \ \\[-0.05\baselineskip]
        \end{itemize}}{0}
 {SVGリスト7.6では同じ\texttt{text}要素内に空白がないがエラーが起きない
 ことを確認せよ。}{0}
 {図形を変えてその位置を示す属性の値を表示するものを作成せよ}{0}
 {\Must 「直線を引く」を実行し、自分の言葉で解説をする。また、ソースコー
 ドとDOMツリーに変化があるか確認せよ。}{0}
 {配布資料155ページ図7.14にあるような色を変えた直線が引けるものを作成せ
 よ。}{0}
 {長方形をドラッグで描くものを作成せよ}{0}
 }
 \newpage
\Probs{プログラムで図形のデータを計算する}{演習のビデオ2を見て次の問いに答えよ。}{
 {\Must 「サイクロイドを描く」を実行せよ。}{0}
 {\Must 正7角形を描け。}{0}
 {\Must 「サイクロイドのアニメーション」を実行せよ。}{0}
 {一年時のProccessingで描いたアニメーションを自前のアニメーションの方法
 で移植せよ。また、両者のコードの比較をせよ。}{0}
 }
\Probs{配列とオブジェクト}{演習のビデオ3の前半を見て次の問いに答えよ。}{
 {\Must ビデオを参考にして、コンソールで適当な配列を操作した結果を報告せ
 よ。コンソールの画面のキャプチャをつけること。}{0}
 {\Must ビデオを参考にして、コンソールで適当なオブジェクト(都道府県名を
 キーに都道府県庁所在地を値、日本語の色名をキーにして英語の色名を値にす
 る)を操作した結果を報告せ
 よ。コンソールの画面のキャプチャをつけること。}{0}
 }
\Probs{自前のライブラリで今までのコードを書き直す}{演習のビデオ3の後半を
 見て次の問いに答えよ。}
 {
 {\Must 関数群を利用して「ドラッグして直線を引く」を書き直せ。}{0}
 {\Must 関数群を利用したときと利用しなかったときのコードの比較をせ
 よ。\\コードの比較:}{0.2}
 {「乱数を使用して円をいくつか描く」において配列に色を追加したプログラム
 を作成せよ。また、色の追加だけで動く理由も述べよ。\\理由}{0.02}
 {形の異なる3角形をランダムに表示するものを作成せよ}{0}
 }
%\newpage
\Rubric{第回(5/23)}{ノートの内容}{
項目の最後の文字は次に示す項目の評価である。
{\bfseries リ}(プログラム等のリスト)、{\bfseries 説}(プログラ
ム説明が手書きまたは印刷である)、{\bfseries 図}(結果のキャプチャ画面)、
{\bfseries 考}( 考察が手書きまたは印刷である)を意味し、次の記号で評価を
示す。
$\times$(不備またはない)、$\triangle$(もう一息)、$\bigcirc$(良い)、
$\circledcirc$(大変良い)
}
{{課題1}{20}
{
  {プログラムリストと解説、実行時のキャプチャ画像、考察が十分にある。
  \ResultA}
  {\texttt{text}要素内に空白がないとエラーが起こることをコンソールで確認。
  \ResultFI}
  {実行後のDOMツリーの変化を示す図がある。\ResultFI}
  {\texttt{rect}要素の位置を変更したときに起こる不具合}
}
{
  {}
}
{
  {}
}
 {課題2}{30}
 {
   {}
 }
 {
   {}
 }
 {
   {}
 }
 {課題3}{25}
 {
   {}
 }
 {
   {}
 }
 {
   {}
 }
 {課題4}{25}
 {
   {}
 }
 {
   {}
 }
 {
   {}
 }
}
 \rublicPresenII{第回(5/23)}

\end{document}