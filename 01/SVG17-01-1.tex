%-*- coding: utf-8 -*-
\input ../beamerHead.tex
\TITLE{1}{1}{演習に関する準備}{4/18}
\begin{document}
\frame{\maketitle}
%\frame{\tableofcontents}
\section{SVGについて}
\begin{frame}[containsverbatim]
 \frametitle{SVGの概略}
 \begin{itemize}
  \item Web上での各種規格を定めている World Wide Web Consortium(W3C)によ
        る2Dの図形を記述する規格
  \item eXtensible Markup Language(XML)の規格に基づいて図形などを定義
        \begin{itemize}
         \item HTML文書のようにタグを用いて情報を記述
         \item タグは自由に定義可能(extensible)
         \item 情報を構造化
        \end{itemize}
  \item SVGには図形を定義する要素、図形の属性(位置、長さ、色など)を時間経過に伴
        い変化させるアニメーション要素がある
  \item XML形式の構造を変化させる手段がある\\
        Document Object Model(DOM)の操作
 \end{itemize}
\end{frame}
 \section{SVGの作成と表示}
\begin{frame}[containsverbatim]
 \frametitle{SVGの図形の表示}
 この授業ではブラウザでSVGの図形を表示\\[1ex]
 
 テキストのキャプチャ画像はChromeの最新版
\end{frame}
\section{授業の準備}
\begin{frame}[containsverbatim]
 \frametitle{授業の進め方(ガイダンスの再掲)}
 \begin{itemize}
\item 配布資料はPDFでネット上に公開
      (\texttt{http://www.hilano.org/hilano-lab})
\item 授業の予習用のビデオ(mp4)は上記のネットで公開
\item 演習の内容を記録するルーズリーフを配布する
  \item ノートに予習内容や課題の解答をまとめる。また、疑問点なども書きこむ
  \item 作成したコードなどは印刷できるように準備をしておく
  \item 予習の内容に関してルーブリック評価表(後で解説)により自己評価する
 \end{itemize}
\end{frame}
\begin{frame}
\frametitle{演習時間内}
 \begin{itemize}
\item 予習内容と配布資料で与えられた課題をグループ内で議論して解く。
%\item グループ内で発表内容をまとめる     
\item グループごとに議論した内容を代表者が報告する。
  \item その日の演習内容についてルーブリック評価表で自己採点する。
\item ルーズリーフは議論した内容やその日のまとめを書いて、授業終了後に提出
  \item 演習室にはプリンタとルーズリーフに閉じるための穴あけ機を設置する
\item 提出されたノートとルーブリック評価表は教員が添削して翌日のセミナ終了後の昼に返却
 \end{itemize}
\end{frame}
\begin{frame}[containsverbatim]
 \frametitle{SVGファイルの作成方法}
\begin{itemize}
 \item テキストエディタ(TeraPadなど)で編集
 \item 文字コードはUTF-8(BOM なし)\\
       BOM = Byte Order Mark
 \item Windowsに標準のメモ帳はUTF-8(BOM なし)で{\color{red}保存できない}
\end{itemize}
% いくつかの画像処理ソフトはSVGで保存することができる。
\end{frame}
\section[ファイルの保存]{テキストエディタにおけるファイルの保存}
\subsection{TeraPad}
\begin{frame}[containsverbatim]
 \frametitle{ファイルの保存方法---TeraPad}
 実際に画面を見せます。
\end{frame}
\begin{frame}[containsverbatim]
 \frametitle{ファイルの保存方法---TeraPad(まとめ)}
\begin{itemize}
 \item 「ファイル」メニューを開く
 \item 「文字/改行コード指定保存」を開く
 \item コードは「UTF-8N」を選択
\end{itemize}
 
\end{frame}
\subsection{サクラエディタ}
\begin{frame}[containsverbatim]
 \frametitle{ファイルの保存方法---サクラエディタ(1)}
 実際に画面を見せます。
\end{frame}
\begin{frame}[containsverbatim]
 \frametitle{ファイルの保存方法---サクラエディタ(まとめ)}
\begin{itemize}
 \item 「ファイル」メニューを開く
 \item 「名前を付けて保存」を選ぶ
 \item 「文字コードセット」プルダウンメニューを開く
 \item UTF-8 を選択し、BOM のチェックボックスには{\color{red} チェックを入れない}
\end{itemize}
\end{frame}
\begin{frame}
 \frametitle{プリンタドライバのインストール}
\begin{itemize}
 \item \href{http://www.hilano.org/hilano-lab/svg/driverDownload.pdf}{ドライバー
 のダウンロードとインストールの説明書}(PDF内にリンクあり)に従ってドライ
 バーをインストール
 \item ネットワークプリンタとして使用するの
 でIPアドレスも指定したものにする。
\end{itemize}
\end{frame}
\begin{frame}[containsverbatim]
 \frametitle{これでおしまい}
 ご苦労様でした。予習教材がありますので忘れずに見てください。
\end{frame}
\end{document}
\begin{frame}[containsverbatim]
\frametitle{}
\end{frame}

