\input ../beamerHead.tex
\TITLE{8}{2}{JSONと分割代入}{6/13}
\renewcommand{\theFancyVerbLine}{\footnotesize\arabic{FancyVerbLine}}
\begin{document}
\frame{\maketitle}
%\frame{\tableofcontents}
\section{JSON}
\begin{frame}[containsverbatim]
 \frametitle{JSONとは}
 JSONの情報は\texttt{www.json.org/json-ja.html}で得られる。
 \begin{itemize}
  \item  JSONは JavaScript Object Notation の略
  \item JavaScriptのオブジェクトリテラルを定義する文字列の形式をまねたも
				の
				\begin{itemize}
				 \item オブジェクト形式では全体を\texttt{\{}と\texttt{\}}で囲
               む。
         \item データのリスト(配列)では\texttt{[}と\texttt{]}で囲
               む。
				 \item 各項目はキーと値の間に\texttt{:}で区切り、間を\texttt{,}でつなぐ。
				 \item キーは文字列で、\texttt{"}で挟む%"
				 \item 値はオブジェクト、配列、数、文字列、\texttt{true}、
							 \texttt{false}、\texttt{null}が可能
				\end{itemize}
 \end{itemize}
 この定義から無条件にJavaScriptのオブジェクトリテラルがそのままJSON形式
 として正しいとは限らない。
\end{frame}
\begin{frame}[containsverbatim]
 \frametitle{JSONオブジェクトの利用法}
 \REF{160}
\begin{Verbatim}[numbers=left, fontsize=\scriptsize,numbersep=1pt,
	commandchars=\\\{\},
	codes={\catcode`$=3\catcode`^=7}]
>A=\{x:10,y:20,add:function()\{return(this.x+this.y);\}\}
Object \{x: 10, y: 20\}
>A.add()
30
>B=JSON.stringify(A);
"\{"x":10,"y":20\}"
>typeof B;
"string"
>JSON.parse(B);
$\blacktriangledown$ Object
	x: 10
	y: 20
	$\blacktriangleright$__proto__: Object
\end{Verbatim}
\end{frame}
\begin{frame}[containsverbatim]
 \frametitle{JSONオブジェクトの利用法(解説)}
\begin{itemize}
 \item 1行目でオブジェクトを定義。メソッドとして\Verb+add()+が
			 あり、2つのメンバー\Verb+x+ と\Verb+y+ を加えた値を返す(3行
			 目)
 \item 5行目でオブジェクトをJSON形式に変換。この中にはメソッド
			 の\Verb+add+が含まれていない。
 \item 7行目では変換されたものが文字列であることを確認している。
 \item 9行目では変換後の文字列をオブジェクトに変換している。
\end{itemize}
\end{frame}
\begin{frame}[containsverbatim]
 \frametitle{Storageに配列データを保存(1)}
 \LISTN{pinna-storage-rev2.js}{1}{12}{\scriptsize}
\end{frame}
\begin{frame}[containsverbatim]
 \frametitle{Storageに配列データを保存(1)--解説}
 \begin{itemize}
	\item 5行目で\ELM{input}で\ATTR{text}であるものを得ている。
	\item 6つの\ELM{path}要素を作成するために、\JSKey{Array(6).fill(1)}で
				6回の繰り返しを行っている(6行目)。
	\item 9行目でテキスボックスの値の初期値を\JSKey{localStorage}にある場
				合にはそこから、そうでない場合には初期値を設定している。\\
				\JSKey{localStorage}のデータは文字列なので\JSKey{JSON.parse}で配
				列に変換していることに注意。
	\item その色のリストを利用して、5行目で得たテキストボックスに初期値を
				表示
	\item 11行目でボタンのクリックイベントの処理関数を登録
	\item その関数を呼び出して、初期値で図を表示(12行目)
 \end{itemize}
\end{frame}
\begin{frame}[containsverbatim]
 \frametitle{Storageに配列データを保存(2)}
 \framesubtitle{表示する関数}
 \LISTN{pinna-storage-rev2.js}{13}{last}{\scriptsize}
\end{frame}
\begin{frame}[containsverbatim]
 \frametitle{Storageに配列データを保存(2)--解説}
 \framesubtitle{表示する関数}
\begin{itemize}
 \item テキストボックスの値を色のリストに保存(15行目)\\
			 テキストボックスのリスト\texttt{Cs}に\JSKey{forEach}メソッドを適
			 用していないことに注意(\texttt{Cs}には直接配列のメソッドが適用で
			 きない)
 \item 書き直された色のリストを\JSKey{JSON.stringify()}で文字列に変換し、
			 \JSKey{localStorage}に保存(16行目)
 \item 6つの曲線のパラメータを配列にしたもの(17行目から19行目)に対して、
			 それぞれの曲線の属性を変更()
			 \begin{itemize}
				\item 21行目の代入は分割代入と呼ばれる。配列の一部を対応する変数
							に代入ができる。
				\item 23行目から27行目は図形の道のりの途中の点を計算し、
							\JSKey(map)で配列のデータにしている。
				\item 配列のメソッド\JSKey{join()}は配列の要素を引数の文字列を間
							に挟む文字列を作成する(27行目)
			 \end{itemize}
\end{itemize}
\end{frame}
\section{やってみよう}
\begin{frame}[containsverbatim]
 \frametitle{HTMLとSVGの間でデータを交換}
 \begin{itemize}
	\item 「HTMLとSVGの間でデータを交換」において最終のデータを
				個別に\JSKey{localStorage}に保存するようにする。
	\item 「HTMLとSVGの間でデータを交換」において最終のデータを
				JSON形式で\JSKey{localStorage}に保存するようにする。
	\item 今までに作成したSVG図形の各種パラメータをHTMLのフォームから設定
				できるように変更する
	\item JSONデータを自分で作成し、JavaScriptのオブジェクトリテラルとの違
				いを調べなさい。
 \end{itemize}
\end{frame}
\begin{frame}[containsverbatim]
 \frametitle{JavaScriptのプログラムを作成}
 次のJavaScriptのプログラムに関する事項に答えよ。
 \begin{itemize}
	\item 分割代入を利用して2つの変数の値を入れ替える式を1回の代入で済ませ
				るプログラムを書く。
	\item 次のプログラムを実行するとエラーが発生するか、結果がどうなるか確
				認する。
				\begin{itemize}
				 \item \Verb+[a,,b]= [1,2,3]+
				 \item \Verb+[a,,[b]] = [1,2,[5,6,7]]+
				 \item \Verb+[a,...c] = [1,2,3,4,5]+
				\end{itemize}
	\item このPDFファイルのPinnaの錯視のプログラムで

				\texttt{document.getElementsByTagName('input[type="text"]')}

				に対して\JSKey{forEach}が直接実行できないことを確認する(10行目の
				\JSKey{Colors.forEach(...)}を\JSKey{Cs.forEach(...))}で書くとエ
				ラーが発生する)。
	\item 分割代入は前問のものに対しても可能であるか確認する
 \end{itemize}
\end{frame}
\end{document}
\begin{frame}[containsverbatim]
\frametitle{}
\end{frame}
