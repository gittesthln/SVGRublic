\input ../beamerHead.tex
\TITLE{6}{1}{HTML5入門}{5/30}
\begin{document}
\frame{\maketitle}
\section{HTML5とは}
\begin{frame}[containsverbatim]
 \frametitle{HTML5とは}
 \begin{itemize}
  \item 2014年10月にW3CのRecommendationとなった最新のHTMLの規格
  \item ここではユーザーがデータを入力できる\ELM{form}を中心に解説
 \end{itemize}
\end{frame}
\section{ユーザー入力要素}
\begin{frame}[containsverbatim]
 \frametitle{ユーザ入力要素の例}
    \begin{itemize}
     \item テキストボックス::文字列の入力
     \item パスワード::文字列を入力するが、入力された文字列が非表示
     \item チェックボックス::項目を選択か非選択の2つの状態をとる
     \item ラジオボタン::いくつかのグループ化されたもののうち一つだけが
           選択可能
     \item ボタン::押すことで何らかのアクションを起こさせる
     \item プルダウンメニュー::いくつかの項目を開いてその中から選択可能
    \end{itemize}
\end{frame}
\begin{frame}[containsverbatim]
 \frametitle{\ELM{form}の例}\small
 \LISTN{form-example.html}{1}{12}{\scriptsize}
 36行目以降にある\ELM{form}に対してその中にある入力要素の値に変化
        があったときにおこる\JSKey{change}イベントやクリックイベント処理
        関数を登録
\end{frame}
\begin{frame}[containsverbatim]
 \frametitle{\ELM{form}の例(1)}\small
 \LISTN{form-example.html}{13}{24}{\scriptsize}
 \begin{itemize}
  \item 13行目から17行目で\JSKey{change}イベントの処理関数を定義
  \item イベントが発生したオブジェクト(\JSKey{target})とその要素名
        (\JSKey{tagName})、イベント処理が定義されている要素
        (\JSKey{currentTarget})とイベントが発生した要素の値
        (\JSKey{value})をコンソールに表示
  \item 18行目から22行目で\JSKey{click}イベントの処理関数を定義。内容は
        \texttt{change}と同じ
 \end{itemize}
\end{frame}
\begin{frame}[containsverbatim]
 \frametitle{\ELM{form}の例(2)}\small
 \LISTN{form-example.html}{25}{33}{\scriptsize}
 各入力要素を区別するための文字列のスタイルシート
 \begin{itemize}
  \item 背景の色(\texttt{background})
  \item ボックスの大きさ(\texttt{width})
  \item 表示する文字列の位置(\texttt{text-align}値は\texttt{center}--中央ぞろえ)
  \item フォントの大きさ(\texttt{font-size})
 \end{itemize}
\end{frame}
\begin{frame}[containsverbatim]
 \frametitle{\ELM{form}の例(3)}\small
 \LISTN{form-example.html}{34}{41}{\scriptsize}
 \begin{itemize}
  \item 入力要素は\ELM{input}
  \item 種別は\ATTR{type}で指定
        \begin{itemize}
         \item \VAL{"text"}はテキストボックス(37行目)
         \item \VAL{"password"}はパスワード(38行目)
         \item \VAL{"checkbox"}はチェックボックス(40行目と41行目)
        \end{itemize}
 \end{itemize}
\end{frame}
\begin{frame}[containsverbatim]
 \frametitle{\ELM{form}の例(4)}{\footnotesize}
 \LISTN{form-example.html}{42}{49}{\scriptsize}
 \begin{itemize}
  \item ラジオボタンは\VAL{"radio"}で指定
  \item \ATTR{name}が同じものが同一グループとみなされ、同時に一つしか選
        択できない
  \item 43行目から45行目と47行目から49行目の2つのグループがある
 \end{itemize}
\end{frame}
\begin{frame}[containsverbatim]
 \frametitle{\ELM{form}の例(5)}
 \LISTN{form-example.html}{50}{last}{\scriptsize}
 \end{frame}
\begin{frame}[containsverbatim]
 \frametitle{\ELM{form}の例(5)--解説}
 \begin{itemize}
  \item ボタンは\VAL{button}で定義(51行目)
        \begin{itemize}
         \item \ATTR{value}の属性値がボタン上に表示
         \item ボタンの種類はこのほかに\VAL{submit}(フォームの
               \ATTR{action}で指定された関数が実行される)や
               \VAL{reset}(入力データの初期化)がある
        \end{itemize}
  \item プルダウンメニューは\ELM{select}で定義
    \begin{itemize}
     \item 子要素の\ELM{opation}がリストとして現れる
     \item \ELM{option}の\ATTR{value}の属性値が\ELM{select}の値となる
    \end{itemize}
 \end{itemize}
\end{frame}
 \section{やってみよう}
\begin{frame}[containsverbatim]
 \frametitle{やってみよう}
 \begin{itemize}
  \item \JSKey{change}イベントがいつどこで発生するか確認する
  \item \JSKey{click}イベントがいつどこで発生するか確認する
 \end{itemize}
\end{frame}
\end{document}
\begin{frame}[containsverbatim]
 \frametitle{}
\end{frame}
