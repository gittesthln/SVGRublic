\input ../beamerHead.tex
\TITLE{2}{2}{色、一般の要素のアニメーション}{4/25}
\begin{document}
\frame{\maketitle}
\section{一般の要素にアニメーションをつける}
\subsection{色のアニメーション}
\begin{frame}[containsverbatim]
 \frametitle{色のアニメーション}
\begin{itemize}
 \item 位置以外の属性にアニメーションをつけるには\ELM{animate}を用いる。
 \item 色に関しても例外ではない。
 \item 色はCSS3で定義されている色名を使うので、\ATTR{attributeType}に
			 は\VAL{CSS}を指定
 \item 残りの属性は他のアニメーションの場合と同じ
\end{itemize}
 色のアニメーションの例を示す(デモ)
\end{frame}
\begin{frame}[containsverbatim]
 \frametitle{色のアニメーション--ソースコード\REF{79}}
 \LISTAll{4}{svg-animation-color.svg}
\end{frame}
\begin{frame}[containsverbatim]
 \frametitle{色のアニメーション--ソースコード(解説)}
 \begin{itemize}
	\item 7行目から定義されている円の属性にアニメーションをつけるので
				\ELM{circle}の最後が\texttt{/>}ではなく、\texttt{>}となっ
				ている。
	\item この要素を閉じるために12行目に\texttt{</circle>}がある。
	\item 9行目から10行目では塗りつぶしの色(\ATTR{fill})にアニメーショ
				ンをつけている。
	\item 赤(\VAL{\#ff0000})から黄色(\VAL{\#ffff00})
				に変化する。
	\item \ATTR{atributeType}は\VAL{CSS}となっている
	\item 11行目から12行目では縁取り(\ATTR{stroke})にアニメーションをつ
				けている
	\item 黒(\VAL{\#000000})から青(\VAL{\#0000ff})に変化
 \end{itemize}
\end{frame}
\begin{frame}[containsverbatim]
 \frametitle{やってみよう}
 色のアニメーションで次のことをやってみよう
 \begin{enumerate}
  \item アニメーションの色を変える。
  \item 複数の図形に別の色のアニメーションをつける
 \end{enumerate}
\end{frame}

\begin{frame}[containsverbatim]
 \frametitle{一般の要素のアニメーション}
\begin{itemize}
 \item 長方形の表示位置を決める\ATTR{x}などにアニメーションをつけるため
       にも\ELM{animate}を用いる。
 \item  線形グラデーションで \ATTR{gradientUnits} の値を \VAL{userSpaceOnUse}
にするとグラデーションの開始位置(\ATTR{x1} や \ATTR{y1}) や終了位置
 (\ATTR{x2} や \ATTR{y2}) を図形とは無関係な位置に指定できる。
 \item  これらの属性にアニメーションをつけるとグラデーションの色が横に流れるよ
 うにできる
\end{itemize}
 ここでデモ\REF{82}
\end{frame}
\begin{frame}[containsverbatim]
 \frametitle{グラデーションにアニメーション---ソースコード(1)}
 \LIST{4}{svg-mask.svg}{1}{19}
\end{frame}
\begin{frame}[containsverbatim]
 \frametitle{グラデーションにアニメーション---ソースコード(2)}
 \LIST{4}{svg-mask.svg}{20}{last}
\end{frame}
\begin{frame}[containsverbatim]
 \frametitle{グラデーションにアニメーション---ソースコード(解説)}
 \begin{itemize}
  \item 7行目から18行目にグラデーションが定義されている。
  \item 前回のグラデーションと異なっているのは次の2点。
  \begin{itemize}
   \item \ATTR{gradientUnits}が\VAL{userSpaceOnUse}
   \item グラデーションの大きさが定義されている(\ATTR{x1}など)。
   \item グラデーションの横幅が\texttt{800}で、21行目から22行目の
         \ELM{rect}の幅(\ATTR{width})\texttt{400}の2倍
  \end{itemize}
  \item 9行目から13行目でグラデーションの色が定義
     \item 黄色$\rightarrow$赤$\rightarrow$黄色のパターンが2回繰り返され
           ている。
  \item 14行目から15行目にグラデーションの\ATTR{x1}の、
        16行目から17行目にグラデーションの\ATTR{x2}のアニメーショ
        ンがそれぞれついている。
  \item \ATTR{x1}は\VAL{0}から\VAL{-400}に、\ATTR{x2}
        は\VAL{800}から\VAL{400}に変化するので、グラデーションの横
        幅は変化しない。
  \item 長方形の内部はグラデーションの横の値が\VAL{0}から\VAL{400}
        だけの範囲が塗られるのでグラデーションが右から左に流れる。
 \end{itemize}
\end{frame}
\begin{frame}[containsverbatim]
 \frametitle{やってみよう}
 グラデーションのアニメーションで次のように変えてみよう。
\begin{enumerate}
 \item アニメーションが左から右に流れる。
 \item \ATTR{x1}と\ATTR{x2}のアニメーションのスピードを変える
 \item \ATTR{stop-color}や\ATTR{offset}にアニメーションをつける
\end{enumerate}
\end{frame}
\begin{frame}[containsverbatim]
 \frametitle{これでおしまい}
 このビデオはこれでおしまい。

 アニメーションの途中の値を複数指定する方法を次のビデオで紹介
\end{frame}
\end{document}
\begin{frame}[containsverbatim]
\frametitle{}
\end{frame}
