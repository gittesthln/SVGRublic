\usepackage{array,longtable,amsfonts,amssymb,fancyvrb}
\newcommand{\ShowItem}[1]{%
\ifx\relax#1\else%
{\large$\Box$}\parbox[t]{12zw}{%
  \rule[0.4\baselineskip]{0em}{1em}#1}\newline%
  \expandafter\ShowItem\fi%
}
%
%評価段階の数の増減は\Targetの引数を変える
%
\newcommand{\Target}[5]{%現在は3段階
	\ifx\relax#1\else
	#1{\small \hspace*{-0.2em}\newline(#2\%)}%
	&\ShowItem#3\relax  %この部分を必要に応じて増減
	&\ShowItem#4\relax
	&\ShowItem#5\relax\\ \hline\expandafter\Target\fi}
\newcommand{\TargetN}[6]{%現在は3段階+項目
	\ifx\relax#1\else
	#1{\small \hspace*{-0.2em}\newline(#2\%)}%
	&\ShowItem#3\relax  %この部分を必要に応じて増減
	&\ShowItem#4\relax
	&\ShowItem#5\relax
	&#6\relax\\ \hline\expandafter\TargetN\fi}
\newcommand{\LecName}[1]{\newcommand{\LName}{#1}}
\newcommand{\Date}[1]{\newcommand{\LDate}{#1}}
\newcommand{\Aim}[1]{\newcommand{\LAim}{#1}}
\newcommand{\Contents}[1]{%
  \newcommand{\ContentsItems}{#1}
}
\def\EtitleDo#1,#2\relax{%
  	\def\tmp{&}\tmp\multicolumn{1}{c|}{#1}%
		\ifx\relax#2\else\expandafter\EtitleDo\fi#2\relax
	}
\newcommand{\EvalTitle}[1]{%
	\newcommand{\ETitle}{\hline 評価項目\EtitleDo#1,\relax\\\hline}
  }
\newcommand{\EvalTitleN}[1]{%
	\newcommand{\ETitle}{\hline 評価項目\EtitleDo#1,\relax\\\hline}
  }
\newcommand{\Format}[1]{%
   \newcommand{\LongTable}{\begin{longtable}{|m{2zw}|#1}}
                                                    }
\newcommand{\Rubric}[4]{%
\newpage
\begin{center}
 {\LARGE \LName}\\#1 #2\\[2ex]
% {\Large 学籍番号 \underline{\makebox[10zw]{#1}}}
\end{center}
#3{\small
\LongTable
	 	  \ETitle
\endfirsthead
	 	  \ETitle
\endhead
	\multicolumn{4}{r}{\bfseries 次のページに続きがあります}
\endfoot
\endlastfoot
\Target#4\relax\relax\relax\relax\relax%
%評価段階の数を変更したら\relaxの数を\Targetの引数の数以上にする
\end{longtable}}
\newpage
}
\newcommand{\RubricN}[4]{%
\newpage
\begin{center}
 {\LARGE \LName}\\#1 #2\\[2ex]
% {\Large 学籍番号 \underline{\makebox[10zw]{#1}}}
\end{center}
#3{\small
\LongTable
	 	  \ETitle
\endfirsthead
	 	  \ETitle
\endhead
	\multicolumn{4}{r}{\bfseries 次のページに続きがあります}
\endfoot
\endlastfoot
\TargetN#4\relax\relax\relax\relax\relax\relax%
%評価段階の数を変更したら\relaxの数を\Targetの引数の数以上にする
\end{longtable}}
\newpage
}
