\documentclass[a4j]{jreport}
\input ../rubricHead.tex
\input ../rubricPresentation.tex
\input ../rubricUnitIHead2.tex
\begin{document}
\setcounter{chapter}{7}
\chapter{WebStorageとJSON、Webサーバーの動作確認}
\changePage{6/13}
今回の演習の目的は次のとおりである。
\begin{itemize}
 \item ブラウザ内に情報を保存するWebStorageの利用法
 \item 構造化されたデータを表現する一つの方法としてあるJSON形式を理解し、
			 利用できること
 \item 他の授業でインストールしてあったXAMPPの設定を確認し、Webサーバー
			 が各自のノートパソコンで動作できることを再確認する。
\end{itemize}
課題に\Must と書かれたものを最低行うこと。それ以外の課題は
いくつか選択してよい。
\Probs{バンジオ・ビンナの錯視図形}{演習のビデオ1を見て次の問いに答えよ。}{
 {次のことについて報告をする。
 \begin{itemize}
	\item \Must バンジオ・ビンナの錯視図形のページで、構成する色を変える
				ことができること
	\item \Must \texttt{localStorage}版のバンジオ・ビンナの錯視図形のペー
				ジを表示させ、構成する色を変えた後でページを閉じ、再度表示させた
				ときに最後の色が表示されること
	\item \Must\texttt{localStorage}の値を確認する。また、ブラウザで
				\texttt{localStorage}を直接変えた後、再度表示した結果
	\item \Must\texttt{localStorage}版の9行目のコメントを外した時に上記と
				同様のことを行うとどうなるか。
	\item バンジオ・ビンナの錯視図形のページで表示させる図形の形を変えるパ
				ラメータを外部から指定できるようにする。同様の動作は
				\texttt{localStorage}版でもできるようにする。
 \end{itemize}
 }{0}
 }
% \newpage
\Probs{JSON}{演習のビデオ2を見て次の問いに答えよ。}{
 {
 \begin{itemize}
	\item \Must \texttt{localStorage}版のJSON形式で保存するバンジオ・ビン
				ナの錯視図形のページを表示させ、構成する色を変えた後でページを閉じ、再度表示させた
				ときに最後の色が表示されること。
	\item バンジオ・ビンナの錯視図形のページで表示させる図形の形を変えるパ
				ラメータを外部から指定できるようにしたもののデータをJSON形式で
				\texttt{localStorage}の一つのところに保存するようにする。
 \end{itemize}
 }{0}
 {「HTMLとSVGの間でデータを交換」において次のことができるようにする。
 \begin{itemize}
	\item \Must 最終のデータを個別に\texttt{localStorage}に保存する。
	\item \Must 最終のデータをJSON形式で\texttt{localStorage}に保存するる。
	\item 今までに作成したSVG図形の各種パラメータを\texttt{localStorage}に
				保存できるように変更する
 \end{itemize}
 }{0}
 {分割代入に関する事項を答えよ。
  \begin{itemize}
	\item 分割代入で2つの変数の値を入れ替える式を1回の代入で済ませ
				るプログラムを書け。
	\item 次のプログラムを実行するとエラーが発生するか、結果がどうなるか確
				認する。
				\begin{itemize}
				 \item \Verb+[a,,b] = [1,2,3]+
				 \item \Verb+[a,,[b]] = [1,2,[5,6,7]]+
				 \item \Verb+[a,...c] = [1,2,3,4,5]+
				\end{itemize}
	\item バンジオ・ビンナの錯視のプログラムで得られる

				\texttt{document.getElementsByTagName('input[type="text"]')}

				に対して\texttt{forEach}が直接実行できないことを確認する(10行目の
				\texttt{Colors.forEach(...)}を\texttt{Cs.forEach(...))}で書くとエ
				ラーが発生する)。
	\item 分割代入は前問のものに対しても可能であるか確認する
 \end{itemize}
}{0}				
 }
 \Probs{XAMPPの起動と基本設定の確認}{演習のビデオ3を見て次の問いに答えよ。}{
 {\Must XAMPPの起動ができることを確認する。}{0}
 {\Must XAMPPの設定に関する次のことを答えよ。
\begin{enumerate}
  \item \texttt{XAMPP}のインストール場所
  \item \texttt{http://localhost}にアクセスしたときに表示される画面
  \item \texttt{http://localhost/index.html}と
       \texttt{http://localhost/index.php}にアクセスしたときに表示される
        画面
\end{enumerate}
 \texttt{http://localhost}にアクセスしたときに表示されるファイルは何か考
 }{0}
 {\Must\texttt{localhost}にアクセスしたときに表示されるページの左側にあ
 る \texttt{phpinfo()}から次の項目を探し、その下に記せ。
  \begin{itemize}
	 \item \texttt{Document\textunderscore Root}\hfill
				 \underline{\makebox[20zw]{}\rule{0em}{1.2\baselineskip}}
 \item \texttt{Loaded Configuration File}\hfill
				 \underline{\makebox[20zw]{}\rule{0em}{1.2\baselineskip}}
  \item Apacheの設定ファイルの名称と所在\hfill
				 \underline{\makebox[20zw]{}\rule{0em}{1.2\baselineskip}}
  \item \texttt{php.ini}がある場所\hfill
				 \underline{\makebox[20zw]{}\rule{0em}{1.2\baselineskip}}
 \end{itemize}
 }{0}
 {\Must 簡単なHPを作成して表示できることを確認せよ。}{0}
 {ビデオを参考にPHP をサーバーモードで起動できるようにせよ。
 次の画面を付けて報告せよ
 \begin{itemize}
	\item コマンドプロンプトのショートカットの作成
	\item ショートカットの作業フォルダの変更
	\item コマンドラインからPHPを実行した結果
	\item \texttt{localhost}で表示されたページ
 \end{itemize}}{0}
 }
%\newpage
\RubricN{第7回(6/6)}{ノートの内容}{
\GradeLegend
}
{
{課題1-1}{10}
{
  {\texttt{var}と\texttt{let}による変数の宣言の違いが例とともに十分にあ
  る。}
}
{
  {同一ブロックにおける\texttt{var}と\texttt{let}による変数の宣言が2回あ
  り違いの説明がある。}
  {入れ子になったブロックにおける\texttt{var}と\texttt{let}による変数の
  宣言がともにあり、違いの説明がある。}
  {入れ子になったブロックにおける\texttt{var}と\texttt{let}による
  内側でのブロックが終了した後の変数の値の違いの説明がある。}
}
{
  {同一ブロックにおける\texttt{var}と\texttt{let}による変数の宣言が2回な
  いか、違いの説明がないか間違っている。}
  {入れ子になったブロックにおける\texttt{var}と\texttt{let}による変数の
  宣言がないか、違いの説明がないか間違っている。}
  {入れ子になったブロックにおける\texttt{var}と\texttt{let}による
  内側でのブロックが終了した後の変数の値の違いの説明がないか間違っている。}
}
{\ResultA}
{課題1-2}{10}
{
  {引数のデータ型が異なる関数を定義し、関数内で仮引数の値を変更するサン
  プルを作成している。}
  {作成した関数をコンソールから動作を十分にチェックしていて考察が正しい。}
}
{
  {引数のデータ型が異なる関数を定義し、関数内で仮引数の値を変更するサン
  プルを作成していが、変更する部分が少し足りない。}
  {作成した関数のコンソールからのチェックが少し足りないか考察が不
  十分である。}
}
{
  {引数のデータ型が異なる関数を定義し、関数内で仮引数の値を変更するサン
  プルの作成が足りない。}
  {定義した関数内で仮引数を変更する部分が足りない。}
  {作成した関数のコンソールからのチェックが足りない。}
  {考察が不十分である。}
}
{\ResultA}
{課題1-3}{10}
{
  {他の言語と変数の宣言の比較が十分なされている。}
  {他の言語と変数のスコープルールの比較が十分なされている。}
  {他の言語と関数のスコープルールの比較が十分なされている。}
}
{
  {他の言語と変数の宣言の比較が\texttt{var}と\texttt{let}でともになされ
  ていない。}
  {他の言語と変数のスコープルールの比較が\texttt{var}と\texttt{let}でともになされ
  ていない。}
  {他の言語と関数のスコープルールの比較がローカルとグローバルの一方でし
  かなされていない。}
  {同一関数名の定義ができるかどうかの説明が不十分である。。}
}
{
  {他の言語と変数の宣言の比較がないか、不十分である。}
  {他の言語と変数のスコープルールの比較がない。}
  {他の言語と関数のスコープルールの比較がない。}
  {同一関数名の定義ができるかどうかの項目がない。。}
}
{\ResultEI}
{課題2-1}{10}
{
  {アニメーションの途中でコンソールから変数の内容を
  \texttt{console.log()}を用いて出力している。}
  {ブロックレベルが異なる変数をチェックしている。}
  {関数についてもチェックしている。}
}
{
  {アニメーションの終了後にコンソールから変数の内容を
  \texttt{console.log()}を用いて出力している。}
  {ブロックレベルが異なる変数をチェックしている。}
  {関数についてチェックしていない。}
}
{
  {コンソールから変数の内容を
  \texttt{console.log()}を用いて出力している。}
  {ブロックレベルが異なる変数をチェックしている。}
  {図がアニメーションが途中になっていないか、途中であることがわからない。}
}
{\ResultEFI}
{課題2-2\newline2-3}{10}
{
  {グローバル変数減少の利点について十分な説明がある。}
  {即時実行関数の利点について十分な説明がある。}
  {グローバル変数減少の方法に関して他の言語との比較がある。}
}
{
  {グローバル変数減少の利点の開発側からの視点がある。}
  {グローバル変数減少の利点のライブラリー利用者側からの視点がある。}
  {即時実行関数の利点について十分な説明がある。}
  {グローバル変数減少の方法に関して他の言語との比較がない。}
}
{
  {グローバル変数減少の利点の開発側からの視点がない。}
  {グローバル変数減少の利点のライブラリー利用者側からの視点がない。}
}
{\ResultEI}
{課題2-4}{10}
{
  {今までの課題でグローバル変数をすべてなくしたものに書き直している。}
  {書き直しの方針について十分な説明がある。}
}
{
  {今までの課題でグローバル変数をほとんどなくしたものに書き直している。}
  {書き直しの方針について説明がある。}
}
{
  {今までの課題でグローバル変数をなくしかたが不十分であるか、全くしてい
  ない。}
  {書き直しの方針について説明がない。}
}
{\ResultA}
{課題3-1}{10}
{
  {配列のメソッドを的確に用いて2つのプログラムを作成している。}
  {いろいろな場合について作成したプログラムをチェックしている。}
}
{
  {\texttt{map()}を用いて5で割った余りの配列を正しく作成している。}
  {\texttt{filter()}を用いて奇数である要素を選び出している。}
  {処理される配列がチェックにふさわしい。}
}
{
  {\texttt{map()}なしで5で割った余りの配列を作成している。}
  {\texttt{filter()}なしで奇数である要素を選び出している。}
  {処理される配列がチェックにふさわしくない。}
}
{\ResultEI}
{課題3-2}{10}
{
  {配列のメソッドを利用するものと利用しないものを
	正しく作成している。}
  {利用している配列のメソッドの種類が十分にある。}
  {配列のメソッドを利用する場合としない場合の違いを比較検討している。}
}
{
  {配列のメソッドを利用するものと利用しないものを作成している。}
  {利用している配列のメソッドの種類が2つしかない。}
  {配列のメソッドを利用する場合としない場合の比較検討が少し不十分
  である。}
}
{
  {配列のメソッドを利用するものと利用しないものを作成していないか、利用
  の仕方が間違っている。}
  {利用している配列のメソッドの種類が1つ以下である。}
  {配列のメソッドを利用する場合としない場合の比較検討がないか不十分
  である。}
}
{\ResultEI}
{課題3-3}{10}
{
  {今までの課題で必要なところを配列のメソッドですべて書き直している。}
  {書き直しの方針について十分な説明がある。}
}
{
  {今までの課題で必要なところを配列のメソッドでほとんど書き直している。}
  {書き直しの方針について十分な説明がある。}
}
{
  {今までの課題で必要なところを配列のメソッドでの書き直しが不十分である
  かほとんどしていない。}
  {書き直しの方針について十分な説明がない。}
}
{\ResultA}
}
\rublicPresenP{第7回(6/6)}

\end{document}