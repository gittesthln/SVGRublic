\input ../beamerHead.tex
\TITLE{9}{2}{クライアントとサーバーでのデータのやり取りの基本}{6/20}
\begin{document}
\frame{\maketitle}
\begin{frame}[containsverbatim]
 \frametitle{データを入力する}
 次のファイルを\texttt{start.php}とする。
 %\footnote{このテキストにしたがっ
 %て PHP の設定をした場合には保存するエンコーディングをUTF-8にする。}
 \LISTN{09-3exchange.php}{1}{last}{\scriptsize}
\end{frame}
\begin{frame}[containsverbatim]
 \frametitle{データを入力する--解説}
 \begin{itemize}
	\item 4行目でファイルがUTF-8でエンコーディングされていることを示してい
				る
	\item 8行目から11行目で入力フォームを定義
  \item \texttt{action} では\texttt{submit}ボタンが押されたときに呼び出さ
       れる関数またはサーバー側のファイルを指定
	\item ここでは\texttt{hello.php}が呼び出される。
 \end{itemize}
\end{frame}
\begin{frame}[containsverbatim]
 \frametitle{データを入力する--実行方法}
 \begin{itemize}
  \item \texttt{start.php}を適当なサーバーにコピー
  \item ブラウザーで\texttt{start.php}を表示
  \item このファイルには PHP のプログラム部分がまったくないので
       拡張子を \texttt{html} にしてもよい
  \item テキストボックスにデータを入力した後、隣にある「送信」ボタンをク
       リック
 \end{itemize}
\end{frame}
\begin{frame}[containsverbatim]
 \frametitle{データを受け取る}
 \LISTN{09-5hello.php}{1}{last}{\scriptsize}
\end{frame}
\begin{frame}[containsverbatim]
 \frametitle{データを受け取る--解説}
 \begin{itemize}
  \item  \texttt{hello.php}を適当なサーバーにコピー
  \item \texttt{start.php}の\ELM{form}で属性\ATTR{method}を
       \texttt{POST}にしている(8行目)のでこのフォームに含まれる\texttt{input}で
       指定されている値はスーパーグローバル変数\texttt{\$\_POST} の連想配
       列のなかに入れられる。
  \item 値は\texttt{\$\_POST['user']}で参照可能
  \item 9行目で入力された文字があるかどうかの判定
  \item 空の文字列ではないときには10行目の文が実行される
       のでメッセージを表示
  \item 空の文字列の場合には13行目から15行目までで「戻る」ボタンが表示
  \item このボタンを表示する部分はHTMLの要素で記述
  \item その後に
       PHPで書かれている9行目の\texttt{else}節のを閉じるために16行目から18
       行目で閉じるための\texttt{\}}を挿入
 \end{itemize}
\end{frame}
\begin{frame}[containsverbatim]
 \frametitle{やってみよう}
 \begin{itemize}
  \item 上記のプログラムを実行した結果を報告
	\item 9行目の\texttt{method=\"POST\"}を
				\texttt{method=\"PUT\"}に変更し\texttt{hello.php}の9行目と10行
				目の\Verb+$_POST+を\Verb+$_GET+に変更したもので同じように実行さ
				せて動作することを確認
	\item \texttt{method=\"POST\"}との\texttt{method=\"PUT\"}による違いが
				あるか調べる。
%  \item テキストボックスに\texttt{<b>Foo</b>}と入力した結果について報告
%				し、問題点を報告
 \end{itemize}
\end{frame}
\begin{frame}[containsverbatim]
 \frametitle{サーバーに伝えられる情報\REF{214}}
 \LISTN{server.txt}{1}{last}{\scriptsize}
\end{frame}
\begin{frame}[containsverbatim]
 \frametitle{サーバーに伝えられる情報(解説)}
 \begin{itemize}
	\item 6行目で外部のCSSファイル(table.css)を読み込む。
				\begin{itemize}
				 \item \ELM{link}の\ATTR{href}でファイルを指定
				 \item \ATTR{rel}で\Verb+stylesheet+を指定
				\end{itemize}
	\item サーバーに伝えられる情報はスーパーグローバル変数\texttt{\$\_SERVER} の連想配
       列内にある
	\item この配列内のキーをすべてわたるために
				\Verb+foreach($Array as $key=>$val)+形式の構文を用いる。
				\begin{itemize}
				 \item \Verb+$key+にはキーの値、\Verb+$val+にはそれに対応する配列
							 の値が入れられる(変数名は何でもよい)
				 \item \Verb+$Array[$key]=$val+の関係が成立
				\end{itemize}
 \end{itemize}
\end{frame}
\begin{frame}[containsverbatim]
 \frametitle{サーバーに伝えられる情報(解説つづき)}
	これらを並べてテーブルの形で出力(12行目から17行目)
				\begin{itemize}
				 \item 12行目の\Verb+<<<+はヒアドキュメントと呼ばれる形式で文字
							 列を定義することを示す。
				 \item この後の文字列\Verb+_EOL_+はこの文字列の最後を示すための
							 マークを定義
				 \item 同じ文字列が17行目にある。ここで終了となる。
				 \item この文字列は行の先頭に置く。先頭に空白があってはいけない。
				 \item 終了の文字列以外には分の終了を示す\Verb+;+以外は書けない。
				 \item 14行目と15行目にある変数は展開される。
				\end{itemize}
\end{frame}
\begin{frame}[containsverbatim]
 \frametitle{CSSファイル}
 6行目で読み込まれるCSSファイル
 \LISTN{table.css}{1}{last}{\scriptsize}
\end{frame}

\begin{frame}[containsverbatim]
 \frametitle{CSSファイル(解説)}
 \ELM{table}を用いないで表組するCSSとなっている。
 \begin{itemize}
	\item クラス\texttt{table}は\ELM{table}をまねるために\ATTR{display}に
				\texttt{table}を指定
	\item クラス\texttt{Row}は\ELM{tr}をまねるために\ATTR{display}に
				\texttt{table-row}を指定
	\item クラス\texttt{Cell}は\ELM{td}をまねるために\ATTR{display}に
				\texttt{td}を指定
	\item さらにテキストを中央揃えに指定
 \end{itemize}
\end{frame}
\begin{frame}[containsverbatim]
 \frametitle{サーバーに伝えられる情報(確認しよう)}

 これらの結果を見てどのようなデータが送られているか調査すること
\end{frame}
\end{document}
\begin{frame}[containsverbatim]
 \frametitle{}
\end{frame}
