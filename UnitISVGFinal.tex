\documentclass[a4j]{jreport}
\input rubricHead.tex
%\input rubricPresentation.tex
\input rubricUnitIHead2.tex


\newcommand{\TargetF}[6]{%現在は3段階+項目
	\ifx\relax#1\else
	#1{\small \hspace*{-0.2em}\newline#2点}%
	&\ShowItem#3\relax  %この部分を必要に応じて増減
	&\ShowItem#4\relax
	&\ShowItem#5\relax
	&#6%\relax
	\\ \hline\expandafter\TargetF\fi}

\newcommand{\makeSpace}{\rule[-0.5\baselineskip]{0em}{1.5\baselineskip}}
\renewcommand{\thesection}{\arabic{section}}
 \renewcommand{\thepage}{成績評価-\arabic{page}}
 \newcommand{\Note}[1]{%
 #1{\small \hspace*{-0.2em}\newline5点}
 	&\ShowItem
   {与えられた課題のほとんどを行っている。}
   {結果が正しい}
	 {課題の内容や注意点が自分の言葉でまとめられている。}
   {考察が十分ある。}\relax
  &\ShowItem
   {必須課題を中心に行っている。}
   {結果がおおむね正しい}
	 {課題の内容や注意点がまとめられている。}
   {考察が少し足りない。}\relax
  &\ShowItem
   {報告の課題が少ない。}
   {結果が一部正しくない}
	 {課題の内容が他の文献からの引用をほとんどそのまま利用している。}
   {考察が足りないか全くない。}\relax
	 &\PointsV\\\hline}
\newcommand{\Cent}[1]{\hspace*{\fill}#1\hspace*{\fill}\rule{0em}{1ex}}
\newcommand{\PointsV}{\ShowNo{5}{1}}
\newcommand{\PointsX}{\ShowNo{10}{1}}
\newcounter{NN}
\newcounter{STEP}
\newcommand{\ShowNo}[2]{%
  \setcounter{NN}{#1}\setcounter{STEP}{-#2}\Cent{#1}\ShowNoDo}
\newcommand{\ShowNoDo}{%
  \addtocounter{NN}{\value{STEP}}\ifnum\value{NN}>-1\newline\Cent{\arabic{NN}}%
  \expandafter\ShowNoDo\fi
}
\begin{document}
 \begin{center}
  2017年度情報メディア専門ユニットI \\
  {\Large SVGで始めるWeb Graphics の成績評価表}
 \end{center}
 \section{採点区分と評価}
情報メディア専門ユイットI SVGで始めるWeb Graphics の成績評価のための基準
を示すものである。採点の項目と配点は次のとおりである。
\begin{itemize}
 \item 毎回の演習内容の評価(50点)\\
       演習課題の報告のノートの内容を1回につき5点満点で評価する。必
       須課題だけの結果では最高で3点となる。演習を欠席したところは最終提
       出日までに補うこと。また、事前に評価のチェックも行うので積
       極的に利用しすること。
 \item 最終レポート(40点)\\
       評価基準は後述のルーブリック評価を参照のこと。基準点は30点。\\
       レポートの内容は次のいずれかとする。
       \begin{itemize}
        \item JavaSCriptを用いたSVGまたはHTMLを利用したアプリケーション。
              単に、ページを手打ちしただけのものは避け
              ること。表紙も含めて最低25ページ以上あること。
        \item 演習内容をまとめたもの。いくつかの項目に限って取り上げるこ
              と。例として挙げる図形などはテキストの
              ものと類似していてはいけない。表紙を含めた最低ページ数は30
              ページである。
       \end{itemize}
 \item 最終発表(20点)\\
       発表時間は7分を基準とする。
			 評価基準は後述のルーブリック評価を参照のこと。基準点は10点。       
\end{itemize}
評価点の合計は110点である。60点以上を合格とする。また、S評価は90点以上を
予定している。
\section{提出期限と返却期日}
提出物等に関する注意は次のとおりである。
\begin{itemize}
 \item 演習の最終回(7月25日)は最終レポートに関するプレゼンテーションを各自が行う。
       発表の順番は当日決定する。
 \item 演習終了後に演習のノートをすべて提出する。
 \item 最終レポートとプレゼンテーション資料は7月25日中にメールに添付の形
       で提出すること。
\end{itemize}
提出されたノートや成績評価に関するルーブリック評価表は試験期間終了日(8月
4日)までに返却を終了する予定である。返却開始日は後日、メーリングリストで連絡をする。
\section{発表までの演習の内容}
7月4日、11日、18日の演習は最終発表に向けた準備を行う。新しい課題はない。
また、今までの課題のノートの追加を行ってもよい。演習終了後にノートを提出
すれば提出時における各回の演習内容の評価点をつけて返却する。
\newpage
\newcommand{\ShowRubric}[3]{%
{\small
\LongTable
%	 	  \ETitle
\endfirsthead
	 	  \ETitle
\endhead
	\multicolumn{5}{r}{\bfseries 次のページに続きがあります}
\endfoot
\endlastfoot
\hline
\multicolumn{5}{|c|}{\Large #1(#2点)\makeSpace}\\
\ETitle
#3
\end{longtable}}
}
\begin{center}
 {\LARGE \LName}\\最終評価\\[2ex]
% {\Large 学籍番号 \underline{\makebox[10zw]{#1}}}
\end{center}
\ShowRubric{ノートの評価}{50}
{\Note{4/18}%
\Note{4/25}%
\Note{5/9}%
\Note{5/16}%
\Note{5/23}%
\Note{5/30}%
\Note{6/6}%
\Note{6/13}%
\Note{6/20}%
\Note{6/27}%
}
\newpage
\ShowRubric{最終レポートの評価--形式}{15}{
\TargetF{表紙}{5}
{
  {表紙に必要事項がすべてある。}
}
{
  {表紙に必要事項が一部ない。}
}
{
  {表紙が独立したページではない。}
  {授業名がない。}
  {担当教員名がない。}
  {提出日がない。}
  {学籍番号がない。}
  {氏名がない。}
}
{\PointsV}
{体裁}{10}
{
  {分量が十分ある。}
  {章、節の分け方が適切である。}
  {図の大きさや内容が適切である。}
  {図、表の説明文が内容を的確に表している。}
  {図の説明文が図の下にある。}
  {表の説明文が表の上にある。}
  {図、表の番号が正しくついている。}
  {文体が「である調」に統一されている。}
	{参考文献が十分ある。}
	{参考文献の引用形式が正しい。}
}
{
  {分量が指定よりわずかに足りない。}
  {章、節の分け方が一部不適切である。}
  {図の大きさや内容に一部不備な部分がある。}
  {図、表の説明文が内容を十分に表していない。}
%  {図の説明文が図の下にある。}
%  {表の説明文が表の上にある。}
  {図、表の番号が一部正しくついていない。}
  {文体が一部「である調」になっていない。}
	{参考文献が少し足りない。}
	{参考文献の引用形式が十分でない。}
}
{
  {分量が指定量より1割以上少ない。}
  {章、節の分け方が不適切である。}
  {図の大きさや内容に不要な部分が多くある。}
  {図、表の説明文が内容がほとんど同じものになっている。}
  {図や表の説明文がない。}
  {図、表の番号が正しく付いていないか付けていない。}
  {文体が「である調」に統一されていない。}
	{参考文献が足りない。}
	{参考文献にある図書の引用形式に、著者名または編集者名がない。}
	{参考文献にある図書の発行年がない。}
	{参考文献にある図書の出版社名がない。}
	{参考文献にある論文の引用形式に雑誌名、発行年、ページ数がない。}
	{参考文献にあるURLのページ名がない。}
	{参考文献にあるURLの閲覧日がない。}
}
{\PointsX}
\relax\relax\relax\relax\relax\relax}
\newpage
\ShowRubric{最終レポートの評価(内容)--作品制作}{25}
{\TargetF{企画}{5}
{
  {作成した作品の制作意図がはっきりと書かれている。}
  {作品の説明するために十分な図が含まれている。}
	{作品が使いやす、見やすいような工夫の提案がある。}
}
{
  {作成した作品の制作意図の説明が少し曖昧である。}
  {作品の説明するための図が少なすぎるか、多すぎる。}
	{作品が使いやす、見やすいような工夫の提案が少しある。}
}
{
  {作成した作品の制作意図の説明が明確になっていない。}
  {作品の説明するための図が少なすぎるか、多すぎる。}
	{作品が使いやす、見やすいような工夫の提案がほとんどない。}
}
{\PointsV}
{技術}{10}
{
  {データをもとにして、ページなどが作成されている。}
  {データの処理に関して重要な部分のコードの解説がある。}
	{自分で工夫したことについて十分な説明がある。}
	{コードの質が非常に良い。}
  {付録としてすべてのコードがレポート内にある。}
}
{
  {データをもとにして、ページなどが作成するようにできるところが一部残っ
	ている。}
  {データの処理に関して重要な部分のコードの解説が少し不十分である。}
	{自分で工夫したことについて説明が少し足りない。}
	{コードの質が良い。}
  {付録として一部のコードがレポート内にない。}
}
{
  {データをもとにして、ページなどが作成するようにできるところが多くある。}
  {データの処理に関して重要な部分のコードの解説がないか、足りない。}
	{関数の使用が少なすぎる。}
	{コードのインデントがほとんどない。}
	{変数名が分かりにくい}
  {付録としてすべてのコードがレポート内にない。}
}
{\PointsX}
{達成度}{10}
{
  {データの処理に関して重要な部分のコードの解説がある。}
  {意図したとおりに動くという説明が十分にある。}
	{改良点や、やり残した点の言及がある。}
}
{
  {データの処理に関して重要な部分のコードの解説が少し不十分である。}
  {意図したとおりに動くという説明が不十分である。}
	{改良点や、やり残した点の言及が足りない。}
}
{
  {データの処理に関して重要な部分のコードの解説がないか、足りない。}
  {意図したとおりに動くという説明がない。}
	{改良点や、やり残した点の言及がない。}
}
{\ShowNo{10}{1}}
\relax\relax\relax\relax\relax\relax}
\newpage
\ShowRubric{最終レポートの評価(内容)--授業内容のまとめ}{25}{
\TargetF{構成}{10}
{
  {選択した項目の数が適切である。}
	{項目の解説のレベルが適切である。}
  {項目が基本的なものから応用的なものに並んでいる。}
	{まとめの分量が項目により偏りがない。}
	{基本事項の解説とそれにふさわしい自作の例がある。}
}
{
  {選択した項目に一部偏りが見られる。}
	{項目の解説のレベルが一部適切でない。}
  {項目が基本的なものから応用的なものに一部並んでいない。}
	{まとめの分量が項目により偏りがある。}
	{基本事項の解説とそれにふさわしい例がある。}
}
{
  {選択した項目が多岐にわたっているので分野の内容が深くない。}
  {項目の並べ方に統一性がない。}
	{まとめの分量が項目により大いに偏りがある。}
	{配布資料の例やネット上の例をほとんど変更せずに利用している。}
	{基本事項の解説にあっていない例がたくさん見受けられる。}
}
{\PointsX}
{内容}{15}
{
	{ほとんどすべての例を自作している。}
	{例が解説にふさわしい内容である。}
  {例に挙げたコードの重要な点について解説と注意がある。}
	{解説が自分の言葉で正しく述べられている。}
	{解説の長さが適切である。}
}
{
	{ほとんどすべての例を自作している。}
	{一部の例が解説にふさわしくない内容である。}
  {例に挙げたコードの重要な点についてコードの解説と注意が少し足りない。}
	{解説が自分の言葉で一部が他の文献からの引用と見受けられる。}
	{一部の解説の長さが適切でない。}
}
{
	{ほとんどすべての例を自作していない。}
	{例が解説にふさわしくない内容である。}
	{例が他からの引用に近い。}
  {例に挙げたコードの重要な点について解説と注意が足りない。}
	{解説がほとんど文献からの引用である。}
	{多くの解説の長さが適切でない。}
}
{\ShowNo{15}{1}}
\relax\relax\relax\relax\relax\relax
}
\newpage
\ShowRubric{最終発表の評価}{20}{
\TargetF{発表技法}{5}
{	{はっきりと丁寧に説明していた。}
	{発表の際に聴衆の反応を確かめていた。}
	{間の取り方がよかった。}
	{7分に近い時間で行った。}
	{機材の設定や準備がすぐできていた。}
}
	{
	{説明が途切れることが2,3か所あった。}
	{声が少し大きすぎたり小さすぎた。}
	{発表の際に聴衆の方をあまり見ていないか反応を確かめていなかった。}
	{発表時間が8分程度であった。}
	{機材の設定や準備に少し時間がかかった。}
	}
	{
	{説明が途切れることが多くあった。}
	{声が小さすぎて聞き取れなかった。}
	{聴衆のほうを全く見ない、反応を無視して行った。}
	{間がない発表であった。}
	{手元の資料やPC画面を見て発表していた。}
	{発表時間が8分30秒以上であった。}
	{発表時間が6分30秒以下であった。}
	{機材の取り扱いや発表の準備がほとんどできていなかった。}
	}
{\PointsV}
	{発表構成}{5}
	{
	{図や表を使い簡潔にまとめられていた。}
	{項目の内容の分量にばらつきがほどんどなかった。}
	{スライドごとの情報がうまくまとめられていた。}
	{ページの分量と発表時間のバランスが良かった。}
	{引用は適切である。}
	}
	{
	{文字だけの発表で、概略が少しつかみづらかった。}
	{図の内容が少し見づらかった。}
	{項目の内容の分量にばらつきが少しあった。}
	{一つのスライドに複数の項目の情報が一部あった。。}
	{ページの分量が発表時間に対して少し足りない、または多すぎた。}
	}
	{
	{ほとんどのスライドで情報量が少なかった。}
	{一つのスライドに文字を詰めすぎていた。}
	{図が大きすぎたまたは小さすぎた。}
	{スライドごとに情報の詳しさが異なりすぎていた。}
	{スライドの内容が情報ごとにまとまっていなかった。}
	{ページの分量が発表時間に対して足りない、または多すぎた。}
	}
{\ShowNo{5}{1}}
	{発表内容}{10}
	{
	{初めに発表内容に関する概要があった。}
	{発表内容の順序に必然性があった}
	{内容が自分の言葉で述べられていた。}
	{それぞれの項目の関連性とバランスがよかった。}
	{発表したいことが十分に説明されていた。}
	{自分が報告する事項似たする意見が明確であった。}
	}
	{
	{発表内容の順序に必然性が少しなかった。}
	{内容のごく一部に説明不足なところがあった。}
	{内容に関して他からの引用が少し多かった。}
	{それぞれの項目の関連性に少し不十分なところがあった。}
	{発表内容の必要性の説明が少し足りなかった。}
	}
	{
	{内容が少なすぎる。}
	{図を使用していないのでわかりずらい。}
	{スライドの記述と発言内容に差がありすぎる。}
	{内容が多すぎて散漫である。}
	{内容が引用ばかりで自分でまとめた形跡がなかった。}
	{内容の説明が不十分であった。}
	}
{\PointsX}
\relax\relax\relax\relax\relax\relax
}
\end{document}