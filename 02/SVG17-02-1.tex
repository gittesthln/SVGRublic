%-*- coding: utf-8 -*-
\input ../beamerHead.tex
\TITLE{2}{1}{位置の移動のアニメーション}{4/25}
\begin{document}
\frame{\maketitle}
%\frame{\tableofcontents}
\section{アニメーションについて}
\begin{frame}[containsverbatim]
 \frametitle{アニメーションの概略}
 SVGの図形の属性値を時間経過で変化させることができる要素が存在
 \begin{itemize}
  \item 位置の移動など(\texttt{<animateTransform>})
%  \item 色の変化(\texttt{<animateColor>})
  \item 一般の属性の変化(\texttt{<animate>})
\end{itemize}
\end{frame}
 \begin{frame}[containsverbatim]
  \frametitle{アニメーションに共通の属性}
  \footnotesize
  \begin{center}
   \begin{tabular}{|c|l|m{10em}|}\hline
 属性名&\multicolumn{1}{c|}{意味}&\multicolumn{1}{c|}{とりうる値} \\\hline
\texttt{attributeName}&属性名&属性名なら何でも可\\\hline
\texttt{attributeType}&属性値の種類&\texttt{XML}または\texttt{CSS}\\\hline
\texttt{from}&開始時の属性の値&\\\hline
\texttt{to}&終了時の属性の値&\\\hline
\texttt{dur}&変化の継続時間& \texttt{2s}(2秒), \texttt{1m}(1分)\\\hline
\texttt{begin}&開始時間&{時間を与える。例\texttt{2s}(2秒), \texttt{1m}(1分)}\\\hline
\texttt{fill}&終了時の属性値の指定&{\texttt{freeze}(終了値で固定)\newline
                \texttt{remove}(はじめの値に戻る)}\\\hline
\texttt{repeatCount}&繰り返し回数&
    {\texttt{indefinite}は無限回の繰り返し}\\\hline
   \end{tabular}
  \end{center}
 
 \end{frame}
 \section{位置のアニメーション}
\begin{frame}[containsverbatim]
 \frametitle{位置の移動---\texttt{<animateTransform>}}
 次の3種類がある
\begin{itemize}
 \item 平行移動
 \item 回転
 \item 拡大・縮小
\end{itemize}
 まずは平行移動\REF{74}から...
\end{frame}
\begin{frame}[containsverbatim]
 \frametitle{平行移動のアニメーション--ソースコード}
 \LISTAll{4}{svg-animation-transform.svg}
\end{frame}
\begin{frame}[containsverbatim]
 \frametitle{平行移動のアニメーション--ソースコード(解説)}
 \begin{itemize}
  \item 7行目から8行目で水平方向に長い緑の長方形を定義
  \item 10行目から11行目で別の水平方向に長い赤の長方形を定義
  \item 赤の長方形は9行目の\texttt{<g>}要素で$45$度回転
  \item 13行目から14行目で平行移動のアニメーションを定義。アニメーション
        の要素は\texttt{<animateTransform>}
        \begin{itemize}
         \item このアニメーションは2つの長方形を含む6行目の\texttt{<g>}
               要素に対して有効
         \item \texttt{attributeName}属性には\texttt{transform}
         \item \texttt{attributeType}属性は\texttt{XML}
         \item \texttt{type}属性には\texttt{translate}
         \item 開始位置は\texttt{from}属性で定義されている
               \texttt{100,50}\\
               \texttt{translate}の後の\texttt{()}は不要
         \item 終了位置は\texttt{from}属性で定義されている
               \texttt{200,50}\\
               したがって水平方向にまとまって動く
         \item 継続時間は\texttt{dur}属性で定義されて10秒間
               (\texttt{10s})で2つの位置を移動
         \item アニメーション終了時の状態は\texttt{fill}属性で定義され、
               この場合はその位置にとどまる(\texttt{freeze})。
        \end{itemize}
 \end{itemize}
\end{frame}
\begin{frame}[containsverbatim]
 \frametitle{回転のアニメーション\REF{75}}
 $45$度に傾いた赤の長方形だけにアニメーションを付ける。

 赤の長方形は回転しながら平行移動する。(タブを移動します)
\end{frame}
\begin{frame}[containsverbatim]
 \frametitle{回転のアニメーション--ソースコード}
 \LISTAll{4}{svg-animation-rotate.svg}
\end{frame}
\begin{frame}[containsverbatim]
 \frametitle{回転のアニメーション--ソースコード(解説)}
 \begin{itemize}
  \item 前のコードでは赤の長方形を囲む\texttt{<g>}要素があったが、
        今回は省略
  \item 回転のアニメーションを直接、赤の長方形につけている
  \item そのため10行目の要素の終了を示す\texttt{>}の前に\texttt{/}がなく、
        13行目に\texttt{</rect>}があることに注意
  \item 11行目から12行目のアニメーションで14行目から15行目のと異なるのは
        次の通り
  \begin{itemize}
   \item \texttt{type}属性の値が\texttt{rotate}(回転)に
   \item \texttt{from}属性の値が\texttt{0}で\texttt{to}属性の値が
         \texttt{360}なので1回転する
   \item \texttt{repeatCount}属性は繰り返し回数を指定する。ここでは
         \texttt{indefinite}(不定)を指示しているので継続時間が過ぎてもア
         ニメーションは停止しない。
  \end{itemize}
 \end{itemize}
\end{frame}
\begin{frame}[containsverbatim]
 \frametitle{色のアニメーションンは次の動画で...}
 取りあえず位置の移動のアニメーションはここで終わり

 拡大・縮小のアニメーションは各自で確かめること。
\end{frame}
\end{document}
\begin{frame}[containsverbatim]
\frametitle{}
\end{frame}

