\documentclass[a4j]{jreport}
\input ../rubricHead.tex
\input ../rubricPresentation.tex
\input ../rubricUnitIHead.tex
\begin{document}
\setcounter{chapter}{4}
\chapter{テキストの挿入と新しい要素の追加}
\changePage{5/23}
今回の演習の目的は次の通りである。
\begin{itemize}
 \item クリックした位置をSVGの図形内に表示する
 \item DOMツリーに新しい要素を追加する
 \item 初期化の段階で要素を作成する
 \item 自前でアニメーションを作る
 \item JavaScriptの配列とオブジェクトの取り扱い
 \item 自前のライブラリーを利用したプログラムの書き直し
\end{itemize}
課題に\Must と書かれたものを最低行うこと。それ以外の課題は
いくつか選択してよい。
\Probs{SVG内にテキストを表示}{演習のビデオ1を見て次の問いに答えよ。}{
 {\Must 「クリックした位置をSVG内に表示」を実行し、位置を表示する
 \texttt{text}要素内に空白がないとエラーが起こることをコンソールで確認せ
 よ。}{0}
 {\Must 28行目にある\texttt{rect}要素を23行目にある\texttt{circle}要素の
 前に置くと起こる不備な点を指摘せよ。
        \begin{itemize}
         \item \ \\[-0.05\baselineskip]
         \item \ \\[-0.05\baselineskip]
         \item \ \\[-0.05\baselineskip]
        \end{itemize}}{0}
 {SVGリスト7.6では同じ\texttt{text}要素内に空白がないがエラーが起きない
 ことを確認せよ。}{0}
 {図形を変えてその位置を示す属性の値を表示するものを作成せよ}{0}
 {\Must 「直線を引く」を実行し、自分の言葉で解説をする。また、ソースコー
 ドとDOMツリーに変化があるか確認せよ。}{0}
 {配布資料155ページ図7.14にあるような色を変えた直線が引けるものを作成せ
 よ。}{0}
 {長方形をドラッグで描くものを作成せよ}{0}
 }
 \newpage
\Probs{プログラムで図形のデータを計算する}{演習のビデオ2を見て次の問いに答えよ。}{
 {\Must 「サイクロイドを描く」を実行せよ。}{0}
 {\Must 正7角形を描け。}{0}
 {\Must 「サイクロイドのアニメーション」を実行せよ。}{0}
 {一年時のProccessingで描いたアニメーションを自前のアニメーションの方法
 で移植せよ。また、両者のコードの比較をせよ。}{0}
 }
\Probs{配列とオブジェクト}{演習のビデオ3の前半を見て次の問いに答えよ。}{
 {\Must ビデオを参考にして、コンソールで適当な配列を操作した結果を報告せ
 よ。コンソールの画面のキャプチャをつけること。}{0}
 {\Must ビデオを参考にして、コンソールで適当なオブジェクト(都道府県名を
 キーに都道府県庁所在地を値、日本語の色名をキーにして英語の色名を値にす
 る)を操作した結果を報告せ
 よ。コンソールの画面のキャプチャをつけること。}{0}
 }
\Probs{自前のライブラリで今までのコードを書き直す}{演習のビデオ3の後半を
 見て次の問いに答えよ。}
 {
 {\Must 関数群を利用して「ドラッグして直線を引く」を書き直せ。}{0}
 {\Must 関数群を利用したときと利用しなかったときのコードの比較をせ
 よ。\\コードの比較:}{0.2}
 {「乱数を使用して円をいくつか描く」において配列に色を追加したプログラム
 を作成せよ。また、色の追加だけで動く理由も述べよ。\\理由}{0.02}
 {形の異なる3角形をランダムに表示するものを作成せよ}{0}
 }
%\newpage
\Rubric{第回(5/23)}{ノートの内容}{
\GradeLegend}
{{課題1}{25}
{
  {サンプルプログラムに追加、修正があり、リスト、解説、実行時のキャプチャ画像、考察が十分にある。
  \ResultA}
  {\texttt{text}要素内に空白がないとエラーが起こることをコンソールで確認。
  \ResultFI}
  {実行後のDOMツリーの変化を示す図がある。\ResultFI}
  {\texttt{rect}要素の位置を変更したときに起こる不具合が十分指摘されてい
	る。\ResultFI}
}
{
  {プログラムがサンプルのままである。解説、キャプチャ画像の不足(開始時、実行結果のどち
	らかがない)、考察がある。\ResultA}
  {\texttt{text}要素内に空白がないとエラーが起こることをコンソールで確認
	しずらい(キャプチャ画像が小さい。内容が読めない)。\ResultFI}
  {実行後のDOMツリーの変化を示す図(開始時、しばらく後の図のどちらかがな
	い)がある。\ResultFI}
  {\texttt{rect}要素の位置を変更したときに起こる不具合が足りない。証拠の図が不備
	(キャプチャ画像が小さい。内容が読めない)。\ResultEFI}
}
{
  {プログラムがサンプルのままである。解説、キャプチャ画像の不足(開始時、実行結果のどち
	らかがない)、考察がない。\ResultA}
  {\texttt{text}要素内に空白がないとエラーが起こることをコンソールで確認
	していない(キャプチャ画像がない。内容が不適切)。\ResultFI}
  {実行後のDOMツリーの変化を示す図のうち開始時、しばらく後のどちらもな
	いか、図が見にくい。\ResultFI}
  {\texttt{rect}要素の位置を変更したときに起こる不具合がほとんどない。証拠の図が不備
	(キャプチャ画像が小さい。内容が読めない)か足りない。\ResultEFI}
}
 {課題2}{25}
 {
  {起動時にサイクロイド図形を描いている。ソースに十分な追加または修正がある。\ResultA}
  {正7角形の頂点位置を起動時に計算して、表示している。\ResultA}
	{サイクロイドに関するアニメーションが実行できている。ソースに十分な追
	加または修正がある。\ResultA}
	{Proccessingで描いた課題をSVGに移植していて、両者のソースコードとその
	比較がある。\ResultA}
 }
 {
  {起動時にサイクロイド図形のリストの説明が少し不十分か図が小さすぎる。
	ソースはほとんどサンプルのままである。\ResultA}
  {正7角形の頂点位置を起動時に計算している。ほかの多角形に容易に修正できるコード
	になっていない。\ResultA}
	{サイクロイドに関するアニメーションがサンプルのままで実行できている。実行時の図が足り
	ない。\ResultA}
	{Proccessingで描いた課題をSVGに移植しているが、両者のソースコード、図な
	どのどちらかがない、その比較が不十分。\ResultA}
 }
 {
  {起動時にサイクロイド図形のリストの説明が不十分か図が小さすぎる。ソー
	スはサンプルのままである。\ResultA}
  {正7角形の頂点位置を起動時に計算していない。ほかの多角形に修正できるコード
	になっていない。\ResultA}
	{サイクロイドに関するアニメーションがサンプルのままで実行できている。
	実行時の図が一つ以下である。\ResultA}
	{Proccessingで描いた課題をSVGに移植しているが、両者のソースコード、図な
	どのどちらかもない、その比較が不十分。\ResultA}
 }
 {課題3}{25}
 {
   {コンソールで空でない独自の配列(要素に数字、文字列、配列などいろいろ
	 なものがある。)を定義ている。\ResultEI}
	 {定義した配列をいろいろ操作した結果が
	 あり、コンソールの画面のキャプチャがある。\ResultFI}
   {コンソールで空でない独自のオブジェクトを定義している。\ResultEI}
	 {定義したオブジェクトをいろいろ操作した結果があり、
	 コンソールの画面のキャプチャがある。\ResultFI}
 }
 {
   {配列の定義が単純である。\ResultEI}
	 {\texttt{length}プロパティ、配列要素の追
	 加、定義時より大きい要素へのアクセス、配列要素をリストアップする、な
	 どの操作した結果のほとんどがコンソールの画面のキャプチャとしてある。\ResultFI}
   {オブジェクトの定義が単純である。\ResultEI}
	 {\texttt{length}プロパティ、オブジェクト要素の追
	 加、存在しないメンバーへのアクセス、要素をリストアップする、な
	 どの操作した結果のほとんどがコンソールの画面のキャプチャとしてある。\ResultFI}
 }
 {
   {配列の定義がサンプルとほとんど同じである。\ResultEI}
	 {\texttt{length}プロパティ、配列要素の追
	 加、定義時より大きい要素へのアクセス、配列要素をリストアップする、な
	 どの操作した結果のほとんどがコンソールの画面のキャプチャとしてない。\ResultFI}
   {オブジェクトの定義がサンプルとほとんど同じである。\ResultEI}
	 {\texttt{length}プロパティ、オブジェクト要素の追
	 加、存在しないメンバーへのアクセス、要素をリストアップする、な
	 どの操作した結果のほとんどがコンソールの画面のキャプチャとしてない。\ResultFI}
 }
 {課題4}{25}
 {
   {関数群を利用して「ドラッグして直線を引く」のすべてのDOMのメソッドを書き直
	 している(\texttt{setAttribute()}、\texttt{getElementById}などが全く
	 ない)。\ResultA}
	 {関数群を利用したときと利用しなかったときのコードの比較\ResultFI}
	 {「乱数を使用して円をいくつか描く」において配列に色を追加したプログラム
 	 が動作する。色の追加だけで動く理由あり。\ResultA}
	 {3角形をランダムに表示するものが正しく動作する。\ResultA}
 }
 {
   {関数群を利用して「ドラッグして直線を引く」のほとんどすべてのDOMのメソッドを書き直
	 している(\texttt{setAttribute()}、\texttt{getElementById}などが少し残っ
	 ている)。\ResultA}
	 {関数群を利用したときと利用しなかったときのコードの比較の内容が不十分\ResultFI}
	 {「乱数を使用して円をいくつか描く」において配列に色を追加したプログラム
 	 が動作する。色の追加だけで動く理由の説明が足りない。\ResultA}
	 {3角形をランダムに表示するものの動作が少しおかしい。\ResultA}
 }
 {
   {関数群を利用して「ドラッグして直線を引く」のほとんどすべてのDOMのメ
	 ソッドが残っている。関数群の使い方が正しくない。\ResultA}
	 {関数群を利用したときと利用しなかったときのコードの比較の内容がほとん
	 どない\ResultFI}
	 {「乱数を使用して円をいくつか描く」において配列に色を追加したプログラム
 	 が動作がしないかおかしい。色の追加がない。色の追加だけで動く理由の説
	 明が間違っているかほとんどない。\ResultA}
	 {3角形をランダムに表示するものの動作がおかしい。\ResultA}
 }
}
 \rublicPresenII{第回(5/23)}

\end{document}