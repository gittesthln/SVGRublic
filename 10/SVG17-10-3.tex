%-*- coding: utf-8 -*-
\input ../beamerHead.tex
\TITLE{10}{3}{この演習のHP}{6/27}
\begin{document}
\frame{\maketitle}
%\frame{\tableofcontents}
\section{演習のHPについて}
\begin{frame}[containsverbatim]
 \frametitle{演習のHPの更新の方法}
\begin{itemize}
 \item 演習のHPに載せるためのファイルなどの情報のファイルをJSON形式で作成
\begin{itemize}
 \item 定型的なもの(予習内容資料、ビデオなど)は通し番号でプログラムで生
       成
 \item 公開のファイルはファイル名と簡単な説明を組にしている
 \item アクセスした日時をもとに表示する範囲を制御
\end{itemize}
\end{itemize}
\end{frame}
\section{演習データの形式}
\begin{frame}[containsverbatim]
 \frametitle{データの一部}
\LISTN{../Lecture.dat}{248}{280}{\tiny}
\end{frame}
\begin{frame}[containsverbatim]
 \frametitle{データの一部(解説)}
 6月27日分のデータを例に解説する。
\begin{itemize}
 \item キー\texttt{date}は日時と表題がら構成(248行目)
 \item キー\texttt{name}は演習の回数で、ファイル名の一部にもなる(249行目)
 \item キー\texttt{Videos}は演習を構成するビデオ1本分を要素とする配列
       (250行目以下)
       \begin{itemize}
        \item キー\texttt{title}はビデオのタイトル(251行目等)
        \item キー\texttt{files}はビデオ内で参照されるファイル名とそのコメ
              ントからなるオブジェクトの配列(251行目から278行目)
       \end{itemize}
\end{itemize}
\end{frame}
\section{処理プログラム}
\begin{frame}[containsverbatim]
 \frametitle{HP表示プログラム(1)}
 \LISTN{../index.php}{1}{14}{\scriptsize}
\end{frame}
\begin{frame}[containsverbatim]
 \frametitle{HP表示プログラム(1)--解説}
\begin{itemize}
 \item 2行目:タイムゾーンの設定
 \item 3行目:アクセス時の時間(秒数)を得ている。
 \item 4行目:表示する範囲の最後の時間。ここでは3週間後に設定
 \item 5行目:ビデオファイルなどのファイル名の先頭部分の設定
 \item 6行目:演習内容のデータファイル名
 \item 7行目:データファイル名の最終更新日時。HPに修正日を表示するために
       利用
 \item 8行目:データファイルの内容を一つの文字列として読み込み
       (\Verb+file_get_contents+)、そのあとオブジェクトに変換
       (\Verb+json_decode+)
 \item 10行目と11行目:ファイルの更新日時から年と月を得ている。\Verb+date+関数は
       与えられた書式で時間を変換する関数で、年と月を得ている。
 \item 12行目と13行目:年度を求めている。
 \item 14行目:ファイルの更新日時を文字列に変換
\end{itemize}
\end{frame}
\begin{frame}[containsverbatim]
 \frametitle{HP表示プログラム(2)}
 \framesubtitle{HTMLファイルの先頭部分の出力}
 \LISTN{../index.php}{15}{39}{\scriptsize}
\end{frame}
\begin{frame}[containsverbatim]
 \frametitle{HP表示プログラム(2)--解説}
 \framesubtitle{HTMLファイルの先頭部分の出力}
 \begin{itemize}
  \item HTMLファイルの先頭部分をヒアドキュメントを用いて出力
  \item 12行目から13行目で求めた年度の値を埋め込み(20行目と23行目)
  \item 24行目でデータファイルの改訂日を埋め込み
 \end{itemize}
\end{frame}
\begin{frame}[containsverbatim]
 \frametitle{HP表示プログラム(3)}
 \framesubtitle{JSON形式を配列に変換した毎回の演習のデータの処理}
 \LISTN{../index.php}{40}{49}{\scriptsize}\scriptsize
  \begin{itemize}
  \item PHPではオブジェクトの参照は\texttt{.}ではなく\Verb+->+で行う
  \item 43行目で日付のデータを\texttt{/}か空白(\Verb+\s+)で分けて、
        初めの2つを\Verb+$m+(月)と\Verb+$d+(日)に代入(正規表現による分解)
  \item 44行目で授業のある日の最後の時間を作成し、
        これがアクセス日よりも前であれば表示しない(45行目)
  \item また、2週目以降であれば処理を打ち切る(46行目)
  \item 授業の回数から2桁の数字を作るために\texttt{sprintf()}を用いる。
        この関数はC言語の\texttt{printf}関数と同様の動作をするが、戻り値
        が変換後の文字列となる。
  \item 49行目でその回の表題を出力
 \end{itemize}
\end{frame}
\begin{frame}[containsverbatim]
 \frametitle{HP表示プログラム(4)}
 \LISTN{../index.php}{50}{last}{\scriptsize}
\end{frame}
\begin{frame}[containsverbatim]
 \frametitle{HP表示プログラム(4)--解説}
\framesubtitle{演習内容のビデオの部分の出力}
\begin{itemize}
 \item 関数\texttt{showLink()}は与えられたファイルが存在すればリンクとし
       て、なければリンクなしで表示する関数(64行目以降に定義)
 \begin{itemize}
 \item PHPの関数\Verb+file_exists+は与えられたファイルが存在するかどうか
       をチェック(65行目)
 \item 存在すれば拡張子が\texttt{php}、\texttt{js}、\texttt{dat}か
			 \texttt{css}であるかをチェックし、その場合はファイルのダウンロー
			 ドのリンクを作成(67行目)
	\item そうでなければ通常のリンクを作成(69行目)
 \item 存在しなければメッセージだけ表示(68行目)
\end{itemize}
\item 51行目で資料ファイルを表示
 \item 52行目から66行目までのループで各ビデオに関するファイルの表示
 \item ビデオ教材とビデオ内のpdfファイルに関する情報を表示(56行目と57行
       目)
 \item 59行目から61行目で関連のファイルを出力
\end{itemize}
\end{frame}
 \section{やってみよう}
\begin{frame}[containsverbatim]
 \frametitle{確認して報告}
 \begin{itemize}
  \item 正規表現とは何か
  \item ファイルを読み込んでそのままクライアントに返すためには
        \texttt{readfile}関数を使うとよい。この関数を利用するためにはファ
        イルのデータの種類を\texttt{header}関数で指定する。適当なファイ
        ルをサーバーから返すページを作成せよ。
  \item テキストファイルを読み込むための関数として\Verb+file()+と
        \Verb+file_get_contents()+がある。両者の違いについて調べよ。
  \item 簡単なデータを用意して、それからWebページを作成せよ。
 \end{itemize}
\end{frame}
\end{document}
\begin{frame}[containsverbatim]
\frametitle{}
\end{frame}

