\input beamerHead.tex
\TITLE{5}{1}{新しい要素の追加とテキストの挿入}{5/24}
\begin{document}
\frame{\maketitle}
 \section{新しい要素を追加する}
\begin{frame}[containsverbatim]
 \frametitle{直線を引く}
 \framesubtitle{新しい要素を追加する例}
 \begin{itemize}
  \item 初期画面は何も要素がない
  \item ドラッグすると直線が引ける
 \end{itemize}
 \end{frame}
\begin{frame}[containsverbatim]
 \frametitle{直線を引く}
 \framesubtitle{ソースコード(1)}
 \LISTINGFP{7}{svg-js-add-lines.svg}{1}{13}
\end{frame}
\begin{frame}[containsverbatim]
 \frametitle{直線を引く}
 \framesubtitle{ソースコード(2)}
 \LISTINGFP{7}{svg-js-add-lines.svg}{14}{32}
\end{frame}
\begin{frame}[containsverbatim]
 \frametitle{直線を引く}
 \framesubtitle{ソースコード(3)--SVGの要素}
 \LISTINGFP{7}{svg-js-add-lines.svg}{33}{last}
\end{frame}
\begin{frame}[containsverbatim]
 \frametitle{直線を引く--ソースコード解説}
 「円をドラッグする」とプログラムの構造はほとんど同じ
 \begin{itemize}
  \item 9行目から13行目でファイルのロードが終了したときに呼ばれる関数を
        定義
  \item 画面全体を覆う長方形(34行目)を子要素に持つ\ELM{g}に
           \JSKey{mousedown}と\JSKey{mouseup}のイベント処理関数を登録
  \item 14行目から24行目で\JSKey{mousedown}のイベント処理関数を定義
    \begin{itemize}
     \item 名前空間を指定して要素を新規に作成するメソッド
           (\JSKey{createElementNS})で\ELM{line}を生成(15行目)
     \item その要素の直線の始点と
           終点をイベントが発生したカーソル位置で設定(16行目から19行目)
     \item 20行目と21行目で色と直線の幅を設定
     \item 23行目で\JSKey{mousemove}のイベント処理関数の登録
    \end{itemize}
  \item 26行目から28行目で\JSKey{mousemove}のイベント処理関数で
        終点の位置をイベントが発生したカーソル位置に設定
  \item 29行目から31行目で\JSKey{mouseup}のイベント処理関数で
        \JSKey{mousemove}のイベント処理関数を取り除く
 \end{itemize}
\end{frame}
 %\section{やってみよう}
\begin{frame}[containsverbatim]
 \frametitle{やってみよう}
 \begin{itemize}
   \item 何本かの直線を引いた後で次のことを確認
  \begin{itemize}
  \item 右ボタンでクリックしてソースコードが変化しているか
   \item DOMツリーを見て、作成した直線の要素があるか
  \end{itemize}
	\item SVG内に小さな正方形で塗りの色が異なるものをいくつか置き、その正
				方形をクリックした後では、クリックした正方形の塗りの色で直線が引
				けるようにする(\REF{171}図7.15参照)
	\item 長方形をドラッグで描く
 \end{itemize}
\end{frame}
\section{クリックした位置をSVG内に表示}
\begin{frame}[containsverbatim]
クリックした位置をSVGの図形内に表示する
\end{frame}
\begin{frame}[containsverbatim]
 \frametitle{クリックした位置をSVG内に表示}
 \framesubtitle{ソースコード(1)--JavaScript の部分}
 \LISTINGNFP{svg-js-showclickpos-new.svg}{1}{17}
\end{frame}
\begin{frame}[containsverbatim]
 \frametitle{クリックした位置をSVG内に表示}
 \framesubtitle{ソースコード(2)--テキストの表示部分}
 \LISTINGNFP{svg-js-showclickpos-new.svg}{18}{last}
\end{frame}
\begin{frame}[containsverbatim]
 \frametitle{クリックした位置をSVG内に表示--ソースコード解説}
 \begin{itemize}
  \item 「クリックした位置に円が移動」に文字を表示する部分が追加
        (26行目から29行目)
  \item 27行目と29行目の\ELM{text}の中には1個空白が置いてあることに注意
  \item これらの\ELM{text}は23行目の\ELM{g}要素の子要素になっている
  \item \JSKey{click}イベントはこの\ELM{g}要素に付けている(9行目)
  \item クリックの処理でテキストノードの値を書き直すのは12行目と13行目で
        行っている
        \begin{itemize}
         \item \JSKey{firstChild}は対象の要素の先頭の子要素を意味する。
         \item ここでは\ELM{text}の中の文字列を指す
         \item これは通常のSVGの要素ではなく、テキストノードになる
         \item \JSKey{nodeValue}はこのテキストノードの文字列を参照する
         \item ここに空白がないと\JSKey{firstChild}が存在しないので、エ
               ラーが発生(確かめよう)
        \end{itemize}
 \end{itemize}
\end{frame}
\begin{frame}
 \frametitle{やってみよう}
 \begin{itemize}
	\item 表示する図形を円から別なものに変えて、その位置を示す属性の値を表
				示する
	\item 23行目にある\ELM{g}を取り除いて、イベント処理関数を24行目の
				\ELM{rect}につけない理由はなにか
	\item このリストの不備な点を指摘し、改善をする
 \end{itemize}
\end{frame}
\end{document}
\begin{frame}[containsverbatim]
 \frametitle{}
\end{frame}
