\LecName{情報メディア専門ユニットI(演習)}
\Format{m{13zw}|m{13zw}|m{13zw}|}
\EvalTitle{優れている,標準的,改良の余地あり}
\newcommand{\Member}{
\begin{center}
 グループメンバー学籍番号\\[2ex]
  \underline{\makebox[0.22\textwidth]{}}\hfill
  \underline{\makebox[0.22\textwidth]{}}\hfill
  \underline{\makebox[0.22\textwidth]{}}\hfill
  \underline{\makebox[0.22\textwidth]{}}
\end{center}  
}
\newtheorem{Report}{課題}
\makeatletter
\@addtoreset{Report}{section}
\makeatother
\newcommand{\Probs}[3]{%
 \begin{Report}[#1]\upshape
  #2
  \begin{enumerate}\upshape
  \ShowProbs#3\relax\relax
  \end{enumerate}
 \end{Report}
 }
 \renewcommand{\postchaptername}{回}
 \newcommand{\ShowProbs}[2]{\ifx#1\relax\else%
 \item #1\vspace{#2\textheight}\expandafter\ShowProbs\fi}

 \newcommand{\changePage}[1]{
 \renewcommand{\thepage}{第\thechapter 回(#1)-\arabic{page}}
  \setcounter{page}{1}
}
\newcommand{\Must}{{(\bfseries 必須)}}
\newcommand{\ResultA}{{ \bfseries\normalsize リ 説 図 考}}
\newcommand{\ResultLI}{{ \bfseries\normalsize リ 説 考}}
\newcommand{\ResultEFI}{{ \bfseries\normalsize 説 図 考}}
\newcommand{\ResultEI}{{ \bfseries\normalsize 説 考}}
\newcommand{\ResultFI}{{ \bfseries\normalsize 図 考}}

\newcommand{\setDate}[1]{
\def\Date{#1}%
 \renewcommand{\thepage}{第\thechapter 回(#1)-\arabic{page}}
  \setcounter{page}{1}%
	}

	\newcommand{\GradeLegend}{項目の最後の文字は次に示す項目の評価である。
{\bfseries リ}(プログラム等のリスト)、{\bfseries 説}(プログラ
ム説明が手書きまたは印刷である)、{\bfseries 図}(結果のキャプチャ画面)、
{\bfseries 考}(考察が手書きまたは印刷である)を意味し、次の記号で評価を
示す。
$\times$(不備またはない)、$\triangle$(もう一息)、$\bigcirc$(良い)、
$\circledcirc$(大変良い)
}