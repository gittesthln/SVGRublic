\input ../beamerHead.tex
\TITLE{1}{2}{SVGの例}{4/18}
\begin{document}
\frame{\maketitle}
%\frame{\tableofcontents}
\section{SVGの基礎}
\section{SVGの基本}
\begin{frame}[containsverbatim]
 \frametitle{座標系}
\begin{itemize}
 \item 表示画面の左上が原点$(0,0)$
 \item  $y$ 座標は下に行くほど大きくなる
\end{itemize} 

 \begin{center}
\setlength{\unitlength}{1cm}
  \begin{picture}(12,6)(-1,-5)
   \put(0,0){\line(1,0){8}}
   \put(0,0){\line(0,-1){5}}
   \put(0,0.3){\makebox[0em]{$(0,0)$}}
   \put(0,0.3){\makebox[8cm][c]{$x\longrightarrow $ 大}}
   \put(-0.5,0.3){\rotatebox{-90}{\makebox[5cm][c]{$y\longrightarrow $
   大}}}
   \put(4,-2){$\bullet$}
   \put(4.3,-2){$(200,100)$}
   \put(2,-4){$\bullet$}
   \put(2.3,-4){$(100,200)$}
  \end{picture}
 \end{center}
\end{frame}
\begin{frame}[containsverbatim]
 \frametitle{色名}
 \begin{itemize}
  \item 色名による指定\\
        \texttt{red}, \texttt{green}などで指定。CSS3で定義されている色名
        が利用可能。
  \item \texttt{rgb}関数による指定\\
        色の三原色(赤、緑、青)成分それぞれに対して$0\sim255$の値を割り当
        てる。\%を用いることも可能
  \item 16進数による指定\\
        16進数6桁または3桁で指定。先頭に\texttt{\#}を付ける。
  \item 塗らないことを指示する\texttt{none}がある。
 \end{itemize}
 詳しくは配布資料の付録Aを参照
\end{frame}
 \begin{frame}[containsverbatim]
  \frametitle{色の表示の例}
  次の表示はすべて同じ色を表す。
  \begin{itemize}
   \item \texttt{red}
   \item \texttt{rgb(255,0,0)}
   \item \texttt{rgb(100\%,0\%,0\%)}
   \item \texttt{\#FF0000}
   \item \texttt{\#F00}
  \end{itemize}
 \end{frame}

\section{SVGの例}
\subsection{Fickの錯視}
\begin{frame}[containsverbatim]
 \frametitle{Fickの錯視}
 \FIG{0.4}{2}{fick1}
 \begin{itemize}
  \item 水平線と垂直線の長さは同じ
  \item 垂直線が水平線の中央にあり、水平線が分断されているため、垂直線が
        長く見える
 \end{itemize}
\end{frame}
\begin{frame}[containsverbatim]
 \frametitle{Fickの錯視---SVGのコード\REF{14}}
 \LISTAll{2}{fick1.svg}
\end{frame}
\begin{frame}[containsverbatim]
 \frametitle{Fickの錯視---SVGのコード---解説(1)}
 \framesubtitle{1行目:XML宣言}
 このファイルがXML形式で書かれていることを示す。
 \begin{itemize}
  \item ファイルの初めのは\texttt{<?xml}で始まる必要がある。
  \item \texttt{<?xml}の部分に空白があってはいけない。
  \item そのあとの\texttt{version}や\texttt{encoding}の部分は属性
        と呼ばれる。
  \item 属性の後の\texttt{=}の右側は属性値と呼ばれ必ず\texttt{""}
       で囲む
 \end{itemize}
\end{frame}
\begin{frame}[containsverbatim]
\frametitle{Fickの錯視---SVGのコード---解説(2)}
 \framesubtitle{2行目から4行目:ルート要素の宣言}
 \begin{itemize}
  \item XML文書にはルート要素と呼ばれる残りの要素をすべて含む要素が存在
  \item SVGの場合には\texttt{svg}がルート要素
  \item 属性\texttt{xmlns}はSVGの規格が定義されているURLを記述
  \item 3行目の\texttt{xmlns:xlink}は他の規格(ここでは\texttt{xlink})を
        使用することを宣言
  \item 属性\texttt{height}と\texttt{width}はSVGの画像の大きさを指定
  \item ここではともに\texttt{100\%}なのでブラウザ画面全体を指定
 \end{itemize}
\end{frame}
\begin{frame}[containsverbatim]
\frametitle{Fickの錯視---SVGのコード---解説(3)}
 \framesubtitle{5行目以降}
 \begin{itemize}
  \item 5行目はタブに表示されるテキストを定義
  \item 6行目の\texttt{g}要素はいくつかのSVGの要素をグループ化
  \item 属性\texttt{transform}は図形全体を移動、拡大などを指定
  \item \texttt{translate}は平行移動を意味
  \item ここでは横に\texttt{50px},下方に\texttt{150px}移動
  \item 7行目から8行目の\texttt{line}要素は直線(水平線)を定義
        \begin{itemize}
         \item 属性\texttt{x1}と\texttt{y1}は始点の位置$(0,0)$
         \item 属性\texttt{x2}と\texttt{y2}は終点の位置$(100,0)$
         \item 属性\texttt{stroke-width}は線の幅
         \item 属性\texttt{stroke}は線の色
         \item 属性\texttt{stroke-linecap}は直線の終端の形状
        \end{itemize}
  \item 9行目から10行目の\texttt{line}要素は直線(垂直線)を定義
        \begin{itemize}
         \item 始点は$(50,0)$ (水平線の中央)
         \item 終点は水平線の中央から垂直に上方の位置$(50,-100)$
        \end{itemize}
 \end{itemize}
\end{frame}
\begin{frame}[containsverbatim]
 \frametitle{これはおしまい}
 もう少し複雑なものを書いて第1回目の予習はおしまいです。
\end{frame}
\end{document}
\begin{frame}[containsverbatim]
\frametitle{}
\end{frame}

