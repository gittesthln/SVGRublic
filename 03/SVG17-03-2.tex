%-*- coding: utf-8 -*-
\input ../beamerHead.tex
\TITLE{3}{2}{同じ図形で塗りつぶす。--- パターン}{5/9}
\begin{document}
\frame{\maketitle}
%\frame{\tableofcontents}
 \section{SVGのパターン}
\begin{frame}[containsverbatim]
 \frametitle{パターンとは}
 図形が小さな図形の繰り返しで構成されているとき、\ATTR{fill}を小さな図形
 で繰り返すことを指定できる
\end{frame}
\begin{frame}[containsverbatim]
 \frametitle{パターンの例--ヘルマン格子\REF{21}}
水平線と垂直線が交わるあたりにぼんやりと黒い斑点が見える
\end{frame}
\begin{frame}[containsverbatim]
 \frametitle{ヘルマン格子--\ELM{line}を用いて書く(1)}
 \LIST{2}{svg-hermann-line.svg}{1}{11}
\end{frame}
\begin{frame}[containsverbatim]
 \frametitle{ヘルマン格子--\ELM{line}を用いて書く(2)}
 \LIST{2}{svg-hermann-line.svg}{12}{last}
\end{frame}
\begin{frame}[containsverbatim]
 \frametitle{ヘルマン格子--\ELM{pattern}を用いて書く}
 \LISTAll{3}{svg-hermann-pattern.svg}
\end{frame}
\begin{frame}[containsverbatim]
 \frametitle{ヘルマン格子--コードの解説}
 \begin{itemize}
  \item \ELM{line}を用いた場合には表示する分だけの直線を記述する必要があ
        る
  \item ここでは、直線の幅などを修正しやすいように\ELM{defs}内で一つだけ
        \ELM{line}を定義し、水平方向と垂直方向はそれを引用している
  \item パターンを利用しているリストでは7行目から11行目でパターンを定義
        している。
        \begin{itemize}
         \item パターンの大きさは\ATTR{width}と\ATTR{height}で定義
         \item \ATTR{patternUnits}は\VAL{userSpaceOnUse}
         \item 8行目で背景を正方形で定義。大きさは\ELM{pattern}と同じ
         \item 9行目と10行目で水平線と垂直線をそれぞれ定義
        \end{itemize}
  \item 14行目で全体の画像を\ELM{rect}で定義し、\ATTR{fill}をパターンで
        指定。
  \item 書き方はグラデーションのときと同じ
	\item パターンを適用する図形は長方形でなくてもよい
 \end{itemize}
\end{frame}
\begin{frame}[containsverbatim]
 \frametitle{パターンにアニメーションを付ける}
 モーガンのねじれのひも\REF{70}をパターンで描いて、それにアニメーション
 を付ける。
\end{frame}
\begin{frame}[containsverbatim]
 \frametitle{パターンにアニメーションを付ける--ソースコード}
 \LISTN{svg-morgan-animation.svg}{1}{last}{\tiny}
 \end{frame}
\begin{frame}[containsverbatim]
 \frametitle{パターンにアニメーションを付ける--ソースコード(解説)}
\begin{itemize}
 \item このアニメーションでは、細長い長方形の移動が2行で組になっているこ
			 とに注意
 \item 縦横300の正方形を\texttt{lightgray}で塗りつぶし(22行目)。
 \item その上に白い細長い長方形のパターン(6行目から9行目)で塗りつぶし(23
			 行目)。
 \item 24行目でアニメーションのついたパターンでこの白い部分を上塗り
			 \begin{itemize}
				\item ねじれのひもを構成する上の部分が12行目の長方形。2つで構成
				\item 下の部分が平行移動のアニメーションを付けている
			 \end{itemize}
 \item どちらも2つが組になっていることに注意
\end{itemize}
 この例からもわかるように全部の領域を塗らない図形でもよい。
\end{frame}
\begin{frame}[containsverbatim]
 \frametitle{パターンを使ってみよう}
 \begin{itemize}
  \item 今までに作成したSVG画像で繰り返しの部分があるものをパターンで書
        き直し、コードの長さ、変更の手間などの比較、考察をする
  \item ザビニの錯視\REF{68}や輝くヘルマン格子、をパターンを用いて作成する
  \item パターンを構成する要素の一部の属性にアニメーションをつける。錯視
        図形の場合にはその見え方の変化を報告する
 \end{itemize}
\end{frame}
\end{document}
\begin{frame}[containsverbatim]
\frametitle{}
\end{frame}

