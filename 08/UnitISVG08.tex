\documentclass[a4j]{jreport}
\input ../rubricHead.tex
\input ../rubricPresentation.tex
\input ../rubricUnitIHead2.tex
\begin{document}
\setcounter{chapter}{7}
\chapter{WebStorageとJSON、Webサーバーの動作確認}
\changePage{6/13}
今回の演習の目的は次のとおりである。
\begin{itemize}
 \item ブラウザ内に情報を保存するWebStorageの利用法
 \item 構造化されたデータを表現する一つの方法としてあるJSON形式を理解し、
			 利用できること
 \item 他の授業でインストールしてあったXAMPPの設定を確認し、Webサーバー
			 が各自のノートパソコンで動作できることを再確認する。
\end{itemize}
課題に\Must と書かれたものを最低行うこと。それ以外の課題は
いくつか選択してよい。
\Probs{バンジオ・ビンナの錯視図形}{演習のビデオ1を見て次の問いに答えよ。}{
 {次のことについて報告をする。
 \begin{itemize}
	\item \Must バンジオ・ビンナの錯視図形のページで、構成する色を変える
				ことができること
	\item \Must \texttt{localStorage}版のバンジオ・ビンナの錯視図形のペー
				ジを表示させ、構成する色を変えた後でページを閉じ、再度表示させた
				ときに最後の色が表示されること
	\item \Must\texttt{localStorage}の値を確認する。また、ブラウザで
				\texttt{localStorage}を直接変えた後、再度表示した結果
	\item \Must\texttt{localStorage}版の9行目のコメントを外した時に上記と
				同様のことを行うとどうなるか。
	\item バンジオ・ビンナの錯視図形のページで表示させる図形の形を変えるパ
				ラメータを外部から指定できるようにする。同様の動作は
				\texttt{localStorage}版でもできるようにする。
 \end{itemize}
 }{0}
 }
% \newpage
\Probs{JSON}{演習のビデオ2を見て次の問いに答えよ。}{
 {
 \begin{itemize}
	\item \Must \texttt{localStorage}版のJSON形式で保存するバンジオ・ビン
				ナの錯視図形のページを表示させ、構成する色を変えた後でページを閉じ、再度表示させた
				ときに最後の色が表示されること。
	\item バンジオ・ビンナの錯視図形のページで表示させる図形の形を変えるパ
				ラメータを外部から指定できるようにしたもののデータをJSON形式で
				\texttt{localStorage}の一つのところに保存するようにする。
 \end{itemize}
 }{0}
 {「HTMLとSVGの間でデータを交換」において次のことができるようにする。
 \begin{itemize}
	\item \Must 最終のデータを個別に\texttt{localStorage}に保存する。
	\item \Must 最終のデータをJSON形式で\texttt{localStorage}に保存するる。
	\item 今までに作成したSVG図形の各種パラメータを\texttt{localStorage}に
				保存できるように変更する
 \end{itemize}
 }{0}
 {分割代入に関する事項を答えよ。
  \begin{itemize}
	\item 分割代入で2つの変数の値を入れ替える式を1回の代入で済ませ
				るプログラムを書け。
	\item 次のプログラムを実行するとエラーが発生するか、結果がどうなるか確
				認する。
				\begin{itemize}
				 \item \Verb+[a,,b] = [1,2,3]+
				 \item \Verb+[a,,[b]] = [1,2,[5,6,7]]+
				 \item \Verb+[a,...c] = [1,2,3,4,5]+
				\end{itemize}
	\item バンジオ・ビンナの錯視のプログラムで得られる

				\texttt{document.getElementsByTagName('input[type="text"]')}

				に対して\texttt{forEach}が直接実行できないことを確認する(10行目の
				\texttt{Colors.forEach(...)}を\texttt{Cs.forEach(...))}で書くとエ
				ラーが発生する)。
	\item 分割代入は前問のものに対しても可能であるか確認する
 \end{itemize}
}{0}				
 }
 \Probs{XAMPPの起動と基本設定の確認}{演習のビデオ3を見て次の問いに答えよ。}{
 {\Must XAMPPの起動ができることを確認する。}{0}
 {\Must XAMPPの設定に関する次のことを答えよ。
\begin{enumerate}
  \item \texttt{XAMPP}のインストール場所
  \item \texttt{http://localhost}にアクセスしたときに表示される画面
  \item \texttt{http://localhost/index.html}と
       \texttt{http://localhost/index.php}にアクセスしたときに表示される
        画面
\end{enumerate}
 \texttt{http://localhost}にアクセスしたときに表示されるファイルは何か考
 }{0}
 {\Must\texttt{localhost}にアクセスしたときに表示されるページの左側にあ
 る \texttt{phpinfo()}から次の項目を探し、その下に記せ。
  \begin{itemize}
	 \item \texttt{Document\textunderscore Root}\hfill
				 \underline{\makebox[20zw]{}\rule{0em}{1.2\baselineskip}}
 \item \texttt{Loaded Configuration File}\hfill
				 \underline{\makebox[20zw]{}\rule{0em}{1.2\baselineskip}}
  \item Apacheの設定ファイルの名称と所在\hfill
				 \underline{\makebox[20zw]{}\rule{0em}{1.2\baselineskip}}
  \item \texttt{php.ini}がある場所\hfill
				 \underline{\makebox[20zw]{}\rule{0em}{1.2\baselineskip}}
 \end{itemize}
 }{0}
 {\Must 簡単なHPを作成して表示できることを確認せよ。}{0}
 {ビデオを参考にPHP をサーバーモードで起動できるようにせよ。
 次の画面を付けて報告せよ
 \begin{itemize}
	\item コマンドプロンプトのショートカットの作成
	\item ショートカットの作業フォルダの変更
	\item コマンドラインからPHPを実行した結果
	\item \texttt{localhost}で表示されたページ
 \end{itemize}}{0}
 }
%\newpage
\RubricN{第7回(6/6)}{ノートの内容}{
\GradeLegend
}
{
{課題1}{10}
{
  {バンジオ・ビンナの錯視図形について初期状態、実行結果の図が共にある。}
  {バンジオ・ビンナの錯視図形(\texttt{{localStorage}}版)について初期状態、
  実行結果の図が共にある。}
  {バンジオ・ビンナの錯視図形(\texttt{localStorage}版)について
  \texttt{localStorage}の値の確認の図がある}
  {バンジオ・ビンナの錯視図形(\texttt{localStorage}版)についてブラウザを閉
  じた後に、再度表示した場合の説明をするためにコンソールに出力して確認し
  ている。}
}
{
}
{
}
{\ResultA}
{課題1-2}{10}
{
}
{
}
{
}
{\ResultA}
{課題1-3}{10}
{
}
{
}
{
}
{\ResultEI}
{課題2-1}{10}
{
}
{
}
{
}
{\ResultEFI}
{課題2-2\newline2-3}{10}
{
}
{

}
{
}
{\ResultEI}
{課題2-4}{10}
{
}
{
}
{
}
{\ResultA}
{課題3-1}{10}
{
}
{
}
{
}
{\ResultEI}
{課題3-2}{10}
{
}
{
}
{
}
{\ResultEI}
{課題3-3}{10}
{
}
{
}
{
}
{\ResultA}
}
\rublicPresenP{第8回(6/13)}

\end{document}