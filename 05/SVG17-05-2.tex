\input ../beamerHead.tex
\TITLE{5}{2}{初期化で要素を作成と\\自前のアニメーション}{5/23}
\begin{document}
\frame{\maketitle}
 \section{初期化で要素を作成}
\begin{frame}[containsverbatim]
 \frametitle{サイクロイドとは}
 \begin{itemize}
  \item 円が直線上を回転しながら移動するとき、円上の一点が描く軌跡
  \item 円の半径を$r$、回転した角を$\theta$とするとき、点の位置は次の式
        で与えられる
    \begin{align*}
     x &= r(\theta-\sin \theta)     \\
     y &= r(1-\cos\theta)
    \end{align*}
  \item ここでは$\theta$の値を小さく変化させてその間の直線として
        \ELM{path}として描く
 \end{itemize}
\end{frame}
\newcommand{\ElmJC}[4]{\texttt{#1}&#3&#4\\\hline}
\begin{frame}[containsverbatim]
 \frametitle{数学関数}
 \JSKey{Math}オブジェクトのプロパティまたはメソッドとして記述
 \begin{center}\small\setlength{\tabcolsep}{0.1em}
   \begin{tabular}{|c|c|c|}\hline
    \ElmJC{名称}{}{種類}{説明}
\ElmJC{E}{e@$e$}{定数}{自然対数の底($2.71828182\dots$)}
\ElmJC{PI}{pi@$\protect\pi$}{定数}{円周率($\pi=3.141592\dots$)}
\ElmJC{abs(x)}{|x|@$|x|$}{関数}{$x$ の絶対値}
\ElmJC{ceil(x)}{}{関数}{$x$以上の整数で最小な値を返す。}
\ElmJC{cos(x)}{}{関数}{余弦関数 $\cos x$}
\ElmJC{exp(x)}{}{関数}{指数関数 $e^x$}
\ElmJC{floor(x)}{}{関数}{$x$の値を超えない整数}
\ElmJC{log(x)}{}{関数}{自然対数$\log x$}
\ElmJC{max([x1, x2, ..., xN])}{}{関数}{与えられた引数の最大値}
\ElmJC{min([x1, x2, ..., xN])}{}{関数}{与えられた引数の最小値}
\ElmJC{pow(x, y)}{xy@$x^y$}{関数}{指数関数 $x^y$}
\ElmJC{random()}{}{関数}{$0$と$1$の間の擬似乱数を返す}
\ElmJC{round(x)}{}{関数}{$x$の値を四捨五入する}
\ElmJC{sin(x)}{}{関数}{正弦関数 $\sin x$}
\ElmJC{sqrt(x)}{}{関数}{平方根を求める。 $\sqrt{x}$}
\ElmJC{tan(x)}{}{関数}{正接関数 $\tan x$}
\end{tabular}
 \end{center}
\end{frame}
\begin{frame}[containsverbatim]
 \frametitle{サイクロイド--コード(1)}
 \LISTN{svg-cycloid.svg}{1}{19}{\scriptsize}
\end{frame}
\begin{frame}[containsverbatim]
 \frametitle{サイクロイド--コード(1)--解説}
\begin{itemize}
 \item 7行目で円の大きさを設定
 \item 変数\texttt{d}はサイクロイドの形状を示す\ELM{path}の\ATTR{d}に設
       定するための値を保持する。\texttt{"M"}で初期化
 \item 角度を$0$から$360$に設定してそれに対する位置を計算(10行目から15行
       目)
   \begin{itemize}
    \item 11行目で角度をラジアンに変換。\JSKey{Math.PI}は円周率を表す。
    \item 12行目で$x$座標の位置を、13行目で$y$座標の位置をそれぞれ計算
    \item 計算した結果を変数\texttt{d}に追加(14行目)
   \end{itemize}
 \item 16行目でサイクロイドを表示する\ELM{path}に\ATTR{d}を設定
\end{itemize}
\end{frame}
\begin{frame}[containsverbatim]
 \frametitle{サイクロイド--コード(2)}
 \LISTN{svg-cycloid.svg}{20}{last}{\scriptsize}
 表示するSVGの要素は円が転がる水平線(21行目)とサイクロイド(22行目)だけ
\end{frame}
\begin{frame}[containsverbatim]
 \frametitle{やってみよう}
\begin{itemize}
 \item 正$n$角形を描く
 \item 正$7$角形の頂点を一つ置きに結んだ図形を描く
\end{itemize}
\end{frame}

\section{自前でアニメーションを作成}
\begin{frame}[containsverbatim]
 \frametitle{サイクロイドのアニメーション(1)}
 \LISTN{svg-cycloid-animation.svg}{1}{14}{\scriptsize}
%\end{frame}
%\begin{frame}[containsverbatim]
% \frametitle{サイクロイドのアニメーション(1)\\コードの解説}
% \framesubtitle{JavaScript}
 9行目から14行目で初期化の関数が定義
    \begin{itemize}
     \item サイクロイドを表示する\ELM{path}と平行移動、回転のための
           \ELM{g}要素を変数に格納
     \item 13行目で部分的に描く関数\texttt{drawCurve}を呼び出し
    \end{itemize}
\end{frame}
\begin{frame}[containsverbatim]
 \frametitle{サイクロイドのアニメーション(2)}
 \LISTN{svg-cycloid-animation.svg}{15}{29}{\scriptsize}
\end{frame}
\begin{frame}[containsverbatim]
 \frametitle{サイクロイドのアニメーション(2)\\コードの解説}
 \framesubtitle{JavaScript}
 15行目から29行目で\texttt{drawCurve}を定義
    \begin{itemize}
     \item 表示済みの角度を保持する変数\texttt{Current}が1回転
           を示していない(17行目)ならば曲線の次の部分を追加
     \item 次の段階までの部分を計算して追加(18行目から24行目)
     \item 転がる円を回転(26行目)と平行移動(25行目)で移動
     \item \JSKey{setTimeout}関数は指定された関数(第1引数)を経過時間(第2
           引数、単位はms)後に実行するように設定。ここでは
           $100\mathrm{ms}=0.1$秒後に設定
     \item  \JSKey{setTimeout}関数の3番目以降の引数は第1引数の関数の引数
            となる。
    \end{itemize}
\end{frame}
\begin{frame}[containsverbatim]
 \frametitle{サイクロイドのアニメーション(3)}
 \LISTN{svg-cycloid-animation.svg}{30}{last}{\scriptsize}
\end{frame}
\begin{frame}[containsverbatim]
 \frametitle{サイクロイドのアニメーション(3)\\コードの解説}
 \framesubtitle{表示する図形}
 \begin{itemize}
  \item 円が転がる水平線(33行目)とサイクロイドを描くための\ELM{path}(34
        行目)
  \item 回転する円(37行目)と、円周上の点(39行目)とこの点と中心を結ぶ線
        (38行目)を囲む\ELM{g}が2つ
  \item 内側は回転、外側は平行移動
 \end{itemize}
\end{frame}
\begin{frame}[containsverbatim]
 \frametitle{やってみよう}
\begin{itemize}
 \item ここで行っていることは1年の時のProccessingで描くアニメーションと
       同じ
 \item したがって、Proccessingで行ったアニメーションをSVGに移植すること
       が可能
 \item \JSKey{setTimeout()}の次の実行時間を変えてアニメーションのスピー
       ドを変える
\end{itemize}
\end{frame}
\end{document}
\begin{frame}[containsverbatim]
 \frametitle{}
\end{frame}
