\input ../beamerHead.tex
\TITLE{6}{2}{CSSセレクタとHTML要素の追加}{5/30}
\begin{document}
\frame{\maketitle}
\section{CSSとは}
\begin{frame}[containsverbatim]
 \frametitle{CSSについて}
 \begin{itemize}
  \item カスケーディングスタイルシート(CSS)はHTML文書の要素の表示方法を
        指定
  \item CSSはJavaScriptからも制御可能
 \end{itemize}
 \REF{付録9}参照
\begin{itemize}
 \item 要素名で要素を選択するためには、要素名を指定する。
 \item 属性\texttt{id}の属性値で要素を選択するためにはその属性値の前に\texttt{\#}をつける。
 \item 属性\texttt{class}の属性値で要素を選択するためにはその属性値の前に\texttt{.}をつける。
\end{itemize}
 \end{frame}
\begin{frame}[containsverbatim]
 \frametitle{CSSセレクタについて}
 {\footnotesize
\begin{tabular}{|m{13em}|m{17zw}|}\hline
\multicolumn{1}{|c|}{セレクタ}&\multicolumn{1}{c|}{解説}\\\hline
\Verb+*+&任意%&Universal selector&2
\\\hline
\Verb+E+&タイプが \Verb+E+ %&Type selector&1
\\\hline
\Verb+E[foo]+&タイプが \Verb+E+ で属性 \Verb+"foo"+ を持つ%&Attribute
	 %selectors
\\\hline
\Verb+E[foo="bar"]+&タイプが \Verb+E+ で属性 \Verb+"foo"+ の属性値が
     \Verb+"bar"+%&Attribute selectors
\\\hline
\Verb+E[foo~="bar"]+&タイプが \Verb+E+ で属性 \Verb+"foo"+ の属性値が
     空白で区切られた一つが \Verb+"bar"+%&Attribute selectors
 \\\hline
\Verb+E:link+, %\newline
\Verb+E:visited+&まだ訪れたことがない(\texttt{:link})か訪れたことがある
     (\texttt{visited})ハイパーリンクのアンカーである要素%&The link pseudo-classes+&1
\\\hline
\Verb+E:active+, %\newline
\Verb+E:hover+, %\newline
\Verb+E:focus+&ユーザーに操作されている状態中%&The user action
	 %pseudo-classes+&1 and 2
\\\hline
\Verb+E:enabled+, %\newline
\Verb+E:disabled+&使用可能(\texttt{:enable})か使用不可のユーザーインター
     フェイス%&The UI element states pseudo-classes
\\\hline
\Verb+E:checked+&チェックされているユーザーインターフェイス%&The UI element states
	 %pseudo-classes
\\\hline
\Verb+E.warning+&属性\texttt{class} が "warning" である要素%&Class selectors&1
\\\hline
\Verb+E#myid+&属性\texttt{id} の属性値が "myid" %&ID selectors&1
\\\hline
\Verb+E F+&要素\texttt{E} の子孫である要素\texttt{F}%&Descendant
	 %combinator&1
\\\hline
\Verb+E > F+&要素\texttt{E} の子である要素\texttt{F}%&Child combinator
\\\hline
\Verb-E + F-&要素\texttt{E} の直後にある要素\texttt{F}%&Adjacent
	 %sibling combinator
\\\hline
\Verb+E ~ F+&要素\texttt{E} の直前にある要素\texttt{F}%&General sibling
	 %combinator
\\\hline
\end{tabular}
 }
\end{frame}
\iffalse
\begin{frame}[containsverbatim]
 \frametitle{CSSセレクタの注意}
\begin{itemize}
 \item \Verb+E F+ と \Verb+E > F+ の違い
\end{itemize} 
\end{frame}
\fi
\begin{frame}[containsverbatim]
 \frametitle{CSSセレクタを用いて要素を得る}
 \JSKey{querySelector()}と\JSKey{querySelectorAll()}は引数にCSSセレクタ
 を与えて要素のリストを得るメソッド
 
 {\bfseries 例}ラジオボタンでチェックしているものを探す

\begin{itemize}
 \item ラジオボタンのグループは\ATTR{name}(属性名を\texttt{foo})が
        共通なのでセレクタは\Verb+"input[name=\"foo\"]:checked"+
 \item このメソッドを使わないと、\JSKey{getElenentsByName()}で該当の要素を選び、
       どれがチェックしているか調べる必要がある。
\end{itemize}        

 %\ATTR{id}、\ATTR{name}、\ATTR{class}の属性値はセレクタを使わなくても直
 %接選択するメソッドがある。

\end{frame}
\begin{frame}[containsverbatim]
 \frametitle{やってみよう(1)}

 前のビデオのサンプルに対して次のことをコンソールで行いなさい。
 \begin{itemize}
 \item ラジオボタンのどこかをチェックした後で次のことを行う

        \Verb+document.querySelector("input[name=\"R1\"]:checked")+
  \item \ELM{select}を\texttt{<select multiple>}にするとどうなるか
        \begin{itemize}
         \item このとき、「Cntrl+右クリック」をいくつかのところで行う
         \item コンソールで次のように入力した結果の違いを確認する。
               
               \texttt{document.querySelector("option:checked")}\\
               \texttt{document.querySelectorAll("option:checked")}
        \end{itemize}
 \end{itemize}
\end{frame}
\begin{frame}[containsverbatim]
 \frametitle{やってみよう(2)}
 次のメソッドを\JSKey{querySelector()}または\JSKey{querySelectorAll()}で
 書き直してみよう。
\begin{itemize}
  \item \JSKey{getElementsByTagName("foo")}
  \item \JSKey{getElementById("foo")}
  \item \JSKey{getElementsByClassName("foo")}
  \item \JSKey{getElementsByName("foo")}
\end{itemize}
\end{frame}
\section{HTML要素をプログラムで作成する}
\begin{frame}[containsverbatim]
 \frametitle{\ELM{option}を追加(1)}
  \LISTN{makeHTMLElms.html}{1}{17}{\footnotesize}
\end{frame}
\begin{frame}[containsverbatim]
 \frametitle{\ELM{option}を追加(1)--解説}
 \begin{itemize}
  \item 6行目で前回作成した要素を作成する関数群のファイルを読み込む
  \item 10行目で\ELM{select}を得ている。
  \item 11行目と12行目でデバッグ用のイベントリスナーの登録
  \item 13行目から16行目で\ELM{option}を\ELM{select}の子要素として追加
        \begin{itemize}
         \item 14行目で、\ELM{option}を作成し、同時にその属性値を設定
         \item メニューに表示する文字列は\ELM{option}の子要素の
               \Verb+textNode+なので\JSKey{createTextNode}で作成
        \end{itemize}
 \end{itemize}
\end{frame}
\begin{frame}[containsverbatim]
 \frametitle{\ELM{option}を追加(2)}
  \LISTN{makeHTMLElms.html}{18}{last}{\footnotesize}
\end{frame}
\begin{frame}[containsverbatim]
 \frametitle{\ELM{option}を追加(2)--解説}
 \begin{itemize}
  \item 18行目から27行目は前のビデオの例と同じ
  \item 35行目に\ELM{option}が全くない\ELM{select}を作成
 \end{itemize}
\end{frame}
 \section{やってみよう}
 \begin{frame}[containsverbatim]
\frametitle{やってみよう}
% \begin{itemize}
%   \item 色が選択できるプルダウンメニューを作る。\texttt{value}にはSVGで
%         指定できる色名を、表示は日本語の色名にするようにすること。
  %  \item
  連番が設定できるメニューを作成する関数を作成せよ。それを用いて
        3つのメニューを並べて年、月、日が指定できるものを作成せよ。
%  \end{itemize}
 \end{frame}
%
\end{document}
\begin{frame}[containsverbatim]
 \frametitle{}
\end{frame}
