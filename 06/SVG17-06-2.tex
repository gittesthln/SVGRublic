\input beamerHead.tex
\TITLE{6}{2}{CSS入門}{5/31}
\begin{document}
\frame{\maketitle}
\section{CSSとは}
\begin{frame}[containsverbatim]
 \frametitle{CSSについて}
 \begin{itemize}
  \item カスケーディングスタイルシート(CSS)はHTML文書の要素の表示方法を
        指定
  \item CSSはJavaScriptからも制御可能
 \end{itemize}
 \REF{付録47}
\begin{itemize}
 \item 属性\texttt{id}の属性値の前に\texttt{\#}をつけるとその要素を選択
 \item 属性\texttt{class}の属性値の前に\texttt{.}をつけるとその要素を選択
 \item \Verb+E F+ は \Verb+E+ の下の \Verb+F+
 \item \Verb+E > F+ は \Verb+E+ の直下の \Verb+F+
\end{itemize}
 \end{frame}
\begin{frame}[containsverbatim]
 \frametitle{CSSについて}
 {\footnotesize
\begin{tabular}{|m{13em}|m{17zw}|}\hline
\multicolumn{1}{|c|}{セレクタ}&\multicolumn{1}{c|}{解説}\\\hline
\Verb+*+&任意%&Universal selector&2
\\\hline
\Verb+E+&タイプが \Verb+E+ %&Type selector&1
\\\hline
\Verb+E[foo]+&タイプが \Verb+E+ で属性 \Verb+"foo"+ を持つ%&Attribute
	 %selectors
\\\hline
\Verb+E[foo="bar"]+&タイプが \Verb+E+ で属性 \Verb+"foo"+ の属性値が
     \Verb+"bar"+%&Attribute selectors
\\\hline
\Verb+E[foo~="bar"]+&タイプが \Verb+E+ で属性 \Verb+"foo"+ の属性値が
     空白で区切られた一つが \Verb+"bar"+%&Attribute selectors
 \\\hline
\Verb+E:link+, %\newline
\Verb+E:visited+&まだ訪れたことがない(\texttt{:link})か訪れたことがある
     (\texttt{visited})ハイパーリンクのアンカーである要素%&The link pseudo-classes+&1
\\\hline
\Verb+E:active+, %\newline
\Verb+E:hover+, %\newline
\Verb+E:focus+&ユーザーに操作されている状態中%&The user action
	 %pseudo-classes+&1 and 2
\\\hline
\Verb+E:enabled+, %\newline
\Verb+E:disabled+&使用可能(\texttt{:enable})か使用不可のユーザーインター
     フェイス%&The UI element states pseudo-classes
\\\hline
\Verb+E:checked+&チェックされているユーザーインターフェイス%&The UI element states
	 %pseudo-classes
\\\hline
\Verb+E.warning+&属性\texttt{class} が "warning" である要素%&Class selectors&1
\\\hline
\Verb+E#myid+&属性\texttt{id} の属性値が "myid" %&ID selectors&1
\\\hline
\Verb+E F+&要素\texttt{E} の子孫である要素\texttt{F}%&Descendant
	 %combinator&1
\\\hline
\Verb+E > F+&要素\texttt{E} の子である要素\texttt{F}%&Child combinator
\\\hline
\Verb-E + F-&要素\texttt{E} の直後にある要素\texttt{F}%&Adjacent
	 %sibling combinator
\\\hline
\Verb+E ~ F+&要素\texttt{E} の直前にある要素\texttt{F}%&General sibling
	 %combinator
\\\hline
\end{tabular}
 }
\end{frame}
\begin{frame}[containsverbatim]
 \frametitle{CSSセレクタを用いて要素を得る}
 \JSKey{querySelector()}と\JSKey{querySelectorAll()}は引数にCSSセレクタ
 を与えて要素のリストを得ることが可能
 \begin{itemize}
  \item ラジオボタンでチェックしているものを探す

        ラジオボタンのグループは属性\ATTR{name}(属性名を\texttt{foo})が
        共通なのでセレクタは\Verb+"input[name=\"foo\"]:checked"+
 \end{itemize}
 \ATTR{id}、\ATTR{name}、\ATTR{class}の属性値はセレクタを使わなくても直
 接選択するメソッドがある。

 次のメソッドを\JSKey{querySelector()}または\JSKey{querySelectorAll()}で
 書き直してみよう
\begin{itemize}
  \item \JSKey{getElementsByTagName("foo")}
  \item \JSKey{getElementById("foo")}
  \item \JSKey{getElementsByClassName("foo")}
  \item \JSKey{getElementsByName("foo")}
 \end{itemize}
\end{frame}

 \section{やってみよう}
\begin{frame}[containsverbatim]
% \frametitle{}
\end{frame}
\end{document}
\begin{frame}[containsverbatim]
 \frametitle{}
\end{frame}
