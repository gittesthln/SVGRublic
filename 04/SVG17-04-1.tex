\input ../beamerHead.tex
\TITLE{4}{1}{JavaScriptとブラウザの開発者ツールの使い方}{5/16}
\begin{document}
\frame{\maketitle}
%\frame{\tableofcontents}
\section{JavvaScriptについて}
\begin{frame}[containsverbatim]
 \frametitle{JavaScriptの記述方法}
 \begin{itemize}
  \item JavaScriptの国際的規約は\texttt{ECMAScript}と呼ばれる言語
        仕様
	\item 2015年にECMAScript第6版が、2016年6月に第7版が発表されている。
	\item この授業では新しい規格を積極的に採用する。参考図書としては
				オライリーから出ている
				\href{https://www.amazon.co.jp/%E5%88%9D%E3%82%81%E3%81%A6%E3%81%AEJavaScript-%E7%AC%AC3%E7%89%88-ES2015%E4%BB%A5%E9%99%8D%E3%81%AE%E6%9C%80%E6%96%B0%E3%82%A6%E3%82%A7%E3%83%96%E9%96%8B%E7%99%BA-Ethan-Brown/dp/4873117836/ref=sr_1_1?s=books&ie=UTF8&qid=1493275993&sr=1-1&keywords=%E5%88%9D%E3%82%81%E3%81%A6%E3%81%AEjavascript}{
				「初めてのJavaScript第3版」}を薦める。
  \item JavaScriptのプログラムは\ELM{script}内に記述
        \begin{itemize}
         \item \ATTR{type}に\VAL{text/ecmascript}を指定
         \item \ELM{script}内に直接プログラムを記述するためには
               \texttt{CDATA}セクションと呼ばれる範囲に記述
         \item \texttt{CDATA}セクションの始まりは\Verb+<![CDATA[+、終
               了は\Verb+]]>+
         \item この記述はJavaScriptとして文法上正しくないので
               \texttt{//}をつけて、JavaScriptからみてコメントになるよう
               にする。
        \end{itemize}
 \end{itemize}
\end{frame}
\begin{frame}[containsverbatim]
 \frametitle{JavaScriptの特徴}
 \begin{itemize}
  \item JavaScriptでは変数はどのような型のデータも保持できる。
  \item 変数は宣言しなくても利用可能
  \item 宣言する場合は \JSKey{let} キーワードを用いる(通常は宣言をする)
  \item 文の最後には\JSKey{;}を置く
  \item 文の記述方法はC言語とほとんど同じ
  \item 関数は \JSKey{function} キーワードで始める。
  \item そのあとに関数名と\texttt{()}内に仮引数の宣言をする
  \item 関数本体は\texttt{\{\}}内に記述
  \item 値を戻す場合には\JSKey{return} 文の後に戻り値を指定
 \end{itemize}
\end{frame}
\section{開発者ツール}
\begin{frame}[containsverbatim]
 \frametitle{開発者ツール -- ブラウザは統合環境}
 各種ブラウザで表現は異なるが機能はほぼ同じ
 \begin{itemize}
  \item 開くためのショートカットキーは「F12」か「Ctrl+Shift+I」
  \item DOMツリーの表示と修正など(Elements)
  \item インターラクティブな操作やエラー表示、プログラムからの出力
        (Console)
  \item JavaScript などのプログラムソースの表示、ブレイクポイントの設定
        (Source)
  \item ...
 \end{itemize}
 順番に見ていく
\end{frame}
\begin{frame}
 \frametitle{開発者ツール--「Elements」}
 \begin{itemize}
	\item DOMツリーを見る
	\item 要素の属性を変えることができる
 \end{itemize}
\end{frame}
\begin{frame}
 \frametitle{開発者ツール--「Console」}
 \begin{itemize}
	\item プログラムの実行中にメッセージが出せる
	\item その場でJavaScript のプログラムが実行できる
	\item オブジェクトや変数の値の確認ができる
 \end{itemize}
\end{frame}
\begin{frame}
 \frametitle{開発者ツール--「Source」}
 \begin{itemize}
	\item JavaScriptのソースコードにブレイクポイントを設定できる
	\item 1行ごとに実行させることができる
 \end{itemize}
\end{frame}
\begin{frame}[containsverbatim]
 \frametitle{確認しよう}
 自分の使用しているブラウザで次のことを確認する
 \begin{itemize}
  \item 「開発者ツール」に相当する機能の名称
  \item 「開発者ツール」を開くショートカットキー
  \item このビデオ内で開いたSVGについて次のことを行う。
\begin{itemize}
 \item 開発者ツールを開いてDOMツリーのノードを展開し、DOMツリー上
        でマウスカーソルを動かしたときの現象
  \item SVGファイルを開いて要素のプロパティを修正する
  \item コンソールの入力がどこでできるか確認する。
  \item コンソールで\texttt{2+3;}と入力して結果が表示されることを確認
  \item コンソールで次のように入力したときの結果を確かめる(結果の展開が可能な
        はずなので確認すること)
\\
        \texttt{document.getElementsByTagName("polyline");}
\end{itemize}        
 \end{itemize}
 ブラウザの機能、いかがでしたか
 \end{frame}
\end{document}
\begin{frame}[containsverbatim]
 \frametitle{}
\end{frame}
