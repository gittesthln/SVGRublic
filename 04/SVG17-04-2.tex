\input ../beamerHead.tex
\TITLE{4}{2}{DOMの操作とイベント}{5/16}
\begin{document}
\frame{\maketitle}
%\frame{\tableofcontents}
\section{JavvaScriptについて}
\begin{frame}[containsverbatim]
 \frametitle{JavaScriptの記述方法}
 \begin{itemize}
  \item JavaScriptの国際的規約は\texttt{ECMAScript}と呼ばれる言語
        仕様
	\item 2015年にECMAScript第6版が、2016年6月に第7版が発表されている。
	\item この授業では新しい規格を積極的に採用する。参考図書としては
				オライリーから出ている
				\href{https://www.amazon.co.jp/%E5%88%9D%E3%82%81%E3%81%A6%E3%81%AEJavaScript-%E7%AC%AC3%E7%89%88-ES2015%E4%BB%A5%E9%99%8D%E3%81%AE%E6%9C%80%E6%96%B0%E3%82%A6%E3%82%A7%E3%83%96%E9%96%8B%E7%99%BA-Ethan-Brown/dp/4873117836/ref=sr_1_1?s=books&ie=UTF8&qid=1493275993&sr=1-1&keywords=%E5%88%9D%E3%82%81%E3%81%A6%E3%81%AEjavascript}{
				「初めてのJavaScript第3版」}を薦める。
  \item JavaScriptのプログラムは\ELM{script}内に記述
        \begin{itemize}
         \item \ATTR{type}に\VAL{text/ecmascript}を指定
         \item \ELM{script}内に直接プログラムを記述するためには
               \texttt{CDATA}セクションと呼ばれる範囲に記述
         \item \texttt{CDATA}セクションの始まりは\Verb+<![CDATA[+、終
               了は\Verb+]]>+
         \item この記述はJavaScriptとして文法上正しくないので
               \texttt{//}をつけて、JavaScriptからみてコメントになるよう
               にする。
        \end{itemize}
 \end{itemize}
\end{frame}
\begin{frame}[containsverbatim]
 \frametitle{JavaScriptの特徴}
 \begin{itemize}
  \item JavaScriptでは変数はどのような型のデータも保持できる。
  \item 変数は宣言しなくても利用可能
  \item 宣言する場合は \JSKey{let} キーワードを用いる(通常は宣言をする)
  \item 文の最後には\JSKey{;}を置く
  \item 文の記述方法はC言語とほとんど同じ
  \item 関数は \JSKey{function} キーワードで始める。
  \item そのあとに関数名と\texttt{()}内に仮引数の宣言をする
  \item 関数本体は\texttt{\{\}}内に記述
  \item 値を戻す場合には\JSKey{return} 文の後に戻り値を指定
 \end{itemize}
\end{frame}
 \section{イベント}
\begin{frame}[containsverbatim]
 \frametitle{イベントとは}
 \begin{itemize}
  \item プログラムの実行中に内部または外部から伝えられる情報
  \item イベントに対応して処理するプログラムをイベント駆動型と呼ぶ
  \item イベント駆動型ではイベントの発生順序が前もって決められないのでそ
        れぞれの処理は独立する必要がある。
 \end{itemize}
\end{frame}
\newcommand{\Event}[1]{\texttt{#1}}
\begin{frame}[containsverbatim]
 \frametitle{イベントの例\REF{134}}
 この演習でよく使うイベントはマウスに関するものとファイルのロードが完了
 したときのイベント
\begin{center}
\begin{tabular}[t]{|c|c|}
 \hline
イベントの発生条件& イベントの属性名%&
%\multicolumn{1}{c|}{対応するSVGでのイベント}
\\\hline
ファイルのロード終了時  &\Event{onload} \\ \hline
ボタンがクリックされた &\Event{onclick}  \\ \hline
ボタンが押された &\Event{onmousedown}  \\ \hline
マウスカーソルが移動した&\Event{onmousemove}  \\ \hline
マウスボタンが離された&  \Event{onmouseup} \\ \hline
マウスカーソルが範囲に入った&\Event{onmouseover}  \\ \hline
マウスカーソルが範囲から出た&\Event{onmouseout}  \\ \hline
\end{tabular} 
\end{center}
\end{frame}
\section{イベント処理}
\begin{frame}[containsverbatim]
 \frametitle{クリックした円の塗りつぶしの色を表示}
 円の上をクリックすると円の色を表示する
\end{frame}
\begin{frame}[containsverbatim]
 \frametitle{デモの解説}
 \begin{itemize}
  \item 円の上をクリックするとクリックした円の塗りつぶしの色がメッセージ
        ボックスに表示
  \item メッセージボックスを消さないと次の操作ができない(モーダルな
        window)
 \end{itemize}
\end{frame}
\begin{frame}[containsverbatim]
 \frametitle{円の上をクリックすると円の色を表示する\\ソースコード\REF{136}}
 \framesubtitle{要素にイベント処理関数を定義}
 \LISTAll{7}{svg-js-click.svg}
\end{frame}
\begin{frame}[containsverbatim]
 \frametitle{円の上をクリックすると円の色を表示する\\ソースコード\REF{136}解説}
 \begin{itemize}
  \item 13行目から15行目で定義されている\ELM{circle}の\ATTR{onclick}に対
        して関数\VAL{click(evt)}が呼び出されるようにしている。
  \item 変数\texttt{evt}は発生したイベントオブジェクト
  \item 呼び出される関数は8行目から10行目で定義
  \item マウスによるイベントはマウスイベントと呼ばれる\REF{145}。
  \item イベントの\ATTR{target}はイベントが発生したオブジェクト
  \item イベントが発生したオブジェクトはSVGの要素で、その属性の値は
        \VAL{getAttribute(属性名)}で求められる
  \item ここでは\ATTR{fill}の値を求めている
  \item JavaScriptでは\JSKey{+}演算子はどちらかの被演算子が文字列のとき
        は文字列の連接(二つの文字列をつなげる)
  \item 関数\JSKey{alert}は与えられた引数をメッセージボックスに表示
 \end{itemize}
\end{frame}
\begin{frame}[containsverbatim]
 \frametitle{やってみよう}
\begin{itemize}
  \item 円の別の属性値を表示
  \item \JSKey{alert}内の初めの文字列("Circle")を
        \JSKey{event.target.tagName}に変える
  \item 図形を変えていくつかの属性値を表示
  \item 異なる図形をいくつか置いて、同様のことをする
\end{itemize}
 \end{frame}
\section{イベント処理関数の登録}
\begin{frame}[containsverbatim]
 \frametitle{円の上をクリックすると円の色を表示する\\ソースコード\REF{144}}
 \framesubtitle{要素にイベント処理関数を定義}
\centering 前のビデオにおけるリスト(比較のために再掲)
 \LISTAll{7}{svg-js-click.svg}
\end{frame}
\begin{frame}[containsverbatim]
 \frametitle{円の上をクリックすると円の色を表示する\\ソースコード\REF{147}}
  \framesubtitle{開始時にイベント処理関数を定義(1)}
 \LISTN{7}{svg-js-click-rev.svg}{1}{17}
\end{frame}
\begin{frame}[containsverbatim]
 \frametitle{円の上をクリックすると円の色を表示する\\ソースコード\REF{147}}
  \framesubtitle{開始時にイベント処理関数を定義(2)}
 \LISTINGFP{7}{svg-js-click-rev.svg}{18}{last}
\end{frame}
\begin{frame}[containsverbatim]
 \frametitle{円の上をクリックすると円の色を表示する\\ソースコード\REF{147}解説}
  \framesubtitle{開始時にイベント処理関数の解説(1)}
 \begin{itemize}
  \item 18行目から20行目にある\ELM{circle}には\ATTR{onclick}がない
  \item 8行目から13行目に\JSKey{window.onload}に関数を代入
  \item JavaScriptでは関数もオブジェクトなので変数に代入が可能
  \item \JSKey{load}イベントはファイルの内容がすべて呼び出されたのちに発生
  \item これにより\JSKey{onload}イベントが発生したときにここで定義した関
        数が呼び出される
 \end{itemize}
\end{frame}
\section{イベント処理関数の登録する要素}
\begin{frame}[containsverbatim]
 \frametitle{円の上をクリックすると円の色を表示する\\ソースコード
 \REF{147}解説}
  \framesubtitle{開始時にイベント処理関数の解説(2)}
\begin{itemize}
 \item \JSKey{document}オブジェクトはSVG文書を指す。
 \item \JSKey{getElementsByTagName}は指定されたオブジェクト(ここでは
       \JSKey{document})の子要素のうち引数で与えられた要素名を持つオブジェ
       クトをすべて求めてリスト(配列のようなもの)として返す
 \item このリストの長さは\JSKey{length}プロパティで得られる
 \item 10行目から12行目でこのリストのそれぞれにイベント処理関数
       (イベントリスナー)を登録
 \item イベント処理関数の登録は\JSKey{addEventListener}メソッドを利用
 \item 引数は、イベント名(文字列)、イベント処理関数(関数名)、イベント処理するタイミン
       グの3つ
 \item イベント処理関数は14行目から16行目にある\texttt{click})
 \item 最後の引数は論理値。詳しくは後日説明
\end{itemize}
\end{frame}
\begin{frame}[containsverbatim]
 \frametitle{属性値を変える}
 \framesubtitle{クリックした円の位置を変更(1)\REF{149}}
 \LISTINGFP{7}{svg-js-click2.svg}{1}{18}
\end{frame}
\begin{frame}[containsverbatim]
 \frametitle{属性値を変える}
 \framesubtitle{クリックした円の位置を変更(2)\REF{149}}
 \LISTINGFP{7}{svg-js-click2.svg}{19}{last}
\end{frame}
\begin{frame}[containsverbatim]
 \frametitle{ソースコードの解説--属性値を変える}
 \begin{itemize}
  \item 以前のコードと違う点はイベント処理関数の内容だけ
  \item 15行目でクリックされた円の中心の$y$座標(\ATTR{cy})を
        \JSKey{getAttribute}メソッドで得ている
  \item 得られた$y$座標の値は文字列なので文字列を数に直す
        \JSKey{parseInt}を用いて数に変換し、その値を$150$から引いている
  \item \ATTR{cy}の初期値はすべての円で$50$なので1回目にクリックされると
        その値が$150-50=100$となる。
  \item 次にクリックされると$150-100=50$となり、初期値に戻る。
  \item この値を\JSKey{setAttribute}でクリックされた円の\ATTR{cy}に設定
        している(16行目)
 \end{itemize}
\end{frame}
 \begin{frame}[containsverbatim]
  \frametitle{やってみよう}
  \begin{itemize}
   \item 円に代わりに正方形をいくつか置いてそのうえでクリックしたときに移動する
   \item クリックした要素の色を変える
   \item 異なる種類の要素を置いてクリックしたときに移動する
         (\JSKey{tagName}プロパティを使う)
  \end{itemize}
 
 \end{frame}
\begin{frame}[containsverbatim]
 \frametitle{クリックした位置に円を移動}
 \framesubtitle{自分以外の要素の属性値を変える}
 \LISTING{7}{svg-js-click3.svg}{0.7}
\end{frame}
\begin{frame}[containsverbatim]
 \frametitle{ソースコードの解説}
 \begin{itemize}
  \item 17行目に画面全体を覆う長方形を定義
  \item この長方形の\ATTR{id}に\VAL{Canvas}が設定
  \item 9行目でこの長方形にイベント処理関数を定義
  \item 18行目から19行目に\ATTR{id}が\VAL{Circle}である円を定義
  \item イベント処理関数の処理手順はつぎのとおり
        \begin{itemize}
         \item 12行目で円の要素を得ている
         \item その要素の\ATTR{cx}にイベントが発生したときのマウスカーソ
               ル位置(\JSKey{E.clientX})を設定
         \item その要素の\ATTR{cy}に対しても同様
        \end{itemize}
  \item このコードでは円に対してクリックイベントの処理関数が定義されてい
        ないので円の上でクリックしても円は移動しない
 \end{itemize}
\end{frame}
\section{やってみよう}
\begin{frame}[containsverbatim]
 \frametitle{やってみよう}
 \begin{itemize}
  \item 円に対してもイベント処理関数を登録してその上をクリックしたときに移動するようにす
        る
  \item 円の上をクリックしたら色が変化する
  \item 2つ以上の要素を置く。要素以外のところでクリックしたら最後にクリッ
        クした要素がクリックした位置に移動する
 \end{itemize}
% 次は属性値を設定する方法を解説
\end{frame}
\end{document}
\begin{frame}[containsverbatim]
 \frametitle{}
\end{frame}
