\documentclass[a4j]{jreport}
\input ../rubricHead.tex
\input ../rubricPresentation.tex
\input ../rubricUnitIHead.tex
\begin{document}
\setcounter{chapter}{4}
\chapter{}
\changePage{5/23}
今回の演習の目的は次の通りである。
\begin{itemize}
 \item
			\end{itemize}
課題に\Must と書かれたものを最低行うこと。それ以外の課題は
いくつか選択してよい。
\Probs{0}{演習のビデオ1を見て次の問いに答えよ。}{
 {0}{0}
 }
\Probs{}{演習のビデオ2を見て次の問いに答えよ。}{
 {0}{0}
 }
\Probs{}{演習のビデオ3を見て次の問いに答えよ。}{
 {0}{0}
 }
%\newpage
\Rubric{第回(/)}{ノートの内容}{
項目の最後の文字は次に示す項目の評価である。
{\bfseries リ}(プログラム等のリスト)、{\bfseries 説}(プログラ
ム説明が手書きまたは印刷である)、{\bfseries 図}(結果のキャプチャ画面)、
{\bfseries 考}( 考察が手書きまたは印刷である)を意味し、次の記号で評価を
示す。
$\times$(不備またはない)、$\triangle$(もう一息)、$\bigcirc$(良い)、
$\circledcirc$(大変良い)
}
{{課題1}{20}
{
  {}
}
{
  {}
}
{
  {}
}
 {課題2}{30}
 {
   {}
 }
 {
   {}
 }
 {
   {}
 }
 {課題3}{25}
 {
   {}
 }
 {
   {}
 }
 {
   {}
 }
}

 \rublicPresenII{第回(/)}

\end{document}