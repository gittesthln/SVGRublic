\documentclass[a4j]{jreport}
\input ../rubricHead.tex
\input ../rubricPresentation.tex
\input ../rubricUnitIHead.tex
\begin{document}
\setcounter{chapter}{4}
\chapter{テキストの挿入と新しい要素の追加}
\changePage{5/23}
今回の演習の目的は次の通りである。
\begin{itemize}
 \item SVG要素内にテキストを表示する
 \item SVG要素を追加してDOMツリーの変化を確認する
 \item \texttt{setTimeout}関数を用いてアニメーションを作成する。
 \item JavaSCriptの配列とオブジェクトの使い方
 \item 要素を追加したりする簡単な自前のライブラリーの作成と使い方
\end{itemize}
課題に\Must と書かれたものを最低行うこと。それ以外の課題は
いくつか選択してよい。
\Probs{SVGにテキストの追加}{演習のビデオ1を見て次の問いに答えよ。}{
{\Must 「クリックした位置をSVG内に表示」を実行する。位置を表示する
\texttt{text}要素に空白を入れておかないとエラーが発生することを確認す
る。}{0}
{\Must クリックイベントが発生している要素は何か答えよ。\\
要素名:\\理由:}{0.03}
{28行目にある\texttt{rect}を23行目にある\texttt{circle}の前に置くと起こ
 る不備な点を指摘し、理由を述べよ。
 \begin{itemize}
  \item \rule{0em}{0.5cm}
  \item \rule{0em}{0.7cm}
  \item \rule{0em}{0.7cm}
 \end{itemize}
 理由:}{0.02}
{SVGリスト7.6でエラーが起きないことを確認する。}{0}
{\Must 「直線を引く」を実行し、ソースコードとDOMツリーに変化があるかを確
認する。}{0}
{配布資料155ページ図7.14にあるように、塗りの色が異なる小さな正方形をいく
つか置き、それらのうち最後にクリックした色で直線が引けるようにする。}{0}
{\Must 長方形をドラッグして描くものを作成する。}{0}
}
\newpage
\Probs{初期化で要素を作成と自前のアニメーション}{演習のビデオ2を見て次の
問いに答えよ。イベント処理関数の設定は\texttt{window.onload}内で行うこと。
}
{
{\Must 初期化時にサイクロイドの図形を描く。}{0}
{\Must 正多角形を初期化時に頂点の位置を計算して描く。描いた結果をDOMツ
リーで確認する。}{0}
{\Must サイクロイドをアニメーションで描く。}{0}
{\Must 1年次のProcessing の演習で行った課題をJavaScriptのアニメーションに移植
する。}{0}
}
\Probs{JavaScriptの配列とオブジェクトリテラル}{演習のビデオ3をの前半部
「JavaScriptの配列とオブジェクトリテラル」を見て次の問いに答えよ。}
{
{\Must JavaScriptの配列の操作についてコンソールで確認をしている。確認の
ための題材はビデオ内のものとは違うものにすること。}{0}
{\Must JavaScriptのオブジェクトリテラルの操作についてコンソールで確認をしている。確認の
ための題材はビデオ内のものとは違うものにすること。}{0}
}
\Probs{ドラッグ処理}{演習のビデオ3の後半部「要素を作成する関数群処理」を見
て次の問いに答えよ。}
{
{\Must 「ドラッグして直線を引く」を関数群を使って書き直しする。以前のも
のとのコードの比較し、考察をつけること。}{0}
{\Must 「乱数を使用して円をいくつか描く」に色を追加して正しく動かしなさ
い。また、配列に色の追加をするだけでプログラムが正しく実行できる理由をこ
こに記述しなさい。}{0.05}
{(形の異なる)3角形をランダムに表示するものを作成する。図形の種類が増えて
いればなおよい。}{0}
}
\Rubric{第4回(5/16)}{ノートの内容}{
今回からJavaScriptによるプログラミングが始まる。細かい文法の説明を特には
しないので今までのプログラミング言語と比較して違いに気を付けること。
\newline
項目の最後の文字は次に示す項目の評価である。
{\bfseries リ}(プログラム等のリスト)、{\bfseries 説}(プログラ
ム説明が手書きまたは印刷である)、{\bfseries 図}(結果のキャプチャ画面)、
{\bfseries 考}( 考察が手書きまたは印刷である)を意味し、次の記号で評価を
示す。
$\times$(不備またはない)、$\triangle$(もう一息)、$\bigcirc$(良い)、
$\circledcirc$(大変良い)
}
{{課題1}{20}
{
  {使用中のブラウザと「開発者ツール」の開き方\ResultFI}
  {SVGファイルに対して要素の属性を
  直接変えた結果に前後の図に開発者ツールの「Elements」タブが表示\ResultA}
  {開発者ツールのコンソールで直接、簡単な算術式や
  \texttt{document}\newline\texttt{.getElementsByTagName}を実行\ResultA}
}
{
  {使用中のブラウザと「開発者ツール」の開き方の図または考察がない\ResultFI}
  {SVGファイルに対して要素の属性を
  直接変えた結果に前後の図に開発者ツールの「Elements」タブがない\ResultA}
  {開発者ツールのコンソールで簡単な算術式や
  \texttt{document}\newline\texttt{.getElementsByTagName}を実行が一部な
  い\ResultA}
}
{
  {使用中のブラウザと「開発者ツール」の開き方の図と考察がないか不十分\ResultFI}
  {SVGファイルに対して要素の属性を
  直接変えた結果に前後の図のいずれかがないか開発者ツールの「Elements」タ
  ブがなく不十分\ResultA}
  {開発者ツールのコンソールで簡単な算術式や
  \texttt{document}\newline\texttt{.getElementsByTagName}を実行がな
  い\ResultA}
}
 {課題2}{30}
 {
 {\texttt{window.onload}内でイベント処理関数を
 登録し、クリック時に円の塗りつぶし以外の属性を表示している。\ResultA}
 {\Must 図形の種類を変えている。クリック時に複数の属性値をテンプレー
 トリテラルを用いて表示している。\ResultA}
 {\Must\texttt{console.log}を用いた実行結果があ
 り、\texttt{alert}を使用した場合との違いがある。 \ResultFI}
 {\Must 正方形をいくつか置いてクリックすると移動する。\ResultA}
 {正方形をいくつか置いてクリックすると色が変化する。\ResultA}
 {異なる種類の要素があり、それらの図形のクリックする前後の位置が変化を示す
 前後の図が十分ある。\ResultA}
 }
 {
 {\texttt{window.onload}内でイベント処理関数を
 登録していない。クリック時に円の塗りつぶし以外の属性を表示している。\ResultA}
 {\Must 図形の種類を変えて、クリック時の複数に属性値をテンプレー
 トリテラルを一部しか用いないで表示している。\ResultA}
 {\Must\texttt{console.log}を用いた実行結果があ
 り、\texttt{alert}を使用した場合との違いの考察が不十分 \ResultFI}
 {\Must 単独の正方形を置いてクリックすると移動する。\ResultA}
 {単独の正方形しか置いていないが、クリック前後の色の変化の図があ
 り、色が変化している。\ResultA}
 {異なる種類の要素に対してクリックする前後の位置が変化を示す
 図前後の図が少し足りない。\ResultA}
 }
 {
 {\texttt{window.onload}内でイベント処理関数を
 登録していない。クリック時に円の塗りつぶし以外の属性を表示していない。\ResultA}
 {\Must 図形の種類を変えていない。クリック時に複数の属性値をテンプレー
 トリテラルを用いて表示していない。\ResultA}
 {\Must\texttt{console.log}を用いた実行結果がな
 い。\texttt{alert}を使用した場合との違いの考察がない。 \ResultFI}
 {\Must 単独の正方形しかない。クリック時に移動動作がおかしい。\ResultA}
 {単独の正方形しか置いていなく、クリック前後の色の変化の図がないか不十分\ResultA}
 {異なる種類の要素がない。クリックする前後の位置が変化を示す
 図が少なすぎるかなく、図形も移動していない。\ResultA}
 }
 {課題3}{25}
 {
 {\Must 画面全体を覆う長方形の属性\texttt{fill}を\texttt{none}にした報告
 があり、考察が正しい。\ResultEI}
 {\Must 円にイベント処理関数を登録して円の上をクリックしたときも移動す
 る図が十分にある。\ResultA}
 {円上をクリックしたときに色が変わる。説明、考察も十分である。\ResultA}
 {要素上以外でクリックしたら最後にクリッ
  クした要素がクリックした位置に移動することを示す図がある。途中の
  動作を示すために\texttt{console.log}を用いている。\ResultA}
 {\Must 円の上をクリックすると「円の色を表示(3)」の動作確認が分かる図が
 十分にある。\ResultA}
 {\texttt{svg}要素にイベント処理を付けたときの動作の確認が十分にある。\ResultA}
 {画面全体を覆う長方形なしで、円上も含めてクリックした位置を円の中心に
 移動する。\ResultA}
 }
 {
 {\Must 画面全体を覆う長方形の属性\texttt{fill}を\texttt{none}にした報告
 があるが、考察が一部正しくない。\ResultEI}
 {\Must 円にイベント処理関数を登録して円の上をクリックしたときも移動す
 る図が少し足りない。\ResultA}
 {要素上以外でクリックしたら最後にクリッ
  クした要素がクリックした位置に移動することを示す図が足りない。途中の
  動作を確認する手段が足りない。\ResultA}
 {\Must 円の上をクリックすると「円の色を表示(3)」の動作確認が分かる図が
 少し足りない。\ResultA}
 {\texttt{svg}要素にイベント処理を付けたときの動作の確認が少し足りない。\ResultA}
 {画面全体を覆う長方形なしで、円上も含めてクリックした位置を円の中心に移
 動することが不十分である。\ResultA}
 }
 {
 {\Must 画面全体を覆う長方形の属性\texttt{fill}を\texttt{none}にした報告
 がないか、考察が正しくない。\ResultEI}
 {\Must 円にイベント処理関数を登録して円の上をクリックしたときも移動す
 る図がないか、非常に足りない。\ResultA}
 {要素上以外でクリックしたら最後にクリッ
  クした要素がクリックした位置に移動することを示す図がないか非常に足りない。途中の
  動作を確認する手段がない。\ResultA}
 {\Must 円の上をクリックすると「円の色を表示(3)」の動作確認が分かる図が
 ないか非常に足りない。\ResultA}
 {\texttt{svg}要素にイベント処理を付けたときの動作の確認が足りない。\ResultA}
 {画面全体を覆う長方形なしで、円上も含めてクリックした位置を円の中心に移
 動しない。\ResultA}
 }
 {課題4}{25}
 {
 {\Must ビデオ内の「マウスのドラッグを処理」の動作の気になる点の指摘が正
 しい。\ResultFI}
{\Must ビデオ内の「マウスのドラッグを処理(改良版)」で前問の気に
なる点が修正の確認とDOMツリーの変化の確認が十分になされている。\ResultA}
{\Must 正方形のドラッグの動作が十分である。}
{通常のアイコンのドラッグとの相違点と改善が十分にある。\ResultA}
{図形の種類が増えている。イベント処理関数の処理が図形の種類が増加しても
手を付けないようになっている。\ResultA}
 }
 {
 {\Must ビデオ内の「マウスのドラッグを処理」の動作の気になる点の指摘が少
 し足りない。\ResultFI}
{\Must ビデオ内の「マウスのドラッグを処理(改良版)」で前問の気に
なる点が修正の確認とDOMツリーの変化の確認が十分にできていない。\ResultA}
{\Must 正方形のドラッグの動作がすこしおかしい。}
{通常のアイコンのドラッグとの相違点と改善が不十分である。\ResultA}
{図形の種類が増えていない。イベント処理関数の処理が図形の種類が増加しても
手を付ける必要がある。\ResultA}
 }
 {
 {\Must ビデオ内の「マウスのドラッグを処理」の動作の気になる点の指摘が足りない。\ResultFI}
{\Must ビデオ内の「マウスのドラッグを処理(改良版)」で前問の気に
なる点が修正の確認とDOMツリーの変化の確認ができていない。\ResultA}
{\Must 正方形のドラッグの動作がすこしおかしい。}
{通常のアイコンのドラッグとの相違点と改善が不十分である。\ResultA}
{図形の種類が増えていない。イベント処理関数の処理が図形の種類が増加する
と改良のために手間をかける必要がある。\ResultA}
}
}
\rublicPresenII{第5回(5/23)}

\end{document}