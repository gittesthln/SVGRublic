\input ../beamerHead.tex
\TITLE{6}{3}{SVGとHTMLの間でデータを交換する}{5/30}
\begin{document}
\frame{\maketitle}
\section{SVGとHTMLの間でデータ交換}
\begin{frame}[containsverbatim]
 \frametitle{ソースコード(1)}
 \LISTN{ShowSetClickPos4.html}{1}{12}{\scriptsize}
 \begin{itemize}
	\item 6行目からJavaScriptの部分が開始
	\item 10行目から11行目で色の選択をするプルダウンメニューに現れる色の情
				報が格納されたオブジェクトを定義
 \end{itemize}
\end{frame}
\begin{frame}[containsverbatim]
 \frametitle{ソースコード(2)--プルダウンメニューの作成}
 \LISTN{ShowSetClickPos4.html}{13}{20}{\scriptsize}
 87行目にある\ELM{select}に現れる\ELM{option}を作成
 \begin{itemize}
	\item 13行目で\ELM{select}を得ている
	\item それぞれの色に対して15行目で\ELM{option}を作成
	\item 16行目で\ATTR{value}を配列のキーに設定
	\item 17行目でテキストノードを作成
	\item 18行目でそれを\ELM{option}の子要素にする
	\item 19行目で\ELM{option}を\ELM{select}の子要素にする
 \end{itemize}
\end{frame}
\begin{frame}[containsverbatim]
 \frametitle{ソースコード(3)--初期化}
 \LISTN{ShowSetClickPos4.html}{21}{35}{\scriptsize}
\end{frame}
\begin{frame}[containsverbatim]
 \frametitle{ソースコード(3)--初期化(解説)}
 \begin{itemize}
	\item 21,22行目で円の中心の座標が入るHTML文書内のテキストボックスの要
				素を得ている
	\item 23行目では円の要素を得ている
	\item 24,25行目では円の中心の座標が入るSVG内のテキスト表示位置の要
				素を得ている
	\item 26,27行目ではSVG内の円の中心位置の座標をHTML内のテキストボック
				スに設定
	\item 31行目ではSVG文書の、HTML文書内での上からと左からの位置を示す
				\texttt{BoundingClientRect}オブジェクトを得ている
	\item これはクリックした位置がHTML文書内からの位置となり、イベントオブ
				ジェクトからの座標をそのまま使うと円の位置がずれるため
	\item SVGは縁取りが\texttt{5px}あるようにつくっであるので32,33行目で
				その分も補正値に加えている
	\item 34行目では画面の情報などをアップデートするための関数
				\texttt{refreah()}を呼んでいる。
 \end{itemize}
\end{frame}
\begin{frame}[containsverbatim]
 \frametitle{ソースコード(4)--イベント処理関数}
 \LISTN{ShowSetClickPos4.html}{36}{57}{\scriptsize}
\end{frame}
\begin{frame}[containsverbatim]
 \frametitle{ソースコード(4)--イベント処理関数(解説)}
\begin{itemize}
	\item 36行目から40行目では画面がクリックされたときの処理関数が定義

				イベントが起きた座標から補正値を引いた値をHTML内に設定し(37行目
				と38行目)その後画面の書き直しを実行
	\item 41行目から45行目で画面の情報をアップデートする関数を定義

				SVG内の座標表示と円の\ATTR{fill}の属性値を変更
	\item 47行目から55行目はSVG内の座標位置の表示関数(以前と同じ)
 \end{itemize}
\end{frame}
 \begin{frame}[containsverbatim]
  \frametitle{外部CSSファイルの読み込み}
 \LISTN{ShowSetClickPos4.html}{58}{60}{\scriptsize}
  \begin{itemize}
   \item 58行目で外部のCSSファイルを読み込む
   \item 読み込むために\ELM{link}を用いる
   \item 59行目は\ELM{title}を記述
  \end{itemize}
 \end{frame}
 \begin{frame}[containsverbatim]
  \frametitle{スタイルシート(その1)}
  \LISTN{HTML.css}{1}{7}{\scriptsize}
  \begin{itemize}
   \item \texttt{.display}はクラス名が\texttt{display}の要素に適用
   \item ここでは表題の部分でフォントの大きさを指定
   \item \texttt{.textStyle}はSVG内の座標の表示の部分に適用
   \item フォントの大きさと文字の位置(右寄せ)を指定
  \end{itemize}
 \end{frame}
\begin{frame}[containsverbatim]
  \frametitle{スタイルシート(その2)}
 \LISTN{HTML.css}{8}{20}{\scriptsize}
 \begin{itemize}
  \item 全体の要素を表の形で表示するためのCSSが定義
  \item \texttt{margin-left}は左側に余白を設定
  \item \texttt{table-row}は表の行を指定
  \item \texttt{table-cell}は表の各項目を指定
 \end{itemize}
\end{frame}
\begin{frame}[containsverbatim]
  \frametitle{スタイルシート(その3)}
 \LISTN{HTML.css}{21}{last}{\scriptsize}
 \begin{itemize}
  \item \texttt{\#XP}は\ATTR{id}が\texttt{XP}の要素に対して適用
  \item ここでは数が入るので右寄せを指定
  \item \texttt{,}で並べるとそれぞれのセレクターに対して適用される
  \item 残りも同じ
 \end{itemize}
\end{frame}
\begin{frame}[containsverbatim]
 \frametitle{ソースコード(4)--SVGの部分}
 \LISTN{ShowSetClickPos4.html}{61}{80}{\scriptsize}
\end{frame}
\begin{frame}[containsverbatim]
 \frametitle{ソースコード(4)--SVGの部分(解説)}
 \begin{itemize}
  \item SVGの要素はHTML内の表の一部として現れる
  \item SVGの大きさが設定されているのが甘えと異なる
  \item クリック範囲をわかりやすくするため、外枠を設定(76行目から77行目)
 \end{itemize}
\end{frame}
\begin{frame}[containsverbatim]
 \frametitle{ソースコード(5)--テキストボックス等の部分}
 \LISTN{ShowSetClickPos4.html}{81}{last}{\scriptsize}
 \begin{itemize}
  \item 円の中心を指定するテキストボックス(82目から85行目)
  \item 色を選択するためのプルダウンメニュー(87行目)
  \item 設定を実行するためのボタン(88行目)
 \end{itemize}
\end{frame}
 \section{やってみよう}
\begin{frame}[containsverbatim]
  \frametitle{やってみよう}
 \begin{itemize}
  \item HTML内の表題の部分のフォントの大きさ(\texttt{.display}の
        \texttt{font-size})を変えてもクリックの位置と値が一致していること
        を確かめる
  \item 画面を表示した後、ブラウザの横幅を変えて表題部分の行数が増えると
        クリックした位置と円の移動位置が一致しないことを確かめる
  \item 上記の不具合を直す
  \item 色の選択の種類を増やす
  \item SVG内に図形を追加し、HTML内に図形を選択するプルダウンメニューま
        たはラジオボタンを置き、クリックしたとき選択された図形だけが移動
        するようにする
 \end{itemize}
\end{frame}
\end{document}
\begin{frame}[containsverbatim]
 \frametitle{}
\end{frame}
