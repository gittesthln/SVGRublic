\documentclass[a4j]{jreport}
\input rubricHead.tex
\input rubricPresentation.tex
\input rubricUnitIHead.tex
\begin{document}
\chapter{SVGの基礎}
\changePage{4/18}
演習の課題については、この中に直接記入してもよい。実行結果が伴う場合には、
印刷したものを添えること。
\Probs{演習の準備}{予習ビデオ1を見て次の問に答えよ。}
 {
 {使用するテキストエディタ名とそのエディタでBOMなしのUTF-8の文字コードで保存
  する方法について報告する。}{0.1}
 {テキストファイルをBOMありとなしで保存したときのファイルサイズ
  の違いがあるか調べて報告しなさい。}{0.1}
  {BOMとは何か。また、なぜ必要なのかを調査しなさい。}{0.2}
%  
  {SVGファイルがBOMなしで保存しなければいけない理由について答えなさい}{0.2}
 }
\Probs{簡単なSVGファイルの作成}{予習ビデオ2を見て次の問いに答えよ。}
 {
 {SVGファイルを表示したブラウザについて報告する。}{0.05}
 {直線の属性をまとめる。1年で学んだProcessingの仕様と比較すること(別紙で
 解答すること)}{0}
 {Fickの錯視を$90^{\circ}$回転させた図形でも同様の錯視が起こることを確認
 する。}{0.0}
 {直線をいくつか組み合わせて図形を作成する。色や幅を変えること。直線の代
 わりに円や長方形を用いてもよい。}{0}
 }
 \Probs{グラデーション}{予習ビデオ3を見て次の問いに答えよ}
 {
 {線形グラデーションで長方形や円を塗りつぶすこと。\texttt{gradientUnits}
 も変えて違いを説明すること。配布資料29ページのリスト2.11を用いてもよ
 い。}{0}
 {放射グラデーションで長方形や円を塗りつぶす。属性\texttt{fx}や\texttt{fy}も
 変えたものをいくつか作成すること}{0}
 }
\Rubric{第1回(4/18)}{ノートの内容}
{この回の予習の目的は次のとおりである。
\begin{itemize}
 \item 簡単なSVGファイルをテキストエディタで作成し、ブラウザで表示できる
 \item 直線、長方形、円をSVGの要素で表示できる
 \item グラデーションの基本を理解する
\end{itemize}
これらの点をまとめて、作成したSVGによる画像とリストとその解説を付ける。
演習内での議論は手書きでよい。}
{{課題1}{20}{
	{使用したテキストエディタの名称、バージョンがある。}
	{ファイルの保存形式の注意が書いてある。}
  {BOMが存在する理由が正しく書かれている。}
  {BOMつきとなしの場合でのファイルの大きさの差とその説明がある。}
  {SVGファイルがBOMなしでない理由が正しく書かれている。}
	}
	{
	{使用したテキストエディタの名称がある。}
	{ファイルの保存形式の説明に図がない。}
  {BOMが存在する理由が一部不正確である。}
  {BOMつきとなしの場合でのファイルの大きさの報告がある。}
  {SVGファイルがBOMなしでない理由が一部不正確である。}
	}
	{
	{使用したテキストエディタの名称がない。}
	{ファイルの保存形式に言及がない。}
  {BOMが存在する理由が不正確である。}
  {BOMつきとなしの場合でのファイルの大きさの差の説明がない。}
  {SVGファイルがBOMなしでない理由がないか他間違っている。}
	}
	{課題2}{30}
	{
	{使用したブラウザの名称、バージョンがある。}
  {直線の属性のまとめが十分にあり、Processingとの比較も十分である。}
  {Fickの錯視で回転させたものや線の幅を変えたものがあり、それらに考察が
  ある。}
  {直線や円、長方形を組み合わせた充分な量の図形を作成し、それに関する考
  察がある。}
	}
	{
	{使用したブラウザの名称がある。}
  {直線の属性のまとめが十分にあるが、Processingとの比較がないか不十分である。}
  {Fickの錯視で回転させたものや線の幅を変えたものがないか、それらの考察が
  不十分である。}
  {直線や円、長方形を組み合わせた充分な量の図形を作成し、それに関する考
  察が不十分である。}
	}
	{
	{使用したブラウザの名称がない。}
  {直線の属性のまとめが十分にないか全くない。}
  {Processingとの比較がない。}
  {Fickの錯視で回転させたものや線の幅を変えたものがない。}
  {直線や円、長方形を組み合わせた充分な量の図形ない。}
	}
	{課題3}{30}
	{
  {線形グラデーションでいろいろな図形を塗りつぶした例が十分にあり、それ
  らの考察も適切である。}
  {線形グラデーションの\texttt{gradientUnits}の違いについて、例を示して、
  正しい考察をしている。}
  {放射グラデーションでいろいろな図形を塗りつぶした例が十分にあり、それ
  らの考察も適切である。}
	}
	{
  {線形グラデーションでいろいろな図形を塗りつぶした例があり、それ
  らの考察もある。}
  {線形グラデーションの\texttt{gradientUnits}の違いについて、例を示しい
  ないが正しい考察をしている。}
  {放射グラデーションでいろいろな図形を塗りつぶした例があり、それ
  らの考察もある。}
	}
	{
  {線形グラデーションでいろいろな図形を塗りつぶした例がないか、それ
  らの考察がないか不十分である。}
  {線形グラデーションの\texttt{gradientUnits}の違いについて、例がなく
  考察がないか、不十分である。}
  {放射グラデーションでいろいろな図形を塗りつぶした例がないか、それ
  らの考察もないか不十分である。}
	}
  {ノートの使い方}{20}
  {
  {プログラムのリスト、実行結果がすべてあり、考察も適切である。}
  {画面のキャプチャが図形を表示するのに十分な程度の大きさになっており、
  ブラウザ全体になっている。}
  {予習の内容が各項目できれいにまとめられている。}
  {ノートの余白が十分にあり、後から記述を追加できるような配慮がある。}
  {グループ内での議論がまとめられていて、経過と解決法が明確になっている。}
  }
  {
  {プログラムのリスト、実行結果が一部ないか、考察が一部不十分である。}
  {画面のキャプチャが図形を表示するよりも大きすぎるか、
  ブラウザ全体になっていない。}
  {予習の内容の項目が一部ない。}
  {ノートの余白があまりない。}
  {グループ内での議論がまとめられているが、経過と解決法が明確になってい
  ない。}
  }
  {
  {プログラムのリスト、実行結果が少なすぎる。}
  {考察の量が不十分である。}
  {画面のキャプチャがデスクトップ全体になっているか、
  ブラウザで実行したかがわからない。}
  {予習の要点のまとめが少なすぎる。}
  {ノートに余白が全くないか、ほとんどない。}
  {グループ内での議論がまとめられていなくて、経過と解決法が明確になって
  ない。}
  }
}
\end{document}