\documentclass[a4j]{jreport}
\input ../rubricHead.tex
\input ../rubricPresentation.tex
\input ../rubricUnitIHead2.tex
\begin{document}
\setcounter{chapter}{6}
\chapter{JavaScriptの進んだプログラミング}
\changePage{6/6}
今回の演習の目的はJavaScriptにおける変数の取り扱いを少し詳しく学ぶ。
今回の課題はJavaScriptの文法に関するものなので、今までに組んだものを今回
の方法で書き直し、以前の記述方法との違いを考察として加えること。
\begin{itemize}
 \item \texttt{var}と\texttt{let}による変数宣言の違い
 \item JavaScriptにおける変数のスコープルールの確認
 \item 関数の引数の取り扱い
 \item クロージャの理解と使用する利点
 \item 配列のメソッドを利用してプログラムを読みやすくする。
\end{itemize}
課題に\Must と書かれたものを最低行うこと。それ以外の課題は
いくつか選択してよい。
\Probs{変数のスコープ}{演習のビデオ1を見て次の問いに答えよ。}{
 {\Must \texttt{var}と\texttt{let}よる変数の宣言の違いを次の表にまとめよ。
 2つ目以降の項目は各自考えよ。
 \begin{center}
 \begin{tabular}{|m{7zw}|m{16zw}|m{16zw}|}
  \hline
  \multicolumn{1}{|c|}{項目}& \multicolumn{1}{c|}{\texttt{var}}&
        \multicolumn{1}{c|}{\texttt{let}}\\\hline
  同じブロックで同じ変数を2回宣言&\rule{0em}{0.05\textheight}&\\ \hline
  &\rule{0em}{0.06\textheight}&\\ \hline
  &\rule{0em}{0.06\textheight}&\\ \hline
  &\rule{0em}{0.06\textheight}&\\ \hline
  &\rule{0em}{0.06\textheight}&\\ \hline
 \end{tabular} 
 \end{center}
 }{0}
 {\Must 関数の引数として数や配列オブジェクトを渡したとき、それらの仮引数
 の変数を書き換えるサンプルを各自作成し、注意点をまとめよ。}{0}
 {JavaScriptの変数のスコープルールや関数の引数の仕様を他の言語と比較してそ
 の特徴を述べよ。}{0}
 }
% \newpage
\Probs{クロージャ}{演習のビデオ2を見て次の問いに答えよ。}{
 {\Must クロージャ版のアニメーションサイクロイドを実行中に、コンソールで7
				行目や8行目で定義してある変数を参照できるか確認せよ。}{0}
 				{\Must グローバル変数減少利点は何か?
				\begin{itemize}
				 \item \rule{0em}{0.02\textwidth}
				 \item \rule{0em}{0.02\textwidth}
				 \item \rule{0em}{0.02\textwidth}
				\end{itemize}}{0}
 	{関数を定義してその場で実行することの利点は何か。}{0.03}
	{\Must 今まで作成したものからグローバル変数を減らしたものを作成せよ。}{0}			
 }
 \Probs{配列のメソッド}{演習のビデオ3を見て次の問いに答えよ。}{
 {\Must 要素の値が整数である配列に対して、次のことを行うプログラムを作成
				せよ。
				\begin{itemize}
				 \item 各要素を5で割った余りを値に持つ新しい配列
				 \item 奇数である要素だけ選び出す
				\end{itemize}}{0}
 {\Must 配列のメソッドをいくつか利用するサンプルプログラムと用いないで同
	様の処理を行うものを作成し、両者の違いを比較検討せよ。たとえば、次のようなも
	のが考えられる。
	\begin{itemize}
	 \item 大きさが10の配列で1から10の要素を順に持つ配列を作成する
	 \item ラジオボタンまたはプルダウンメニューの作成
	\end{itemize}
	}{0}
 {今までに作成したプログラムで配列のメソッドを使うものに書き直せ。}{0}
 }
%\newpage
\RubricN{第7回(6/6)}{ノートの内容}{
\GradeLegend
}
{
{課題1-1}{10}
{
  {\texttt{var}と\texttt{let}による変数の宣言の違いが例とともに十分にあ
  る。}
}
{
  {同一ブロックにおける\texttt{var}と\texttt{let}による変数の宣言が2回あ
  り違いの説明がある。}
  {入れ子になったブロックにおける\texttt{var}と\texttt{let}による変数の
  宣言がともにあり、違いの説明がある。}
  {入れ子になったブロックにおける\texttt{var}と\texttt{let}による
  内側でのブロックが終了した後の変数の値の違いの説明がある。}
}
{
  {同一ブロックにおける\texttt{var}と\texttt{let}による変数の宣言が2回な
  いか、違いの説明がないか間違っている。}
  {入れ子になったブロックにおける\texttt{var}と\texttt{let}による変数の
  宣言がないか、違いの説明がないか間違っている。}
  {入れ子になったブロックにおける\texttt{var}と\texttt{let}による
  内側でのブロックが終了した後の変数の値の違いの説明がないか間違っている。}
}
{\ResultA}
{課題1-2}{10}
{
  {引数のデータ型が異なる関数を定義し、関数内で仮引数の値を変更するサン
  プルを作成している。}
  {作成した関数をコンソールから動作を十分にチェックしていて考察が正しい。}
}
{
  {引数のデータ型が異なる関数を定義し、関数内で仮引数の値を変更するサン
  プルを作成していが、変更する部分が少し足りない。}
  {作成した関数のコンソールからのチェックが少し足りないか考察が不
  十分である。}
}
{
  {引数のデータ型が異なる関数を定義し、関数内で仮引数の値を変更するサン
  プルの作成が足りない。}
  {定義した関数内で仮引数を変更する部分が足りない。}
  {作成した関数のコンソールからのチェックが足りない。}
  {考察が不十分である。}
}
{\ResultA}
{課題1-3}{10}
{
  {他の言語と変数の宣言の比較が十分なされている。}
  {他の言語と変数のスコープルールの比較が十分なされている。}
  {他の言語と関数のスコープルールの比較が十分なされている。}
}
{
  {他の言語と変数の宣言の比較が\texttt{var}と\texttt{let}でともになされ
  ていない。}
  {他の言語と変数のスコープルールの比較が\texttt{var}と\texttt{let}でともになされ
  ていない。}
  {他の言語と関数のスコープルールの比較がローカルとグローバルの一方でし
  かなされていない。}
  {同一関数名の定義ができるかどうかの説明が不十分である。。}
}
{
  {他の言語と変数の宣言の比較がないか、不十分である。}
  {他の言語と変数のスコープルールの比較がない。}
  {他の言語と関数のスコープルールの比較がない。}
  {同一関数名の定義ができるかどうかの項目がない。。}
}
{\ResultEI}
{課題2-1}{10}
{
  {アニメーションの途中でコンソールから変数の内容を
  \texttt{console.log()}を用いて出力している。}
  {ブロックレベルが異なる変数をチェックしている。}
  {関数についてもチェックしている。}
}
{
  {アニメーションの終了後にコンソールから変数の内容を
  \texttt{console.log()}を用いて出力している。}
  {ブロックレベルが異なる変数をチェックしている。}
  {関数についてチェックしていない。}
}
{
  {コンソールから変数の内容を
  \texttt{console.log()}を用いて出力している。}
  {ブロックレベルが異なる変数をチェックしている。}
  {図がアニメーションが途中になっていないか、途中であることがわからない。}
}
{\ResultEFI}
{課題2-2\newline2-3}{10}
{
  {グローバル変数減少の利点について十分な説明がある。}
  {即時実行関数の利点について十分な説明がある。}
  {グローバル変数減少の方法に関して他の言語との比較がある。}
}
{
  {グローバル変数減少の利点の開発側からの視点がある。}
  {グローバル変数減少の利点のライブラリー利用者側からの視点がある。}
  {即時実行関数の利点について十分な説明がある。}
  {グローバル変数減少の方法に関して他の言語との比較がない。}
}
{
  {グローバル変数減少の利点の開発側からの視点がない。}
  {グローバル変数減少の利点のライブラリー利用者側からの視点がない。}
}
{\ResultEI}
{課題2-4}{10}
{
  {今までの課題でグローバル変数をすべてなくしたものに書き直している。}
  {書き直しの方針について十分な説明がある。}
}
{
  {今までの課題でグローバル変数をほとんどなくしたものに書き直している。}
  {書き直しの方針について説明がある。}
}
{
  {今までの課題でグローバル変数をなくしかたが不十分であるか、全くしてい
  ない。}
  {書き直しの方針について説明がない。}
}
{\ResultA}
{課題3-1}{10}
{
  {配列のメソッドを的確に用いて2つのプログラムを作成している。}
  {いろいろな場合について作成したプログラムをチェックしている。}
}
{
  {\texttt{map()}を用いて5で割った余りの配列を正しく作成している。}
  {\texttt{filter()}を用いて奇数である要素を選び出している。}
  {処理される配列がチェックにふさわしい。}
}
{
  {\texttt{map()}なしで5で割った余りの配列を作成している。}
  {\texttt{filter()}なしで奇数である要素を選び出している。}
  {処理される配列がチェックにふさわしくない。}
}
{\ResultEI}
{課題3-2}{10}
{
  {配列のメソッドを利用するものと利用しないものを
	正しく作成している。}
  {利用している配列のメソッドの種類が十分にある。}
  {配列のメソッドを利用する場合としない場合の違いを比較検討している。}
}
{
  {配列のメソッドを利用するものと利用しないものを作成している。}
  {利用している配列のメソッドの種類が2つしかない。}
  {配列のメソッドを利用する場合としない場合の比較検討が少し不十分
  である。}
}
{
  {配列のメソッドを利用するものと利用しないものを作成していないか、利用
  の仕方が間違っている。}
  {利用している配列のメソッドの種類が1つ以下である。}
  {配列のメソッドを利用する場合としない場合の比較検討がないか不十分
  である。}
}
{\ResultEI}
{課題3-3}{10}
{
  {今までの課題で必要なところを配列のメソッドですべて書き直している。}
  {書き直しの方針について十分な説明がある。}
}
{
  {今までの課題で必要なところを配列のメソッドでほとんど書き直している。}
  {書き直しの方針について十分な説明がある。}
}
{
  {今までの課題で必要なところを配列のメソッドでの書き直しが不十分である
  かほとんどしていない。}
  {書き直しの方針について十分な説明がない。}
}
{\ResultA}
}
\rublicPresenP{第7回(6/6)}

\end{document}