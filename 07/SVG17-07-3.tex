\input ../beamerHead.tex
\TITLE{7}{3}{配列のメソッドの利用}{6/6}
\begin{document}
\frame{\maketitle}
\section{配列のメソッド}
\begin{frame}[containsverbatim]
 \frametitle{配列のメソッド(1)}
 \begin{itemize}
	\item \JSKey{Array()}は引数の大きさを指定して新しい配列を作成する。
				各要素は\JSKey{undefined}で初期化される。
	\item \JSKey{fill()}は引数の値で配列の要素をすべてその値に設定する。
 \end{itemize}
\begin{Verbatim}
>A=Array(5);
(5) [undefined × 5]
>A.length;
5
>A.fill(1);
(5) [1, 1, 1, 1, 1]
\end{Verbatim}
\end{frame}
\begin{frame}[containsverbatim]
 \frametitle{配列のメソッド(2)}
\begin{itemize}
 \item 通常、配列の各要素について操作をするためには\JSKey{for}文などのルー
 プで処理を行う。
 \item JavaScriptでは\JSKey{for}文の代わりになるメソッドがいくつか定義さ
       れている。それぞれのメソッドは引数に関数を必要とする。
       \begin{itemize}
        \item 引数に渡される関数には最大で3つの引数を渡される。
        \item 一つ目の引数は配列の要素
        \item 2つ目の引数は配列のインデックス
        \item 3つ目の引数はメソッドが適用される配列自身である。\\
              これを利用すると元来の配列の要素を変更できる。
       \end{itemize}
 \item 要素の値が\JSKey{undefined}のところは実行されない。
\end{itemize}
\end{frame}
\begin{frame}[containsverbatim]
 \frametitle{配列のメソッド(3)--\JSKey{forEach(func)}}
 \begin{itemize}
  \item 与えられた関数を配列の各要素に対して実行
  \item 与えられた関数の戻り値は無視される。
  \item 途中でループの処理を中断できない。
  \end{itemize}
\end{frame}
\begin{frame}[containsverbatim]
 \frametitle{配列のメソッド(3)--\JSKey{forEach(func)実行例}}

\begin{Verbatim}[fontsize=\tiny]
 >A = [1,2,3,4];
(4) [1, 2, 3, 4]
>A[5]=5;
5
>A;
(6) [1, 2, 3, 4, undefined × 1, 5]
>A.forEach(function(V,i,Vs){
    Vs[i] = V*V;
    console.log(V*V);
});
1
4
9
16
25
undefined
>A;
[1, 4, 9, 16, undefined × 1, 25]
>undefined*undefined
NaN
\end{Verbatim}
 \begin{itemize}
  \item 配列の各要素を2乗している。
  \item 関数の3番目の引数(元の配列)に代入しているので、実行後、要素の値
 が変化する。
	\item 要素の値が\JSKey{undefined}のところは実行されていない。
 \end{itemize}
\end{frame}
\begin{frame}[containsverbatim]
 \frametitle{配列のメソッド(4)--\JSKey{map(func)}}
 配列の各要素に対して引数の関数を実行し、その戻り値で
 新しい配列を作成する。
 \begin{Verbatim}[fontsize=\tiny]
>A=[1,2,3]
(3) [1, 2, 3]
>A[5]=5
5
>A;
(6) [1, 2, 3, undefined × 2, 5]
>>A.map(function(V){return V*V;});
(6) [1, 4, 9, undefined × 2, 25]
>A;
(6) [1, 2, 3, undefined × 2, 5]
 \end{Verbatim}
 \begin{itemize}
  \item 要素を2乗した配列を作成している。
  \item \JSKey{undefined}の要素はそのまま\JSKey{undefined}
  \item 元の配列は変化していない。
 \end{itemize}
\end{frame}
\begin{frame}[containsverbatim]
 \frametitle{配列のメソッド利用の例(1)}
 ランダムに円を描く
 \LISTN{svg-js-random-rev.svg}{1}{7}{\scriptsize}
\end{frame}
\begin{frame}[containsverbatim]
 \frametitle{配列のメソッド利用の例(2)}
 ランダムに円を描く
 \LISTN{svg-js-random-rev.svg}{8}{last}{\scriptsize}
\end{frame}
\begin{frame}[containsverbatim]
 \frametitle{配列のメソッド利用の例--解説}
\begin{itemize}
 \item 以前のプログラムではグローバル変数であったものがすべてローカルな
			 変数に変更されている(10から11行目)。
 \item 12行目で円を登録する\texttt{g}要素を求めている。
 \item 円を20個登録するために次のようなメソッドチェインを用いている。
			 \begin{enumerate}
				\item 大きさが20の配列を作成(\texttt{Array(20)})
				\item 要素の値はすべて\JSKey{undefined}なので\JSKey{fill}メソッ
							ドを利用して 0 に設定
				\item その配列に\JSKey{map}を利用して\texttt{<circle>}要素を作成
				\item 作成する円の属性は共通のものだけ
			 \end{enumerate}
 \item 登録した円は配列\texttt{Objs}に保存
 \item 一定時間をに呼ばれる関数では配列\texttt{Objs}に対して
			 \JSKey{forEach}メソッドを用いて属性値を変更
\end{itemize}
\end{frame}
\begin{frame}[containsverbatim]
 \frametitle{配列のメソッド(5)}
\begin{itemize}
 \item \JSKey{every(func)}\\配列の各要素に引数の関数を実行し、そ
       の戻り値がすべて\JSKey{true}のとき、\JSKey{true}を返す。途中で
       \JSKey{false}になると実行が打ち切られ、\JSKey{false}を返す。
 \item \JSKey{some(func)}\\配列の各要素に引数の関数を実行し、そ
       の戻り値がすべて\JSKey{false}のとき、\JSKey{false}を返す。途中で
       \JSKey{true}になると実行は打ち切られ、\JSKey{true}を返す。
 \item \JSKey{filter(func)}\\配列の各要素に引数の関数を実行し、そ
       の戻り値が\JSKey{true}になるものだけを集めた新しい配列を返す。
\end{itemize} 
\end{frame}
\begin{frame}[containsverbatim]
 \frametitle{配列のメソッド(6)--\JSKey{reduce}と\JSKey{reduceRight}}
 \JSKey{reduce}\\
 \begin{itemize}
	\item 引数は処理をする関数とオプションの引数
	\item オプションの引数は処理を開始するときの初期値
	\item 処理をする関数は4つの引数を取る。
				\begin{itemize}
				 \item 第1引数はそれまでの計算結果\\オプションの引数があるときは開始時の
							 値がこれと同じになる
				 \item 第2引数は現在の要素の値
				 \item 第3引数はインデックス
				 \item 第4引数は繰り返し」を行う配列
				\end{itemize}
 \end{itemize}
 \JSKey{reduceright}は配列の要素の最後から先頭に向かって処理される
\end{frame}
 \begin{frame}[containsverbatim]
	\frametitle{実行例}
\begin{Verbatim}
>A=[3,5,4,7]
(4) [3, 5, 4, 7]
>A.reduce(function(x,v){return x+v;});//初期値がないときは適当に処理
19   //要素の総和
>A.reduce(function(x,v){return x+v;},"");//初期値をから文字列に設定
"3547"    //文字列として繋がれる
>A.reduceRight(function(x,v){return x+v;},"");
"7453"   //reduceRight では並び順が逆になる
\end{Verbatim}
 \end{frame}
\begin{frame}[containsverbatim]
 \frametitle{やってみよう}
 \begin{itemize}
	\item 要素の値が整数である配列に対して、次のことを行うプログラムを作成
				せよ。
				\begin{itemize}
				 \item 各要素を5で割った余りを値に持つ新しい配列
				 \item 奇数である要素だけ選び出す
				\end{itemize}
	\item 大きさが10の配列で1から10の要素を順に持つ配列を作成せよ
	\item 今までに作成したものを配列のメソッドを利用して書き直せ
 \end{itemize}
\end{frame}
\end{document}
\begin{frame}[containsverbatim]
 \frametitle{}
\end{frame}
