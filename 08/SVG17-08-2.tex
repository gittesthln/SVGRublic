\input ../beamerHead.tex
\TITLE{8}{1}{JSON}{6/13}
\begin{document}
\frame{\maketitle}
%\frame{\tableofcontents}
\section{JSON}
\begin{frame}[containsverbatim]
 \frametitle{JSONとは}
 JSONの情報は\texttt{www.json.org/json-ja.html}で得られる。
 \begin{itemize}
  \item  JSONは JavaScript Object Notation の略
  \item JavaScriptのオブジェクトリテラルを定義する文字列の形式をまねたも
				の
				\begin{itemize}
				 \item 全体を\texttt{\{}と\texttt{\}}で囲む
				 \item 各項目はキーと値のペアで、間を\texttt{,}でつなぐ。
				 \item キーは文字列で、\texttt{"}で挟む%"
				 \item 値はオブジェクト、配列、数、文字列、\texttt{true}、
							 \texttt{true}、\texttt{false}、\texttt{null}が可能
				\end{itemize}
 \end{itemize}
 この定義から無条件にJavaScriptのオブジェクトリテラルがそのままJSON形式
 として正しいとは限らない。
\end{frame}
\begin{frame}[containsverbatim]
 \frametitle{}
 \REF{160}
\end{frame}
\end{document}
\begin{frame}[containsverbatim]
\frametitle{}
\end{frame}
