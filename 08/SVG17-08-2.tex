\input ../beamerHead.tex
\TITLE{8}{1}{JSON}{6/13}
\renewcommand{\theFancyVerbLine}{\footnotesize\arabic{FancyVerbLine}}
\begin{document}
\frame{\maketitle}
%\frame{\tableofcontents}
\section{JSON}
\begin{frame}[containsverbatim]
 \frametitle{JSONとは}
 JSONの情報は\texttt{www.json.org/json-ja.html}で得られる。
 \begin{itemize}
  \item  JSONは JavaScript Object Notation の略
  \item JavaScriptのオブジェクトリテラルを定義する文字列の形式をまねたも
				の
				\begin{itemize}
				 \item 全体を\texttt{\{}と\texttt{\}}で囲む
				 \item 各項目はキーと値のペアで、間を\texttt{,}でつなぐ。
				 \item キーは文字列で、\texttt{"}で挟む%"
				 \item 値はオブジェクト、配列、数、文字列、\texttt{true}、
							 \texttt{true}、\texttt{false}、\texttt{null}が可能
				\end{itemize}
 \end{itemize}
 この定義から無条件にJavaScriptのオブジェクトリテラルがそのままJSON形式
 として正しいとは限らない。
\end{frame}
\begin{frame}[containsverbatim]
 \frametitle{JSONオブジェクトの利用法}
 \REF{160}
\begin{Verbatim}[numbers=left, fontsize=\scriptsize,numbersep=1pt,
	commandchars=\\\{\},
	codes={\catcode`$=3\catcode`^=7}]
>A=\{x:10,y:20,add:function()\{return(this.x+this.y);\}\}
Object \{x: 10, y: 20\}
>A.add()
30
>B=JSON.stringify(A);
"\{"x":10,"y":20\}"
>typeof B;
"string"
>JSON.parse(B);
$\blacktriangledown$ Object
	x: 10
	y: 20
	$\blacktriangleright$__proto__: Object
\end{Verbatim}
\end{frame}
\begin{frame}[containsverbatim]
 \frametitle{JSONオブジェクトの利用法(解説)}
\begin{itemize}
 \item 1行目でオブジェクトを定義。メソッドとして\Verb+add()+が
			 あり、2つのメンバー\Verb+x+ と\Verb+y+ を加えた値を返す(3行
			 目)
 \item 5行目でオブジェクトをJSON形式に変換。この中にはメソッド
			 の\Verb+add+が含まれていない。
 \item 7行目では変換されたものが文字列であることを確認している。
 \item 9行目では変換後の文字列をオブジェクトに変換している。
\end{itemize}
\end{frame}
\begin{frame}[containsverbatim]
 \frametitle{Storageに配列データを保存(1)}
 \LISTN{pinna-storage-rev2.js}{1}{12}{\scriptsize}
\end{frame}
\begin{frame}[containsverbatim]
 \frametitle{Storageに配列データを保存(2)}
 \LISTN{pinna-storage-rev2.js}{13}{last}{\scriptsize}
\end{frame}
\end{document}
\begin{frame}[containsverbatim]
\frametitle{}
\end{frame}
