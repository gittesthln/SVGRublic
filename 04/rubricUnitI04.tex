\documentclass[a4j]{jreport}
\input ../rubricHead.tex
\input ../rubricPresentation.tex
\input ../rubricUnitIHead.tex
\begin{document}
\setcounter{chapter}{3}
\chapter{イベント処理入門}
\changePage{5/16}
今回の演習の目的は次の通りである。
\begin{itemize}
 \item 使用しているブラウザにおける「開発者ツール」の使い方
 \item 簡単なイベント処理をJavaScriptで処理すること
 \item イベント処理関数をつける要素の位置の理解
\end{itemize}
今回からは課題に\Must と書かれたものを最低行うこと。それ以外の課題は
いくつか選択してよい。
\Probs{「開発者ツール」の使い方\Must}{演習のビデオ1を見て次の問いに答えよ。}{
{使用しているブラウザと「開発者ツール」の開き方を記せ。\\
ブラウザ名:\rule{0em}{0.7cm}\\開き方}{0.03}
{「Elements」タブを開き、今までに作成したSVGファイルに対して要素の属性を
直接変える。\ignorespaces}{-0.03}
{「Console」タブを開き、コンソールへの入力位置を確認する。}{0}
{コンソールで適当な算術式($2+3$など)を入力して結果が表示されることを確認
する。}{0}
{SVGファイルを開き、
\newline\texttt{document.getElementsByTagName("polyline");}\newline を実
行し、その結果を報告する。\texttt{polyline}のところは該当する要素名に変
えること}{0}
}
\Probs{クリックイベントの処理}{演習のビデオ2を見て次の問いに答えよ。イベ
ント処理関数の設定は\texttt{window.onload}内で行うこと。}{
{クリックされた円の塗りつぶし以外の属性を表示する。}{0}
{\Must 図形の種類をを変えて、クリックしたとき、複数の属性値を表示させる。文字
列の表示はテンプレートリテラルの形式を用いること}{0}
{\Must\texttt{alert}の代わりに\texttt{console.log}を用いると開発者ツールの
Consoleに表示される。\texttt{alert}を使用した場合との違いを考察せよ。}{0}
{2種類以上の図形を表示させ、その上でクリックしたときに要素名が表示される
ようにプログラムしなさい。}{0}
{\Must 正方形をいくつか置いてクリックすると移動するようにプログラムしなさい。}{0}
{正方形をいくつか置いてクリックすると色が変化するようにプログラムしなさい。}{0}
{異なる種類の要素を置いてクリックすると位置が変化するようにプログラムしなさい。}{0}
}
\Probs{イベント処理をつける場所}{演習のビデオ3の前半部「クリックした位置
に円を移動」を見て次の問いに答えよ。}
{
{\Must 画面全体を覆う長方形の属性\texttt{fill}を\texttt{none}にしたら動作に変
化があるか、その理由も含めて答えよ。\\
変化: ある ない(該当する方をまるで囲め)\\
理由:}{0.02}
{\Must 円に対してもイベント処理関数を登録してその上をクリックしたときに移動す
るようにせよ。}{0}
{\Must 円の上をクリックしたら色が変化するようにせよ。}{0}
{2つ以上の要素を置き、要素以外のところでクリックしたら最後にクリッ
        クした要素がクリックした位置に移動するようにせよ。初期値は適
        当に設定してよい。}{0}
{\Must 円の上をクリックすると円の色を表示(3)のリストを実行し、動作することを確
認すること。}{0}
{前問と同じリストに対し、16行目の\texttt{g}を取り除いて、\texttt{svg}要
素の属性\texttt{id}を\texttt{Canvas}にすると正しく動くか確認する。}{0}
{「クリックした位置に円を移動」において、画面全体を覆う長方形なしで、円
上も含めてクリックした位置を円の中心にするようにできるか検討せよ。}{0}
}
\Probs{ドラッグ処理}{演習のビデオ3の後半部「マウスのドラッグを処理」を見
て次の問いに答えよ。}
{
{\Must ビデオ内の「マウスのドラッグを処理」を実行し、気になる点を記せ。
\begin{itemize}
 \item \rule{0em}{0.5cm}
 \item \rule{0em}{0.7cm}
\end{itemize}}{0.0}
{\Must ビデオ内の「マウスのドラッグを処理(改良版)」を実行し、前問の気に
なる点が修正されているか述べよ。さらに、DOMツリーが変化しているか確
認せよ}{0}
{\Must 正方形をドラッグするようにせよ}{0}
{通常、Windows 上でアイコンなどをドラッグするときと、ここでのドラッグす
るときの相違点を述べ、それを改善せよ。
\begin{itemize}
 \item \rule{0em}{0.5cm}
 \item \rule{0em}{0.7cm}
\end{itemize}}{0}
{円と正方形の図形に対し、1種類のイベント処理関数でする方法を考えよ。
			 図形の種類が増えてもイベント処理関数に手を付けないようにするには
			 どのような方法があるか検討せよ。\\
       検討した点:}{0}
       }
\Rubric{第4回(5/16)}{ノートの内容}{
今回からJavaScriptによるプログラミングが始まる。細かい文法の説明を特には
しないので今までのプログラミング言語と比較して違いに気を付けること。
\newline
項目の最後の文字は次に示す項目の評価である。
{\bfseries リ}(プログラム等のリスト)、{\bfseries 説}(プログラ
ム説明が手書きまたは印刷である)、{\bfseries 図}(結果のキャプチャ画面)、
{\bfseries 考}( 考察が手書きまたは印刷である)を意味し、次の記号で評価を
示す。
$\times$(不備またはない)、$\triangle$(もう一息)、$\bigcirc$(良い)、
$\circledcirc$(大変良い)
}
{{課題1}{20}
{
  {使用中のブラウザと「開発者ツール」の開き方\ResultFI}
  {SVGファイルに対して要素の属性を
  直接変えた結果に前後の図に開発者ツールの「Elements」タブが表示\ResultA}
  {開発者ツールのコンソールで直接、簡単な算術式や
  \texttt{document}\newline\texttt{.getElementsByTagName}を実行\ResultA}
}
{
  {使用中のブラウザと「開発者ツール」の開き方の図または考察がない\ResultFI}
  {SVGファイルに対して要素の属性を
  直接変えた結果に前後の図に開発者ツールの「Elements」タブがない\ResultA}
  {開発者ツールのコンソールで簡単な算術式や
  \texttt{document}\newline\texttt{.getElementsByTagName}を実行が一部な
  い\ResultA}
}
{
  {使用中のブラウザと「開発者ツール」の開き方の図と考察がないか不十分\ResultFI}
  {SVGファイルに対して要素の属性を
  直接変えた結果に前後の図のいずれかがないか開発者ツールの「Elements」タ
  ブがなく不十分\ResultA}
  {開発者ツールのコンソールで簡単な算術式や
  \texttt{document}\newline\texttt{.getElementsByTagName}を実行がな
  い\ResultA}
}
 {課題2}{30}
 {
 {\texttt{window.onload}内でイベント処理関数を
 登録し、クリック時に円の塗りつぶし以外の属性を表示している。\ResultA}
 {\Must 図形の種類を変えている。クリック時に複数の属性値をテンプレー
 トリテラルを用いて表示している。\ResultA}
 {\Must\texttt{console.log}を用いた実行結果があ
 り、\texttt{alert}を使用した場合との違いがある。 \ResultFI}
 {\Must 正方形をいくつか置いてクリックすると移動する。\ResultA}
 {正方形をいくつか置いてクリックすると色が変化する。\ResultA}
 {異なる種類の要素があり、それらの図形のクリックする前後の位置が変化を示す
 前後の図が十分ある。\ResultA}
 }
 {
 {\texttt{window.onload}内でイベント処理関数を
 登録していない。クリック時に円の塗りつぶし以外の属性を表示している。\ResultA}
 {\Must 図形の種類を変えて、クリック時の複数に属性値をテンプレー
 トリテラルを一部しか用いないで表示している。\ResultA}
 {\Must\texttt{console.log}を用いた実行結果があ
 り、\texttt{alert}を使用した場合との違いの考察が不十分 \ResultFI}
 {\Must 単独の正方形を置いてクリックすると移動する。\ResultA}
 {単独の正方形しか置いていないが、クリック前後の色の変化の図があ
 り、色が変化している。\ResultA}
 {異なる種類の要素に対してクリックする前後の位置が変化を示す
 図前後の図が少し足りない。\ResultA}
 }
 {
 {\texttt{window.onload}内でイベント処理関数を
 登録していない。クリック時に円の塗りつぶし以外の属性を表示していない。\ResultA}
 {\Must 図形の種類を変えていない。クリック時に複数の属性値をテンプレー
 トリテラルを用いて表示していない。\ResultA}
 {\Must\texttt{console.log}を用いた実行結果がな
 い。\texttt{alert}を使用した場合との違いの考察がない。 \ResultFI}
 {\Must 単独の正方形しかない。クリック時に移動動作がおかしい。\ResultA}
 {単独の正方形しか置いていなく、クリック前後の色の変化の図がないか不十分\ResultA}
 {異なる種類の要素がない。クリックする前後の位置が変化を示す
 図が少なすぎるかなく、図形も移動していない。\ResultA}
 }
 {課題3}{25}
 {
 {\Must 画面全体を覆う長方形の属性\texttt{fill}を\texttt{none}にした報告
 があり、考察が正しい。\ResultEI}
 {\Must 円にイベント処理関数を登録して円の上をクリックしたときも移動す
 る図が十分にある。\ResultA}
 {円上をクリックしたときに色が変わる。説明、考察も十分である。\ResultA}
 {要素上以外でクリックしたら最後にクリッ
  クした要素がクリックした位置に移動することを示す図がある。途中の
  動作を示すために\texttt{console.log}を用いている。\ResultA}
 {\Must 円の上をクリックすると「円の色を表示(3)」の動作確認が分かる図が
 十分にある。\ResultA}
 {\texttt{svg}要素にイベント処理を付けたときの動作の確認が十分にある。\ResultA}
 {画面全体を覆う長方形なしで、円上も含めてクリックした位置を円の中心に
 移動する。\ResultA}
 }
 {
 {\Must 画面全体を覆う長方形の属性\texttt{fill}を\texttt{none}にした報告
 があるが、考察が一部正しくない。\ResultEI}
 {\Must 円にイベント処理関数を登録して円の上をクリックしたときも移動す
 る図が少し足りない。\ResultA}
 {要素上以外でクリックしたら最後にクリッ
  クした要素がクリックした位置に移動することを示す図が足りない。途中の
  動作を確認する手段が足りない。\ResultA}
 {\Must 円の上をクリックすると「円の色を表示(3)」の動作確認が分かる図が
 少し足りない。\ResultA}
 {\texttt{svg}要素にイベント処理を付けたときの動作の確認が少し足りない。\ResultA}
 {画面全体を覆う長方形なしで、円上も含めてクリックした位置を円の中心に移
 動することが不十分である。\ResultA}
 }
 {
 {\Must 画面全体を覆う長方形の属性\texttt{fill}を\texttt{none}にした報告
 がないか、考察が正しくない。\ResultEI}
 {\Must 円にイベント処理関数を登録して円の上をクリックしたときも移動す
 る図がないか、非常に足りない。\ResultA}
 {要素上以外でクリックしたら最後にクリッ
  クした要素がクリックした位置に移動することを示す図がないか非常に足りない。途中の
  動作を確認する手段がない。\ResultA}
 {\Must 円の上をクリックすると「円の色を表示(3)」の動作確認が分かる図が
 ないか非常に足りない。\ResultA}
 {\texttt{svg}要素にイベント処理を付けたときの動作の確認が足りない。\ResultA}
 {画面全体を覆う長方形なしで、円上も含めてクリックした位置を円の中心に移
 動しない。\ResultA}
 }
 {課題4}{25}
 {
 {\Must ビデオ内の「マウスのドラッグを処理」の動作の気になる点の指摘が正
 しい。\ResultFI}
{\Must ビデオ内の「マウスのドラッグを処理(改良版)」で前問の気に
なる点が修正の確認とDOMツリーの変化の確認が十分になされている。\ResultA}
{\Must 正方形のドラッグの動作が十分である。}
{通常のアイコンのドラッグとの相違点と改善が十分にある。\ResultA}
{図形の種類が増えている。イベント処理関数の処理が図形の種類が増加しても
手を付けないようになっている。\ResultA}
 }
 {
 {\Must ビデオ内の「マウスのドラッグを処理」の動作の気になる点の指摘が少
 し足りない。\ResultFI}
{\Must ビデオ内の「マウスのドラッグを処理(改良版)」で前問の気に
なる点が修正の確認とDOMツリーの変化の確認が十分にできていない。\ResultA}
{\Must 正方形のドラッグの動作がすこしおかしい。}
{通常のアイコンのドラッグとの相違点と改善が不十分である。\ResultA}
{図形の種類が増えていない。イベント処理関数の処理が図形の種類が増加しても
手を付ける必要がある。\ResultA}
 }
 {
 {\Must ビデオ内の「マウスのドラッグを処理」の動作の気になる点の指摘が足りない。\ResultFI}
{\Must ビデオ内の「マウスのドラッグを処理(改良版)」で前問の気に
なる点が修正の確認とDOMツリーの変化の確認ができていない。\ResultA}
{\Must 正方形のドラッグの動作がすこしおかしい。}
{通常のアイコンのドラッグとの相違点と改善が不十分である。\ResultA}
{図形の種類が増えていない。イベント処理関数の処理が図形の種類が増加する
と改良のために手間をかける必要がある。\ResultA}
}
}
\rublicPresenII{第4回(5/16)}

\end{document}